Un metodo viene dichiarato con una variante della dichiarazione di funzione ordinaria in cui viene visualizzato un parametro aggiuntivo prima del nome della funzione.
Il parametro attribuisce la funzione al tipo di quel parametro.
\begin{lstlisting}[frame=single, label={lst:lstlisting5-1.1}]
package geometry

import %*``*\)math%*''*\)

type Point struct{ X, Y float64 }

// Funzione tradizionale
func Distance(p, q Point) float64 {
    return math.Hypot(q.X-p.X, q.Y-p.Y)
}

// stessa cosa, ma come un metodo del tipo Point
func (p Point) Distance(q Point) float64 {
    return math.Hypot(q.X-p.X, q.Y-p.Y)
}
\end{lstlisting}
Il parametro extra \verb|p| è chiamato \textit{ricevitore} del metodo, un'eredità dei primi linguaggi orientati agli oggetti che descrivono il chiamare un metodo come ``inviare un messaggio ad un oggetto''.

Dal momento che il nome del ricevitore sarà spesso utilizzato, è una buona idea scegliere qualcosa di breve e di essere coerenti tra i metodi.
Una scelta comune è la prima lettera del nome del tipo, come \verb|p| per \verb|Point|.

In una chiamata di metodo, l'argomento del ricevitore appare prima del nome del metodo.
Questo corrisponde alla dichiarazione, in cui il parametro receiver appare prima del nome del metodo.
\begin{lstlisting}[frame=single, label={lst:lstlisting5-1.2}]
p := Point{1, 2}
q := Point{4, 6}
fmt.Println(Distance(p, q)) // chiamata di funzione
fmt.Println(q.Distance(q))  // chiamata del metodo
\end{lstlisting}
Output:
\begin{lstlisting}[language=bash, frame=L, label={lst:lstlisting5-1.3}]
5
5
\end{lstlisting}
Non c'è conflitto tra le due dichiarazioni di funzioni chiamate \verb|Distance|.
La prima dichiara una funzione a livello di pacchetto chiamata \verb|geometry.Distance|.
La seconda dichiara un metodo del tipo \verb|Point|, quindi il suo nome è \verb|Point.Distance|.

L'espressione \verb|p.Distance| è detta \textit{selettore} perché seleziona il metodo \verb|Distance| appropriato per il ricevitore \verb|p| di tipo \verb|Point|.
I selettori sono anche usati per selezionare i campi dei tipi di struct, come in \verb|p.X|.
Poiché i metodi e i campi abitano lo stesso namespace, dichiarare un metodo \verb|X| sul tipo di struct \verb|Point| sarebbe ambiguo e il compilatore lo rifiuterebbe.

Poiché ogni tipo ha il suo spazio dei nomi per i metodi, possiamo usare il nome \verb|Distance| per altri metodi purché appartengano a diversi tipi.
Definiamo un tipo \verb|Path| assegnandoli anche un metodo \verb|Distance|.
\begin{lstlisting}[frame=single, label={lst:lstlisting5-1.4}]
// Un Path collega i punti con linee rette.
type Path []Point

// Distance restituisce la distanza percorsa lungo il percorso.
func (path Path) Distance() float64 {
    sum := 0.0
    for i := range path {
        if i > 0 {
             sum += path[i-1].Distance(path[i])
        }
    }
    return sum
}
\end{lstlisting}
Chiamiamo il nuovo metodo per calcolare il perimetro di un triangolo retto:
\begin{lstlisting}[frame=single, label={lst:lstlisting5-1.5}]
perim := Path{
    {1, 1},
    {5, 1},
    {5, 4},
    {1, 1},
}
fmt.Println(perim.Distance())
\end{lstlisting}
Output:
\begin{lstlisting}[language=bash, frame=L, label={lst:lstlisting5-1.6}]
12
\end{lstlisting}
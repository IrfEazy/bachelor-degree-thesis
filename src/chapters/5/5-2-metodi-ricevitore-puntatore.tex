Poiché la chiamata ad una funzione avvia la copia di ogni valore in input, se una funzione ha bisogno di aggiornare una variabile, o se un argomento è così grande che si vuole evitarne la copia, si deve passare l'indirizzo della variabile usando un puntatore.
Lo stesso vale per i metodi che devono aggiornare la variabile ricevente: devono essere collegati al tipo di puntatore, come \verb|*Point|.
\begin{lstlisting}[frame=single, label={lst:lstlisting5-2.1}]
func (p *Point) ScaleBy(factor float64) {
   p.X *= factor
   p.Y *= factor
}
\end{lstlisting}
Il nome del metodo è \verb|(*Point).ScaleBy|.
Le parentesi sono necessarie;
senza di esse, l'espressione sarebbe analizzata come \verb|*(Point.ScaleBy)|.

In un programma realistico, la convenzione impone che se un qualsiasi metodo di \verb|Point| ha un ricevitore puntatore, allora \textit{tutti} i metodi di \verb|Point| dovrebbero avere un ricevitore puntatore, anche quelli che non ne hanno strettamente bisogno.

I tipi denominati (\verb|Point|) e i puntatori ad essi (\verb|*Point|) sono gli unici tipi che possono apparire in una dichiarazione del ricevitore.
Inoltre, per evitare ambiguità, le dichiarazioni di metodo non sono consentite su tipi nominati che sono essi stessi tipi di puntatore:
\begin{lstlisting}[frame=single, label={lst:lstlisting5-2.2}]
type P *int
func (P) f() { /* ... */ } // compile error: tipo di ricevitore
                           // non valido
\end{lstlisting}
Il metodo (*Point).ScaleBy può essere chiamato fornendo un ricevitore *Point, in questo modo:
\begin{lstlisting}[frame=single, label={lst:lstlisting5-2.3}]
r := &Point{1, 2}
r.ScaleBy(2)
fmt.Println(*r)

p := Point{1, 2}
pptr := &p
pptr.ScaleBy(2)
fmt.Println(p)

t := Point{1, 2}
(&t).ScaleBy(2)
fmt.Println(t)
\end{lstlisting}
Output:
\begin{lstlisting}[language=bash, frame=L, label={lst:lstlisting5-2.4}]
{2, 4}
{2, 4}
{2, 4}
\end{lstlisting}
Ma gli ultimi due casi sono sgraziati.
Fortunatamente, il linguaggio favorisce il codice.
Se il ricevitore \verb|p| è una \textit{variabile} di tipo \verb|Point| ma il metodo richiede un ricevitore \verb|*Point|, possiamo usare questa abbreviazione:
\begin{lstlisting}[frame=single, label={lst:lstlisting5-2.5}]
p.ScaleBy(2)
\end{lstlisting}
e il compilatore eseguirà un \verb|&p| implicito sulla variabile.
Non possiamo chiamare un metodo \verb|*Point| su un ricevitore \verb|Point| non indirizzabile.
\begin{lstlisting}[frame=single, label={lst:lstlisting5-2.6}]
Point{1, 2}.ScaleBy(2) // compile error: non prende l'indirizzo
                       // di un Point literal
\end{lstlisting}
Ma \textit{possiamo} chiamare un metodo \verb|Point| come \verb|Point.Distance| con un ricevitore \verb|*Point|, perché c'è un modo per ottenere il valore dall'indirizzo: basta caricare il valore indicato dal ricevitore.
Il compilatore inserisce un'operazione \verb|*| implicita per il programmatore.
Le seguenti istruzioni sono equivalenti:
\begin{lstlisting}[frame=single, label={lst:lstlisting5-2.7}]
pptr.Distance(q)
(*pptr).Distance(q)
\end{lstlisting}

\subsection{Nil è un valore ricevitore valido}
\label{subsec:nil_e_un_valore_ricevitore_valido}%
Proprio come alcune funzioni permettono i puntatori nil come argomenti, così fanno alcuni metodi per il loro ricevitore, specialmente se \verb|nil| è un valore zero significativo del tipo, come con map e slice.
In questa semplice linked list di numeri interi, \verb|nil| rappresenta la lista vuota:
\begin{lstlisting}[frame=single, label={lst:lstlisting5-2.8}]
// Un IntList %*\textit{è}*\) una linked list di interi.
// Un *IntList nil rappresenta la lista vuota.
type IntList struct {
    Value int
    Tail  *IntList
}

// Sum restituisce la somma degli elementi della lista.
func (list *IntList) Sum() int {
    if list == nil {
        return 0
    }
    return list.Value + list.Tail.Sum()
}
\end{lstlisting}
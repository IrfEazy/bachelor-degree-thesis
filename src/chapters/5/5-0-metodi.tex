Dall'inizio degli anni 90, la programmazione orientata agli oggetti (OOP) è stata il paradigma di programmazione dominante nell'industria e nell'istruzione, e quasi tutti i linguaggi ampiamente utilizzati sviluppati da allora hanno offerto supporto a questo approccio.
Go non fa eccezione.

Anche se non esiste una definizione universalmente accettata di programmazione orientata agli oggetti, per i nostri scopi, un \textit{oggetto} è semplicemente un valore o una variabile che ha metodi, e un \textit{metodo} è una funzione associata a un particolare tipo.
Un programma orientato agli oggetti è un programma che utilizza metodi per esprimere le proprietà e le operazioni di ogni struttura dati in modo che i client non debbano accedere direttamente alla rappresentazione dell'oggetto.


\section{Dichiarazioni dei metodi}
\label{sec:dichiarazioni_dei_metodi}%
\input{chapters/5/5-1-dichiarazioni-metodi}


\section{Metodi con un ricevitore puntatore}
\label{sec:metodi_con_un_ricevitore_puntatore}%
Poiché la chiamata ad una funzione avvia la copia di ogni valore in input, se una funzione ha bisogno di aggiornare una variabile, o se un argomento è così grande che si vuole evitarne la copia, si deve passare l'indirizzo della variabile usando un puntatore.
Lo stesso vale per i metodi che devono aggiornare la variabile ricevente: devono essere collegati al tipo di puntatore, come \verb|*Point|.
\begin{lstlisting}[frame=single, label={lst:lstlisting5-2.1}]
func (p *Point) ScaleBy(factor float64) {
   p.X *= factor
   p.Y *= factor
}
\end{lstlisting}
Il nome del metodo è \verb|(*Point).ScaleBy|.
Le parentesi sono necessarie;
senza di esse, l'espressione sarebbe analizzata come \verb|*(Point.ScaleBy)|.

In un programma realistico, la convenzione impone che se un qualsiasi metodo di \verb|Point| ha un ricevitore puntatore, allora \textit{tutti} i metodi di \verb|Point| dovrebbero avere un ricevitore puntatore, anche quelli che non ne hanno strettamente bisogno.

I tipi denominati (\verb|Point|) e i puntatori ad essi (\verb|*Point|) sono gli unici tipi che possono apparire in una dichiarazione del ricevitore.
Inoltre, per evitare ambiguità, le dichiarazioni di metodo non sono consentite su tipi nominati che sono essi stessi tipi di puntatore:
\begin{lstlisting}[frame=single, label={lst:lstlisting5-2.2}]
type P *int
func (P) f() { /* ... */ } // compile error: tipo di ricevitore
                           // non valido
\end{lstlisting}
Il metodo (*Point).ScaleBy può essere chiamato fornendo un ricevitore *Point, in questo modo:
\begin{lstlisting}[frame=single, label={lst:lstlisting5-2.3}]
r := &Point{1, 2}
r.ScaleBy(2)
fmt.Println(*r)

p := Point{1, 2}
pptr := &p
pptr.ScaleBy(2)
fmt.Println(p)

t := Point{1, 2}
(&t).ScaleBy(2)
fmt.Println(t)
\end{lstlisting}
Output:
\begin{lstlisting}[language=bash, frame=L, label={lst:lstlisting5-2.4}]
{2, 4}
{2, 4}
{2, 4}
\end{lstlisting}
Ma gli ultimi due casi sono sgraziati.
Fortunatamente, il linguaggio favorisce il codice.
Se il ricevitore \verb|p| è una \textit{variabile} di tipo \verb|Point| ma il metodo richiede un ricevitore \verb|*Point|, possiamo usare questa abbreviazione:
\begin{lstlisting}[frame=single, label={lst:lstlisting5-2.5}]
p.ScaleBy(2)
\end{lstlisting}
e il compilatore eseguirà un \verb|&p| implicito sulla variabile.
Non possiamo chiamare un metodo \verb|*Point| su un ricevitore \verb|Point| non indirizzabile.
\begin{lstlisting}[frame=single, label={lst:lstlisting5-2.6}]
Point{1, 2}.ScaleBy(2) // compile error: non prende l'indirizzo
                       // di un Point literal
\end{lstlisting}
Ma \textit{possiamo} chiamare un metodo \verb|Point| come \verb|Point.Distance| con un ricevitore \verb|*Point|, perché c'è un modo per ottenere il valore dall'indirizzo: basta caricare il valore indicato dal ricevitore.
Il compilatore inserisce un'operazione \verb|*| implicita per il programmatore.
Le seguenti istruzioni sono equivalenti:
\begin{lstlisting}[frame=single, label={lst:lstlisting5-2.7}]
pptr.Distance(q)
(*pptr).Distance(q)
\end{lstlisting}

\subsection{Nil è un valore ricevitore valido}
\label{subsec:nil_e_un_valore_ricevitore_valido}%
Proprio come alcune funzioni permettono i puntatori nil come argomenti, così fanno alcuni metodi per il loro ricevitore, specialmente se \verb|nil| è un valore zero significativo del tipo, come con map e slice.
In questa semplice linked list di numeri interi, \verb|nil| rappresenta la lista vuota:
\begin{lstlisting}[frame=single, label={lst:lstlisting5-2.8}]
// Un IntList %*\textit{è}*\) una linked list di interi.
// Un *IntList nil rappresenta la lista vuota.
type IntList struct {
    Value int
    Tail  *IntList
}

// Sum restituisce la somma degli elementi della lista.
func (list *IntList) Sum() int {
    if list == nil {
        return 0
    }
    return list.Value + list.Tail.Sum()
}
\end{lstlisting}


\section{Tipi di composizione per Struct Embedding}
\label{sec:tipi_di_composizione_per_struct_embedding}
\input{chapters/5/5-3-tipi-composizione-struct-embedding}

\section{Valori ed espressioni del metodo}
\label{sec:valori_ed_espressioni_del_metodo}%
\input{chapters/5/5-4-valori-espressioni-metodo}


\section{Incapsulamento}
\label{sec:incapsulamento}%
Una variabile o un metodo di un oggetto viene \textit{incapsulato} se è inaccessibile ai client dell'oggetto.
L'incapsulamento, a volte chiamato \textit{occultamento delle informazioni}, è un aspetto chiave della program- mazione orientata agli oggetti.

Go ha un solo meccanismo per controllare la visibilità dei nomi: gli identificatori con il primo carattere maiuscolo vengono esportati dal package in cui sono definiti, e i nomi con il primo carattere non minuscolo non lo sono.
Lo stesso meccanismo che limita l'accesso ai membri di un package limita anche l'accesso ai campi di una struttura o ai metodi di un tipo.
Di conseguenza, per incapsulare un oggetto, bisogna farne una struct.

La conseguenza di questo meccanismo basato sul nome è che l'unità di incapsulamento è il package, non il tipo come in molti altri linguaggi.
I campi di un tipo di struttura sono visibili a tutto il codice all'interno dello stesso package.
Se il codice appare in una funzione o in un metodo non fa alcuna differenza.

L'incapsulamento offre tre vantaggi.
Innanzitutto, poiché i client non possono modificare direttamente le variabili dell'oggetto, è necessario ispezionare un minor numero di istruzioni per comprendere i possibili valori di tali variabili.

In secondo luogo, nascondere i dettagli di implementazione impedisce ai client di dipendere dalle cose che potrebbero cambiare, il che dà al progettista una maggiore libertà di far evolvere l'implementazione senza rompere la compatibilità API\@.

Il terzo vantaggio dell'incapsulamento, e in molti casi il più importante, è che impedisce ai client di impostare arbitrariamente le variabili di un oggetto.
Poiché le variabili dell'oggetto possono essere impostate solo da funzioni nello stesso pacchetto, l'autore di quel pacchetto può assicurare che tutte queste funzioni mantengano gli invarianti interni dell'oggetto.


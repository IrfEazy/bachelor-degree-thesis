\documentclass[../../thesis.tex]{subfiles}
\begin{document}
    \subsection{Stringhe}\label{subsec:stringhe}
    Una stringa è una sequenza immutabile di byte.
    Le stringhe possono contenere dati arbitrari, inclusi byte con valore \verb"0", ma di solito contengono testo leggibile dall'uomo.
    Le stringhe di testo sono convenzionalmente interpretate come sequenze codificate UTF-8 di punti di codice Unicode (rune).
    \begin{lstlisting}[frame = single,label={lst:lstlisting2.1}]
s := "hello, world"
fmt.Println(len(s))
fmt.Println(s[0], s[7])
    \end{lstlisting}
    Output:
    \begin{lstlisting}[language = bash, frame = L,label={lst:lstlisting2.2}]
12
104 119 ('h' and 'w')
    \end{lstlisting}
    Il tentativo di accedere a un byte al di fuori di questo intervallo si traduce in un \verb"panic":
    \begin{lstlisting}[frame = single,label={lst:lstlisting2.3}]
C := s[len(s)] // panic: indice fuori dal range
    \end{lstlisting}
    L'operazione di \textit{sottostringa} \verb"s[i:j]" produce una nuova stringa composta dai byte della stringa originale che inizia dall'indice \verb"i" e continua fino, ma non include, il byte all'indice \verb"j".
    Il risultato contiene \verb"j-i" byte.
    \begin{lstlisting}[frame = single,label={lst:lstlisting2.4}]
fmt.Println(s[0:5])
    \end{lstlisting}
    Output:
    \begin{lstlisting}[language = bash, frame = L,label={lst:lstlisting2-2.5}]
hello
    \end{lstlisting}
    Uno o entrambi gli operandi \verb"i" e \verb"j" possono essere omessi, nel qual caso i valori predefiniti di \verb"0" (l'inizio della stringa) e \verb"len(s)" (la sua fine) sono assunti, rispettivamente.
    \hfill \vspace{12pt}

    L'operatore \verb"+" crea una nuova stringa concatenando due stringhe.
    Le stringhe possono essere confrontate con operatori di confronto come \verb"==" e \verb"<"; il confronto viene fatto byte per byte, quindi il risultato è l'ordinamento lessicografico naturale.
    \hfill \vspace{12pt}

    I valori di stringa sono immutabili: la sequenza di byte contenuta in un valore di stringa non può mai essere modificata, anche se ovviamente possiamo assegnare un nuovo valore a una \textit{variabile} di stringa.
    \hfill \vspace{12pt}

    Immutabilità significa che è sicuro per due copie di una stringa condividere la stessa memoria sottostante, il che rende conveniente copiare le stringhe di qualsiasi lunghezza.
    Allo stesso modo, una stringa \verb"s" e una sottostringa \verb"s[7:]" possono tranquillamente condividere gli stessi dati, quindi anche l'operazione di sottostringa è economica.
    Nessuna nuova memoria viene allocata in entrambi i casi.
    \subfile{2-stringhe-subsection/1-stringhe-costanti}
    \subfile{2-stringhe-subsection/2-unicode}
    \subfile{2-stringhe-subsection/3-utf-8}
    \subfile{2-stringhe-subsection/4-stringhe-byte-slices}
\end{document}
\documentclass[../../../thesis.tex]{subfiles}
\begin{document}
    \subsubsection{UTF-8}
    UTF-8 è una codifica a lunghezza variabile dei punti di codice Unicode (unità di byte).
    UTF-8 è stato inventato da Ken Thompson e Rob Pike, due dei creatori di Go, ed è ora uno standard Unicode.
    Usa tra 1 e 4 byte per rappresentare ogni runa, ma solo 1 byte per i caratteri ASCII. I bit di ordine elevato del primo byte della codifica di una runa indicano quanti byte seguono.
    Un \verb"0" di alto ordine indica un ASCII a 7 bit, dove ogni runa prende solo 1 byte, quindi è identico all'ASCII convenzionale.
    Un ordine alto \verb"110" indica che la runa prende 2 byte;
    il secondo byte inizia con \verb|10|.
    Le rune più grandi hanno codifiche analoghe.
    \hfill \vspace{12pt}

    Una codifica a lunghezza variabile preclude l'indicizzazione diretta per accedere al carattere \textit{n}-esimo di una stringa, ma UTF-8 ha molte altre proprietà apprezzabili a compensare tale difetto.
    La codifica è compatta, compatibile con ASCII e auto-sincronizzante: è possibile trovare l'inizio di un carattere eseguendo il backup di non più di tre byte.
    Può essere decodificato da sinistra a destra senza alcuna ambiguità o necessità di un lookahead.
    La codifica di nessuna runa è una sottostringa di qualsiasi altra, o di una sequenza di altre, quindi una runa è individuabile semplicemente cercando i suoi byte, senza preoccuparsi del contesto precedente.
    \hfill \vspace{12pt}

    Il pacchetto \verb"unicode" fornisce funzioni per lavorare con singole rune (come distinguere le lettere dai numeri, o convertire una lettera maiuscola in una minuscola), e il pacchetto \verb"unicode/utf8" fornisce funzioni per codificare e decodificare le rune come byte usando UTF-8.
    \hfill \vspace{12pt}

    Molti caratteri Unicode sono difficili da digitare su una tastiera o da distinguere visivamente da caratteri simili;
    alcuni sono addirittura invisibili, gli escape Unicode in Go ci permettono di specificarli in base al loro valore numerico.
    Ci sono due forme, \verb"\u"\textit{hhhhhh} per un valore a 16 bit e \verb"\U"\textit{hhhhhhhh} per un valore a 32 bit, dove ogni \textit{h} è una cifra esadecimale.
\end{document}
\documentclass[../../../thesis.tex]{subfiles}
\begin{document}
    \subsubsection{Unicode}
    Unicode (\verb"unicode.org") raccoglie tutti i caratteri di tutti i sistemi di scrittura del mondo, oltre a accenti e altri segni diacritici, codici di controllo come tabulazione e ritorno a capo, e un sacco di esoterica, e assegna a ciascuno un numero standard chiamato \textit{punto di codice Unicode} o, nella terminologia Go, una \textit{runa}.
    \hfill \vspace{12pt}

    Il tipo di dati naturale per contenere una singola runa è \verb|int32|.
    Potremmo rappresentare una sequenza di rune come una sequenza di valori \verb"int32".
    In questa rappresentazione, chiamata UTF-32 o UCS-4, la codifica di ogni punto di codice Unicode ha la stessa dimensione, 32 bit.
    Questo è semplice e uniforme, ma utilizza molto più spazio del necessario poiché la maggior parte del testo leggibile dal computer è in ASCII, che richiede solo 8 bit (quindi 1 byte) per carattere.
\end{document}
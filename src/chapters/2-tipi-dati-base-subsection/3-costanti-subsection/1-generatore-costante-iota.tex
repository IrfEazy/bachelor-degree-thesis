\documentclass[../../../thesis.tex]{subfiles}
\begin{document}
    \subsubsection{Il generatore costante iota}
    Una dichiarazione \verb"const" può usare il \textit{generatore costante} \verb"iota", che viene usato per creare una sequenza di valori correlati senza specificare esplicitamente ciascuno di essi.
    In una dichiarazione \verb"const", il valore di \verb"iota" inizia a zero e aumenta di uno per ogni elemento della sequenza.
    \hfill \vspace{12pt}

    Tipi di questo tipo sono spesso chiamati \textit{enumerazioni}, o \textit{enum} in breve.
    \begin{lstlisting}[frame = single, label = {lst:lstlisting2-3-1.1}]
type Weekday int

const (
    Sunday Weekday = iota
    Monday
    Tuesday
    Wednesday
    Thursday
    Friday
    Saturday
)
    \end{lstlisting}
    Questo dichiara \verb"Sunday" essere 0, \verb"Monday" essere 1, e così via.
    \hfill \vspace{12pt}

    Possiamo usare \verb"iota" anche in espressioni più complesse, per esempio possiamo assegnare ai 5 bit più bassi di un intero senza segno un nome distinto e un'interpretazione booleana:
    \begin{lstlisting}[frame = single, label = {lst:lstlisting2-3-1.2}]
type Flags uint

const (
    FlagUp Flags = 1 << iota // %*{\textit{è}}*) up
    FlagBroadcast            // supporta la capacit%*{\textit{à}}*) di accesso broadcast
    FlagLoopback             // %*\textit{è}*) un'interfaccia di loopback
    FlagPointToPoint         // appartiene a un collegamento punto a punto
    FlagMulticast            // supporta la capacit%*{\textit{à}}*) di accesso multicast
)
    \end{lstlisting}
    Così come incrementa \verb"iota", ad ogni costante viene assegnato il valore di \verb"1 << iota", che valuta le potenze di due, ciascuno corrispondente ad un singolo bit.
\end{document}
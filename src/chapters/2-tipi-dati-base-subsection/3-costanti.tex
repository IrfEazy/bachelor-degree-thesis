\documentclass[../../thesis.tex]{subfiles}
\begin{document}
    \subsection{Costanti}\label{subsec:costanti}
    Le costanti sono espressioni il cui valore è noto al compilatore e la cui valutazione è garantita che si verifichi al momento della compilazione, non al tempo di esecuzione.
    Il tipo sottostante di ogni costante è un tipo di base: booleano, stringa o numero.
    \hfill \vspace{12pt}

    Una dichiarazione \verb"const" definisce valori denominati che sembrano sintatticamente variabili ma il cui valore è costante, il che impedisce cambiamenti accidentali (o nefasti) durante l'esecuzione del programma.
    Per esempio, una costante è più appropriata di una variabile per una costante matematica come \verb"pi", poiché il suo valore non cambierà:
    \begin{lstlisting}[frame = single, label = {lst:lstlisting2-3.1}]
const pi = 3.14159 // approssimato; math.Pi %*\textit{è}*) migliore
    \end{lstlisting}
    Quando una sequenza di costanti è dichiarata come un gruppo, l'espressione a destra può essere omessa per tutti tranne che per il primo del gruppo, il che implica che l'espressione precedente e il suo tipo debbano essere nuovamente usati.
    Ad esempio:
    \begin{lstlisting}[frame = single, label = {lst:lstlisting2-3.2}]
const (
    a = 1
    b
    c = 2
    d
)

fmt.Println(a, b, c, d)
    \end{lstlisting}
    Output:
    \begin{lstlisting}[language = bash, frame = L, label = {lst:lstlisting2-3.3}]
1 1 2 2
    \end{lstlisting}
    Tuttavia, questo non è molto utile nel caso la copia implicita dell'espressione del lato destro valuti sempre la stessa cosa, ma se potesse variare?
    Otteniamo \verb"iota".
    \subfile{3-costanti-subsection/1-generatore-costante-iota}
\end{document}
\documentclass[../thesis.tex]{subfiles}
\begin{document}
    \newpage


    \section{Tipi compositi}\label{sec:tipi-compositi}
    I tipi base fanno da atomi per le strutture di dati in un programma Go. I tipi \textit{compositi} sono quindi le molecole create combinando i tipi di base in vari modi.
    Ne esistono di quattro tipi: array, slice, map e struct.
    \hfill \vspace{12pt}

    Gli array e le struct sono tipi \textit{aggregati}; i loro valori sono concatenazioni di altri valori in memoria.
    Gli array sono omogenei (i loro elementi hanno tutti lo stesso tipo) mentre le struct sono eterogenee.
    Entrambi gli array e le struct sono di dimensioni fisse.
    Al contrario, le slice e le map sono strutture di dati dinamici che crescono con l'aggiunta di valori.
    \subfile{3-tipi-compositi-subsection/1-array}
    \subfile{3-tipi-compositi-subsection/2-slice}
    \subfile{3-tipi-compositi-subsection/3-maps}
    \subfile{3-tipi-compositi-subsection/4-structs}
    \clearpage
\end{document}
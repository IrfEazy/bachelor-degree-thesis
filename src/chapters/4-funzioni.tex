\documentclass[../thesis.tex]{subfiles}
\begin{document}
    \newpage


    \section{Funzioni}\label{sec:funzioni}
    Una funzione ci permette di avvolgere una sequenza di istruzioni come un'unità che può essere chiamata da altrove in un programma, anche più volte.
    Le funzioni permettono di suddividere un grande lavoro in parti più piccole.
    Una funzione nasconde i dettagli della sua implementazione ai suoi utenti.
    Per tutti questi motivi, le funzioni sono una parte critica di tutto il linguaggio di programmazione.
    \subfile{4-funzioni-subsection/1-dichiarazione-di-funzione}
    \subfile{4-funzioni-subsection/3-valori-di-ritorno-multipli}
    \subfile{4-funzioni-subsection/4-errori}
    \subfile{4-funzioni-subsection/5-valori-di-funzione}
    \subfile{4-funzioni-subsection/6-funzioni-anonime}
    \subfile{4-funzioni-subsection/7-funzioni-variadic}
    \subfile{4-funzioni-subsection/8-chiamate-funzione-differita}
    \subfile{4-funzioni-subsection/9-panic}
    \clearpage
\end{document}
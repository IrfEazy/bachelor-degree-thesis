\documentclass[../../../thesis.tex]{subfiles}
\begin{document}
    \subsubsection{Assegnazione di tuple}
    Un'altra forma di assegnazione, nota come \textit{assegnazione di tuple}, permette di assegnare più variabili contemporaneamente.
    Tutte le espressioni di destra vengono valutate prima che una delle variabili venga aggiornata, rendendo questo modulo più utile quando alcune delle variabili appaiono su entrambi i lati dell'assegnazione, come accade, ad esempio, quando si scambiano i valori di due variabili:
    \begin{lstlisting}[frame = single,label={lst:lstlisting1-4-1.1}]
x, y = y, x
a[i], a[j] = a[j], a[i]
    \end{lstlisting}
    o quando si calcola il massimo comune divisore (MCD) di due interi:
    \begin{lstlisting}[frame = single,label={lst:lstlisting1-4-1.2}]
func gcd(x, y int) int {
    for y != 0 {
        x, y = y, x%y
    }
    return x
}
    \end{lstlisting}
    o quando si calcola iterativamente l'\textit{n}-esimo numero di Fibonacci:
    \begin{lstlisting}[frame = single,label={lst:lstlisting1-4-1.3}]
func fib(n int) int {
    x, y := 0, 1
    for i := 0; i < n; i++ {
        x, y = y, x+y
    }
    return x
}
    \end{lstlisting}
\end{document}
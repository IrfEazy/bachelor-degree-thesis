\documentclass[../../../thesis.tex]{subfiles}
\begin{document}
    \subsubsection{Inizializzazione dei pacchetti}
    L'inizializzazione dei pacchetti inizia inizializzando le variabili a livello di pacchetto nell'ordine in cui sono dichiarate, tranne che le dipendenze vengono risolte per prime:
    \begin{lstlisting}[frame = single,label={lst:lstlisting1-5-1.1}]
var a = b + c // a %*\textit{è}*) inizializzato per terzo, a 3
var b = f()   // b %*\textit{è}*) inizializzato per secondo, a 2, chiamando f
var c = 1     // c %*\textit{è}*) inizializzato per primo, a 1

func f() int {
    return c + 1
}
    \end{lstlisting}
    Ogni variabile dichiarata a livello di pacchetto inizia la vita con il valore della sua espressione di inizializzazione, se presente, ma per alcune variabili, come le tabelle di dati, un'espressione di inizializzazione potrebbe non essere il modo più semplice per impostare il suo valore iniziale.
    In tal caso, il servizio della funzione \verb"init" può semplificare il lavoro.
    Ogni file può contenere la funzione \verb"init" la cui dichiarazione è:
    \begin{lstlisting}[label={lst:lstlisting1-5-1.2}]
func init() { codice }
    \end{lstlisting}
    Tali funzioni di \verb"init" non possono essere chiamate o referenziate, perché altrimenti sarebbero funzioni normali.
    All'interno di ogni file, le funzioni di \verb"init" vengono eseguite automaticamente all'avvio del programma, nell'ordine in cui vengono dichiarate.
    \hfill \vspace{12pt}

    L'inizializzazione procede dal basso verso l'alto;
    il pacchetto principale è l'ultimo ad essere inizializzato.
    In questo modo, tutti i pacchetti sono completamente inizializzati prima dell'inizio della funzione principale dell'applicazione.
    In particolare, un pacchetto viene inizializzato seguendo l'ordine delle importazioni nel programma, risolvendo prima le dipendenze, quindi un pacchetto \verb"p" che importa \verb"q" può essere certo che \verb"q" sia completamente inizializzato prima che inizi la sua inizializzazione.
\end{document}
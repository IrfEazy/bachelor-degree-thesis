\documentclass[../../thesis.tex]{subfiles}
\begin{document}
    \subsection{Variabili}\label{subsec:variabili}
    Una dichiarazione di una \verb"var" crea una variabile di un particolare tipo, le dà un nome, e imposta il suo valore iniziale.
    Ogni istruzione ha una forma generale:
    \begin{lstlisting}[label={lst:lstlisting1-3.1}]
var nome tipo = espressione
    \end{lstlisting}
    In particolare, può essere omesso uno tra il \verb"tipo" e \verb"= espressione" ma entrambi non possono mancare nella dichiarazione.
    Se l'espressione è omessa, il valore iniziale è il valore zero del tipo, vale a dire \verb"0" per i numeri, \verb"false" per il booleano, \verb|""| per le stringhe, e \verb"nil" per le interfacce e i tipi di riferimento (slice, puntatori, mappe, canali, funzioni).
    Il valore zero di un tipo aggregato come un array o una struct ha valore zero in tutti i suoi elementi o campi.
    \hfill \vspace{12pt}

    Il meccanismo a valore zero assicura che una variabile abbia sempre un valore ben definito del suo tipo;
    non ci sono variabili non inizializzate in Go. Per esempio,
    \begin{lstlisting}[frame = single,label={lst:lstlisting1-3.2}]
var s string
    \end{lstlisting}
    assegna ad \verb"s" una stringa vuota \verb|""|, così da poterlo utilizzare in seguito senza causare errore o comportamento imprevedibile.
    I programmatori di Go spesso fanno uno sforzo per rendere significativo il valore zero di un tipo più complicato, in modo che le variabili inizino la loro vita in uno stato utile.
    \hfill \vspace{12pt}

    È possibile dichiarare e, facoltativamente, inizializzare un insieme di variabili in una singola dichiarazione, con un elenco di espressioni corrispondenti:
    \begin{lstlisting}[frame = single,label={lst:lstlisting1-3.3}]
var i, j, k int			// int, int, int
var b, f, s = true, 2.3, "four"	// bool, float64, string
    \end{lstlisting}
    Le variabili visibili a livello di pacchetto vengono inizializzate prima dell'avvio del \verb"main", e le variabili locali vengono inizializzate quando si incontrano le loro dichiarazioni mentre la funzione è in esecuzione.\\
    Un insieme di variabili può anche essere inizializzato chiamando una funzione che restituisce più valori:
    \begin{lstlisting}[frame = single,label={lst:lstlisting1-3.4}]
var f, err = os.Open(name)
    \end{lstlisting}
    dove \verb"os.Open" restituisce un file e un errore.
    \subfile{3-variabili-subsection/1-short-variable-declaration}
    \subfile{3-variabili-subsection/2-puntatori}
    \subfile{3-variabili-subsection/3-funzione-new}
    \subfile{3-variabili-subsection/4-durata-variabili}
\end{document}
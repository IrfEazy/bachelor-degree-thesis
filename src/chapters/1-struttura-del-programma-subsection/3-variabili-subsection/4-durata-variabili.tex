\documentclass[../../../thesis.tex]{subfiles}
\begin{document}
    \subsubsection{La durata delle variabili}
    La \textit{durata} della variabile è l'intervallo di tempo durante il quale la variabile esiste durante l'esecuzione del programma.
    La durata di una variabile a livello di pacchetto è uguale all'intera esecuzione del programma.
    Al contrario, le variabili locali hanno una durata dinamica: una nuova istanza viene creata ogni volta che l'istruzione di dichiarazione viene eseguita, e la variabile vive fino a quando non diventa \textit{irraggiungibile}, momento in cui il suo archivio può essere riciclato.
    Anche i parametri e i risultati delle funzioni sono variabili locali;
    vengono creati ogni volta che viene chiamata una funzione invece di un parametro.
    \hfill \vspace{12pt}

    Ad esempio,
    \begin{lstlisting}[frame = single,label={lst:lstlisting1-3-4.1}]
for t := 0.0; t < cycles*2*math.Pi; t += res {
    x := math.Sin(t)
    y := math.Sin(t*freq + phase)
    img.SetColorIndex(size+int(x*size+0.5), size+int(y*size+0.5),
        blackIndex)
}
    \end{lstlisting}
    la variabile \verb"t" viene creata ogni volta che inizia il ciclo \verb"for", e le nuove variabili \verb"x" e \verb"y" vengono create ad ogni iterazione del ciclo.
    \hfill \vspace{12pt}

    Questo discorso diventa importante per il programmatore in Go, dove spesso fa uso di puntatori a oggetti di breve durata all'interno di oggetti di lunga durata, come variabili globali, perché così facendo impedirà al garbage collector di recuperare gli oggetti di breve durata.
\end{document}
\documentclass[../../../thesis.tex]{subfiles}
\begin{document}
    \subsubsection{Puntatori}
    Una variabile è una parte di un archivio contenente un valore.
    Le variabili create dalle dichiarazioni sono identificate da un nome, come \verb"x", ma la maggior parte delle variabili sono identificate solo da espressioni come \verb"x[i]" o \verb"x.f".
    Tutte queste espressioni leggono il valore di una variabile, tranne quando appaiono sul lato sinistro di un'assegnazione, nel qual caso viene assegnato un nuovo valore alla variabile.
    \hfill \vspace{12pt}

    Un valore di un \textit{puntatore} è un \textit{indirizzo} di una variabile.
    Un puntatore è quindi la posizione in cui un valore è memorizzato.
    Non tutti i valori hanno un indirizzo, ma ogni variabile ne ha uno.
    Con un puntatore, è possibile leggere o aggiornare \textit{indirettamente} il valore di una variabile, senza l'uso o la conoscenza del nome di una variabile, sempre ammesso abbia un nome.
    \hfill \vspace{12pt}

    Se una variabile è dichiarata \verb"var x int", l'espressione \verb"&x" (``\verb"x" address") restituisce un puntatore ad una variabile intera, che è un valore di tipo \verb"*int", si legge ``pointer to int".
    Se questo valore è definito come \verb"p", si dirà che ``\verb"p" punta a \verb"x"", o equivalentemente ``\verb"p" contiene l'indirizzo di \verb"x"". La variabile a cui \verb"p" punta è indicata da \verb"*p".
    L'espressione \verb"*p" restituisce il valore di quella variabile, un \verb"int", ma poiché \verb"*p" denota una variabile, allora può anche apparire a sinistra di un'assegnazione, nel qual caso aggiorna la variabile.
    \begin{lstlisting}[frame = single,label={lst:lstlisting1-3-2.1}]
x := 1
p := &x // p, di tipo *int, punta a x
fmt.Println(*p)
*p = 2 // equivalente a x = 2
fmt.Println(x)
    \end{lstlisting}
    Output:
    \begin{lstlisting}[language = bash, frame = L,label={lst:lstlisting1-3-2.2}]
1
2
    \end{lstlisting}
    Il valore zero di un puntatore per ogni tipo è \verb"nil".
    Se \verb"p" punta a una variabile, allora vale sempre \verb"p != nil".
    I puntatori sono comparabili;
    due puntatori sono uguali se e solo se puntano alla stessa variabile o se entrambi sono \verb"nil".
    \begin{lstlisting}[frame = single,label={lst:lstlisting1-3-2.3}]
var x, y int
fmt.Println(&x == &x, &x == &y, &x == nil)
    \end{lstlisting}
    Output:
    \begin{lstlisting}[language = bash, frame = L,label={lst:lstlisting1-3-2.4}]
true false false
    \end{lstlisting}
    Con i puntatori, possiamo ottenere le modifiche alle variabili locali da altre funzioni.
    Per esempio,
    \begin{lstlisting}[frame = single,label={lst:lstlisting1-3-2.5}]
func incr(p *int) int {
    *p++ // incrementa il puntato da p, non cambia p
    return *p
}

func main() {
    v := 1
    incr(&v) // side effect: v vale ora 2
    fmt.Println(incr(&v))
    fmt.Println(v)
}
    \end{lstlisting}
    Output:
    \begin{lstlisting}[language = bash, frame = L,label={lst:lstlisting1-3-2.6}]
3
3
    \end{lstlisting}
\end{document}
\documentclass[../../../thesis.tex]{subfiles}
\begin{document}
    \subsubsection{Short variable declaration}
    All'interno di una funzione, una forma alternativa chiamata \textit{short variable declaration} può essere usata per dichiarare e inizializzare le variabili locali.
    Questa istruzione assume la forma di \verb"nome := espressione" e il tipo di \verb"nome" è determinato dal tipo di \verb"espressione".
    Esempi di brevi istruzioni di variabili sono:
    \begin{lstlisting}[frame = single,label={lst:lstlisting1-3-1.1}]
anim := gif.GIF{LoopCount: nframes}
freq := rand.Float64() * 3.0
t := 0.0
    \end{lstlisting}
    A causa della loro compattezza e flessibilità, le short variable declaration sono usate per dichiarare e inizializzare la maggior parte delle variabili locali.
    Una dichiarazione \verb"var" tende ad essere riservata alle variabili locali che hanno bisogno di un tipo esplicito diverso da quello dell'espressione di inizializzazione, o per quando alla variabile viene assegnato un valore in seguito e il suo valore iniziale non è importante.
    \begin{lstlisting}[frame = single,label={lst:lstlisting1-3-1.2}]
i := 100		  // un int
var boiling float64 = 100 // un float64

var names []string
var err error
var p Point
    \end{lstlisting}
    Come per le dichiarazioni \verb"var", più variabili possono essere dichiarate e inizializzate con la stessa short variable declaration
    \begin{lstlisting}[frame = single,label={lst:lstlisting1-3-1.3}]
i, j := 0, 1
    \end{lstlisting}
    ma le istruzioni con più espressioni di inizializzazione dovrebbero essere usate solo quando aiutano la leggibilità.
    \hfill \vspace{12pt}

    Va sottolineato che \verb":=" è una dichiarazione, mentre \verb"=" è un'assegnazione.
    Un'istruzione multi-variabile non dovrebbe essere confusa con un'\textit{assegnazione tupla}, in cui ad ogni variabile sul lato sinistro viene assegnato un valore sul lato destro:
    \begin{lstlisting}[frame = single,label={lst:lstlisting1-3-1.4}]
i, j = j, i
    \end{lstlisting}
    in questo caso si scambiano i valori di \verb"i" e \verb"j".
    \hfill \vspace{12pt}

    Come le normali istruzioni \verb"var", le short variable declaration possono essere usate per le chiamate di funzione.
    Una short variable declaration si comporta come un'assegnazione solo per le variabili che sono già state dichiarate nello stesso blocco lessicale;
    le dichiarazioni in blocchi esterni sono ignorate.
\end{document}
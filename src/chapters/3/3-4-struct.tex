%
Una \textit{struct} è un tipo di dati aggregato che raggruppa insieme zero o più valori denominati di tipi arbitrari in una singola entità.
Ogni valore è detto \textit{campo}.

Queste due istruzioni dichiarano un tipo di struct chiamato \verb|Employee| e una variabile chiamata \verb|dilbert| che è un'istanza di un \verb|Employee|:
\begin{lstlisting}[frame=single, label={lst:lstlisting3-4.1}]
type Employee struct {
    ID 	      int
    Name      string
    Address   string
    DoB       time.Time
    Position  string
    Salary    int
    ManagerID int
}

var dilbert Employee
\end{lstlisting}
I singoli campi di \verb|dilbert| sono accessibili usando la notazione a punti (\verb|dilbert.Name| e \verb|dilbert.DoB|).
Poiché \verb|dilbert| è una variabile, anche i suoi campi sono variabili, quindi possiamo assegnare ad un campo:
\begin{lstlisting}[frame=single, label={lst:lstlisting3-4.2}]
dilbert.Salary -= 5000
\end{lstlisting}
o prendere il suo indirizzo e accedervi tramite un puntatore:
\begin{lstlisting}[frame=single, label={lst:lstlisting3-4.3}]
position := &dilbert.Position
*position = %*``*\)Senior %*''*\) + *position
\end{lstlisting}
La notazione a punti funziona anche con un puntatore a una struct:
\begin{lstlisting}[frame=single, label={lst:lstlisting3-4.4}]
var employeeOfTheMonth *Employee = &dilbert
employeeOfTheMonth.Position += %*``*\) (proactive team player)%*''*\)
\end{lstlisting}
Dove l'ultima istruzione è equivalente a
\begin{lstlisting}[frame=single, label={lst:lstlisting3-4.5}]
(*employeeOfTheMonth).Position += %*``*\) (proactive team player)%*''*\)
\end{lstlisting}
I campi sono di solito scritti uno per riga, con il nome del campo che precede il suo tipo, ma i campi consecutivi dello stesso tipo possono essere combinati, come con \verb|Name| e \verb|Address| qui:
\begin{lstlisting}[frame=single, label={lst:lstlisting3-4.6}]
type Employee struct {
    ID            int
    Name, Address string
    DoB           time.Time
    Position      string
    Salary        int
    ManagerID     int
}
\end{lstlisting}
Un tipo di struct denominato \verb|S| non può dichiarare un campo dello stesso tipo \verb|S|: un valore aggregato non può contenere se stesso.
Ma \verb|S| può dichiarare un campo del tipo di puntatore \verb|*S|, che ci permette di creare struct di dati ricorsive come liste concatenate e alberi.
Questo è illustrato nel codice qui sotto, che utilizza un albero binario per implementare un ordinamento di inserimento:
\begin{lstlisting}[frame=single, label={lst:lstlisting3-4.7}]
type tree struct {
    value       int
    left, right *tree
}

// Sorts ordina i valori in posizione.
func Sort(values []int) {
    var root *tree
    for _, v := range values {
        root = add(root, v)
    }
    appendValues(values[:0], root)
}

// appendValues aggiunge gli elementi di t ai valori
// in ordine e restituisce la slice risultante.
func appendValues(values []int, t *tree) []int {
    if t != nil {
        values = appendValues(values, t.left)
        values = append(values, t.value)
        values = appendValues(values, t.right)
    }
    return values
}

func add(t *tree, value int) *tree {
    if t == nil {
        // Equivalente a ritornare &tree{value:value}
        t = new(tree)
        t.value = value
        return t
    }
    if value < t.value {
        t.left = add(t.left, value)
    } else {
        t.right = add(t.right, value)
    }
    return t
}
\end{lstlisting}
Il valore zero per una struct è composto dai valori zero di ciascuno dei suoi campi.

Il tipo di struct senza campi è chiamato \textit{empty struct}, scritto \verb|struct{}|.

\subsection{Struct literal}
\label{subsec:struct_literal}%
Un valore di un tipo di struttura può essere scritto usando una \textit{struct literal} che specifica i valori per i suoi campi.
\begin{lstlisting}[frame=single, label={lst:lstlisting3-4-1.1}]
type Point struct { X, Y int }

p := Point{1, 2}
\end{lstlisting}
Ci sono due forme di struct literal.
La prima forma, mostrata sopra, richiede che sia specificato un valore per \textit{ogni} campo nell'ordine giusto.
Rende il codice fragile se l'insieme dei campi dovesse poi crescere o essere riordinato.
Di conseguenza, questa forma tende ad essere usata solo all'interno del pacchetto che definisce il tipo di struct, o con tipi di struct più piccoli per i quali esiste un'ovvia convenzione di ordinamento dei campi, come \verb|image.Point{x, y}| o \verb|color.RGBA{red, green, blue, alpha}|.

Più spesso, viene usata la seconda forma, in cui un valore di struct viene inizializzato elencando alcuni o tutti i nomi dei campi e i loro valori corrispondenti.
Se un campo è omesso in questo tipo di literal, viene impostato al valore zero per il suo tipo.

I valori di struct possono essere passati come argomenti alle funzioni e restituiti da esse.
Per esempio, questa funzione scala un ``Punto'' di un fattore specificato:
\begin{lstlisting}[frame=single, label={lst:lstlisting3-4-1.2}]
func Scale(p Point, factor int) Point {
    return Point{p.X * factor, p.Y * factor}
}

fmt.Println(Scale(Point{1, 2}, 5))
\end{lstlisting}
Output:
\begin{lstlisting}[language=bash, frame=L, label={lst:lstlisting3-4-1.3}]
{5 10}
\end{lstlisting}
Per efficienza, la struct più grande è solitamente passata a o restituita da funzioni indirettamente usando un puntatore
\begin{lstlisting}[frame=single, label={lst:lstlisting3-4-1.4}]
func Bonus(e *Employee, percent int) int {
    return e.Salary * percent / 100
}
\end{lstlisting}
e questo è necessario se la funzione deve modificare il suo argomento, poiché in un linguaggio call-by-value come Go, la funzione chiamata riceve solo una copia di un argomento, non un riferimento all'argomento originale.

\subsection{Confronto tra struct}
\label{subsec:confronto_tra_struct}%
Se tutti i campi di una struct sono comparabili, la struct stessa è comparabile, quindi due espressioni di quel tipo possono essere confrontate usando \verb|==| o \verb|!=|.
L'operazione \verb|==| confronta i campi corrispondenti delle due struct in ordine, quindi le due espressioni stampate qui sotto sono equivalenti:
\begin{lstlisting}[frame=single, label={lst:lstlisting3-4-2.1}]
type Point struct { X, Y int }

p := Point{1, 2}
q := Point{2, 1}
fmt.Println(p.X == q.X && p.Y == q.Y)
fmt.Println(p == q)
\end{lstlisting}
Output:
\begin{lstlisting}[language=bash, frame=L, label={lst:lstlisting3-4-2.2}]
false
false
\end{lstlisting}

\subsection{Struct embedding e campi anonimi}
\label{subsec:struct_embedding_e_campi_anonimi}%
Un insolito meccanismo \textit{struct embedding} ci permette di usare un tipo di struct detto \textit{campo anonimo} di un altro tipo di struct, fornendo una comoda scorciatoia sintattica in modo che una semplice espressione puntiforme come \verb|x.f| possa rappresentare una catena di campi come \verb|x.d.e.f|.
Dati le seguenti struct:
\begin{lstlisting}[frame=single, label={lst:lstlisting3-4-3.1}]
type Circle struct {
    X, Y, Radius int
}

type Wheel struct {
    X, Y, Radius, Spokes int
}
\end{lstlisting}
conviene raccogliere i campi uguali a individuare una nuova struct
\begin{lstlisting}[frame=single, label={lst:lstlisting3-4-3.2}]
type Point struct {
    X, Y int
}
type Circle struct {
    Center Point
    Radius int
}

type Wheel struct {
    Circle Circle
    Spokes int
}
\end{lstlisting}
L'applicazione diventerebbe più chiara, ma questo cambiamento renderebbe anche l'accesso ai campi di ``Wheel'' più prolisso:
\begin{lstlisting}[frame=single, label={lst:lstlisting3-4-3.3}]
var w Wheel
w.Circle.Center.X = 8
w.Circle.Center.Y = 8
w.Circle.Radius = 5
w.Spokes = 20
\end{lstlisting}
Go ci permette di dichiarare un campo con un tipo ma senza nome;
tali campi sono chiamati \textit{campi anonimi}.
Il tipo di campo deve essere un tipo con nome o un puntatore a un tipo con nome.
Sotto, \verb|Circle| e \verb|Wheel| hanno un campo anonimo ciascuno.
Diciamo che un \verb|Point| è \textit{incorporato} all'interno di \verb|Circle|, e un \verb|Circle| è incorporato all'interno di \verb|Wheel|.
\begin{lstlisting}[frame=single, label={lst:lstlisting3-4-3.4}]
type Circle struct {
    Point
    Radius int
}

type Wheel struct {
    Circle
    Spokes int
}
\end{lstlisting}
Grazie all'incorporazione, possiamo fare riferimento ai nomi alle foglie dell'albero implicito senza dare i nomi intermedi:
\begin{lstlisting}[frame=single,label={lst:lstlisting3-4-3.5}]
var w Wheel
w.X = 8	     // equivalente a w.Circle.Point.X = 8
w.Y = 8	     // equivalente a w.Circle.Point.Y = 8
w.Radius = 5 // equivalente a w.Circle.Radius = 5
w.Spokes = 20
\end{lstlisting}
Poiché i campi ``anonimi'' hanno nomi impliciti, non è possibile avere due campi anonimi dello stesso tipo perché in tal caso i loro nomi entrerebbero in conflitto.

Il tipo struct esterno guadagna non solo i campi del tipo embedded ma anche i suoi metodi.
Questo meccanismo è il modo principale in cui i comportamenti degli oggetti complessi sono composti da quelli più semplici.
La \textit{composizione} è centrale per la programmazione orientata agli oggetti in Go.
%
La tabella hash è una delle più ingegnose e versatili di tutte le strutture dati.
Si tratta di una raccolta non ordinata di coppie chiave/valore in cui tutte le chiavi sono distinte e il valore associato a una determinata chiave può essere recuperato, aggiornato o rimosso utilizzando in media un numero costante di confronti di chiave, non importa quanto grande sia la tabella hash.

In Go, una \textit{map} è un riferimento a una tabella hash, ed è scritto \verb|map[K]V|, dove \verb|K| e \verb|V| sono i tipi delle sue chiavi e dei suoi valori.
Tutte le chiavi in una determinata map sono dello stesso tipo e tutti i valori sono dello stesso tipo, ma le chiavi non devono necessariamente essere dello stesso tipo dei valori.
Il tipo di chiave \verb|K| deve essere comparabile usando \verb|==|, in modo che la map possa verificare se una determinata chiave è uguale a una già all'interno di essa.
Non ci sono restrizioni sul tipo di valore \verb|V|.

La funzione integrata \verb|make| può essere utilizzata per creare una map:
\begin{lstlisting}[label={lst:lstlisting3-3.1}]
ages := make(map[string]int) // map da string a int
\end{lstlisting}
Possiamo anche usare una \textit{map literal} per creare una nuova map popolata da alcune coppie chiave/valore iniziali:
\begin{lstlisting}[frame=single, label={lst:lstlisting3-3.2}]
ages := map[string]int {
    %*``*\)alice%*''*\):   31,
    %*``*\)charlie%*''*\): 34,
}
\end{lstlisting}
Ciò equivale a
\begin{lstlisting}[frame=single, label={lst:lstlisting3-3.3}]
ages := make(map[string]int)
ages[%*``*\)alice%*''*\)] = 31
ages[%*``*\)charlie%*''*\)] = 34
\end{lstlisting}
quindi un'espressione alternativa per una nuova map vuota è \verb|map[string]int{}|.

Gli elementi della map sono accessibili attraverso la consueta notazione:
\begin{lstlisting}[frame=single, label={lst:lstlisting3-3.4}]
ages[%*``*\)alice%*''*\)] = 32
fmt.Println(ages[%*``*\)alice%*''*\)])
\end{lstlisting}
Output:
\begin{lstlisting}[language=bash, frame=L, label={lst:lstlisting3-3.5}]
32
\end{lstlisting}
e rimosso con la funzione integrata \verb|delete|:
\begin{lstlisting}[language=go, frame=single, label={lst:lstlisting3-3.6}]
delete(ages, %*``*\)alice%*''*\)) // rimuovere elemento et%*\textit{à}*\)[%*``*\)alice%*''*\)]
\end{lstlisting}
Tutte le operazioni sono sicure anche se l'elemento non è nella map;
una ricerca di una chiave che non è presente nella map restituisce il valore zero per il suo tipo, così, per esempio, il seguente codice funziona anche quando ``bob'' non è ancora una chiave della map perché il valore di \verb|ages["bob"]| sarà \verb|0|.
\begin{lstlisting}[frame=single, label={lst:lstlisting3-3.7}]
ages[%*``*\)bob%*''*\)]++
\end{lstlisting}
Ma un elemento della map non è una variabile, per cui non si può ottenere il suo indirizzo
\begin{lstlisting}[frame=single, label={lst:lstlisting3-3.8}]
_ = &ages[%*``*\)bob%*''*\)] // compile error: impossibile prendere
                // l'indirizzo di un elemento di map
\end{lstlisting}
Il valore zero per una map è \verb|nil|, cioè un riferimento a nessuna tabella hash.
\begin{lstlisting}[frame=single, label={lst:lstlisting3-3.9}]
var ages map[string]int
fmt.Println(ages == nil)
fmt.Println(len(ages) == 0)
\end{lstlisting}
Output:
\begin{lstlisting}[language=bash, frame=L, label={lst:lstlisting3-3.10}]
true
true
\end{lstlisting}
La maggior parte delle operazioni sulle map, inclusi i cicli di ricerca, \verb|delete|, \verb|len| e \verb|range|, sono sicure da eseguire su una map di riferimento nil, poiché si comporta come una mappa vuota.
Ma la memorizzazione di una map nil provoca panic:
\begin{lstlisting}[frame=single, label={lst:lstlisting3-3.11}]
ages[%*``*\)carol%*''*\)] = 21 // panic: assegnazione ad una map nil
\end{lstlisting}
È necessario allocare la map prima di poter memorizzare in esso coppie chiave/valore.
In particolare, se si vuole sapere se una chiave è presente nella map, allora si può effettuare una chiamata:
\begin{lstlisting}[frame=single, label={lst:lstlisting3-3.12}]
age, ok = ages[%*``*\)bob%*''*\)]
\end{lstlisting}
In cui il primo valore contiene il valore associato alla chiave \verb|bob| nella map (se presente), mentre il secondo contiene un valore booleano che indica se la chiave è presente o meno nella map.

Come nelle slice, le map non possono essere confrontate fra loro;
l'unico confronto legale è con \verb|nil|.
\documentclass[../../../thesis.tex]{subfiles}
\begin{document}
    \subsubsection{Struct embedding e campi anonimi}
    Un insolito meccanismo \textit{struct embedding} ci permette di usare un tipo di struct detto \textit{campo anonimo} di un altro tipo di struct, fornendo una comoda scorciatoia sintattica in modo che una semplice espressione puntiforme come \verb"x.f" possa rappresentare una catena di campi come \verb"x.d.e.f".
    \hfill \vspace{12pt}

    Dati le seguenti struct
    \begin{lstlisting}[frame = single,label={lst:lstlisting3-4-3.1}]
type Circle struct {
    X, Y, Radius int
}

type Wheel struct {
    X, Y, Radius, Spokes int
}
    \end{lstlisting}
    conviene raccogliere i campi uguali a individuare una nuova struct
    \begin{lstlisting}[frame = single,label={lst:lstlisting3-4-3.2}]
type Point struct {
    X, Y int
}
type Circle struct {
    Center Point
    Radius int
}

type Wheel struct {
    Circle Circle
    Spokes int
}
    \end{lstlisting}
    L'applicazione diventerebbe più chiara, ma questo cambiamento renderebbe anche l'accesso ai campi di ``Wheel" più prolisso:
    \begin{lstlisting}[frame = single,label={lst:lstlisting3-4-3.3}]
var w Wheel
w.Circle.Center.X = 8
w.Circle.Center.Y = 8
w.Circle.Radius = 5
w.Spokes = 20
    \end{lstlisting}
    Go ci permette di dichiarare un campo con un tipo ma senza nome;
    tali campi sono chiamati \textit{campi anonimi}.
    Il tipo di campo deve essere un tipo con nome o un puntatore a un tipo con nome.
    Sotto, \verb"Circle" e \verb"Wheel" hanno un campo anonimo ciascuno.
    Diciamo che un \verb"Point" è \textit{incorporato} all'interno di \verb"Circle", e un \verb"Circle" è incorporato all'interno di \verb"Wheel".
    \begin{lstlisting}[frame = single,label={lst:lstlisting3-4-3.4}]
type Circle struct {
    Point
    Radius int
}

type Wheel struct {
    Circle
    Spokes int
}
    \end{lstlisting}
    Grazie all'incorporazione, possiamo fare riferimento ai nomi alle foglie dell'albero implicito senza dare i nomi intermedi:
    \begin{lstlisting}[frame = single,label={lst:lstlisting3-4-3.5}]
var w Wheel
w.X = 8	     // equivalente a w.Circle.Point.X = 8
w.Y = 8	     // equivalente a w.Circle.Point.Y = 8
w.Radius = 5 // equivalente a w.Circle.Radius = 5
w.Spokes = 20
    \end{lstlisting}
    Poiché i campi ``anonimi" hanno nomi impliciti, non è possibile avere due campi anonimi dello stesso tipo perché in tal caso i loro nomi entrerebbero in conflitto.
    \hfill \vspace{12pt}

    Il tipo struct esterno guadagna non solo i campi del tipo embedded ma anche i suoi metodi.
    Questo meccanismo è il modo principale in cui i comportamenti degli oggetti complessi sono composti da quelli più semplici.
    La \textit{composizione} è centrale per la programmazione orientata agli oggetti in Go.
\end{document}
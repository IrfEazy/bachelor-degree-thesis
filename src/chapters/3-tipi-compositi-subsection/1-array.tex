\documentclass[../../thesis.tex]{subfiles}
\begin{document}
    \subsection{Gli array}\label{subsec:gli-array}
    Un array è una sequenza a lunghezza fissa di zero o più elementi di un particolare tipo.
    A causa della loro lunghezza fissa, gli array sono raramente utilizzati in modo diretto in Go. Le slice, che possono crescere e ridursi, sono molto più versatili, ma per capire le slice bisogna prima capire le matrici.
    \hfill \vspace{12pt}

    I singoli elementi dell'array sono accessibili con la notazione indice convenzionale, dove gli indici vanno da zero a uno in meno rispetto alla lunghezza dell'array.
    La funzione integrata \verb|len| restituisce il numero di elementi nell'array.
    \begin{lstlisting}[frame = single,label={lst:lstlisting3-1.1}]
var a [3]int		 // array di 3 interi
fmt.Println(a[0])	 // stampa il primo elemento
fmt.Println(a[len(a)-1]) // stampa l'ultimo elemento

// Stampa gli indici e gli elementi.
for i, v := range a {
    fmt.Printf("%d %d\n", i, v)
}

// Stampa solo gli elementi.
for _, v := range a {
    fmt.Printf("%d\n", v)
}
    \end{lstlisting}
    Per impostazione predefinita, gli elementi di una nuova variabile di matrice sono inizialmente impostati al valore zero per il particolare tipo di elemento, che è \verb|0| per i numeri.
    Possiamo usare un \textit{array letterale} per inizializzare un array con una lista di valori:
    \begin{lstlisting}[frame = single,label={lst:lstlisting3-1.2}]
var q [3]int = [3]int{1, 2, 3}
var r [3]int = [3]int{1, 2}
fmt.Println(r[2])
    \end{lstlisting}
    Output:
    \begin{lstlisting}[language = bash, frame = L,label={lst:lstlisting3-1.3}]
0
    \end{lstlisting}
    In un array letterale, se un ellissi ``\verb|...|" appare al posto della lunghezza, la lunghezza dell'array è determinata dal numero di inizializzatori.
    La definizione di \verb|q| può essere semplificata a:
    \begin{lstlisting}[frame = single,label={lst:lstlisting3-1.4}]
q := [...]int{1, 2, 3}
fmt.Println("%T\n", q)
    \end{lstlisting}
    Output:
    \begin{lstlisting}[language = bash, frame = L,label={lst:lstlisting3-1.5}]
[3]int
    \end{lstlisting}
    La dimensione di un array fa parte del suo tipo, quindi \verb|[3]int| e \verb|[4]int| sono tipi diversi.
    La dimensione deve essere un'espressione costante, cioè un'espressione il cui valore può essere calcolato mentre il programma viene compilato.
    \begin{lstlisting}[frame = single,label={lst:lstlisting3-1.6}]
q := [3]int{1, 2, 3}
q = [4]int{1, 2, 3, 4} // compile error: impossibile assegnare [4]int a
                       // [3]int
    \end{lstlisting}
    La sintassi letterale è simile per array, slice, map e struct.
    La forma specificata sopra è una lista ordinata di valori, ma è anche possibile specificare una lista di coppie di indici e valori, come:
    \begin{lstlisting}[frame = single,label={lst:lstlisting3-1.7}]
type Currency int

const (
    USD Currency = iota
    EUR
    GBP
)

symbol := [...]string{USD: "%*{\dollar}*)", EUR: "%*{€}*)", GBP: "%*{£}*)"}

fmt.Println(GBP, symbol[GBP])
    \end{lstlisting}
    Output:
    \begin{lstlisting}[language = bash, frame = L,label={lst:lstlisting3-1.8}]
2 %*{£}*)
    \end{lstlisting}
    Se il tipo di elemento di un array è comparabile, anche il tipo di array è comparabile, quindi possiamo confrontare direttamente due array di quel tipo usando l'operatore \verb"==", che riporta se tutti gli elementi corrispondenti sono uguali.
    L'operatore \verb"!=" è la sua negazione.
    \hfill \vspace{12pt}

    Come esempio, la funzione \verb"Sum256" nel pacchetto \verb"crypto/sha256" produce l'hash crittografico SHA256 o il \textit{digest} di un messaggio memorizzato in una slice di byte arbitraria.
    Il digest ha 256 bit, quindi il suo tipo è \verb"[32]byte".
    Se due digest sono uguali, è estremamente probabile che i due messaggi siano uguali;
    se i digest differiscono, i due messaggi sono diversi.
    Questo programma stampa e confronta i digest SHA256 di ``\verb"x"" e ``\verb"X"":
    \begin{lstlisting}[frame = single,label={lst:lstlisting3-1.9}]
import "crypto/sha256"

func main() {
    c1 := sha256.Sum256([]byte("x"))
    c2 := sha256.Sum256([]byte("X"))
    fmt.Printf("%x\n%x\n%t\n%T\n", c1, c2, c1 == c2, c1)
}
    \end{lstlisting}
    Output:
    \begin{lstlisting}[language = bash, frame = L,label={lst:lstlisting3-1.10}]
2d711642b726b04401627ca9fbac32f5c8530fb1903cc4db02258717921a4881
4b68ab3847feda7d6c62c1fbcbeebfa35eab7351ed5e78f4ddadea5df64b8015
false
[32]uint8
    \end{lstlisting}
    I due ingressi differiscono di un solo bit, ma molti bit dei loro digest sono diversi fra loro.
    \hfill \vspace{12pt}

    Quando viene chiamata una funzione, una copia di ogni valore di argomento viene assegnata alla variabile parametro corrispondente, quindi la funzione riceve una copia, non l'originale.
    Il passaggio di array di grandi dimensioni in questo modo può essere inefficiente, e qualsiasi modifica apportata dalla funzione agli elementi di array influisce solo sulla copia, non sull'originale.
    A questo proposito, Go tratta gli array come qualsiasi altro tipo, ma questo comportamento è diverso dai linguaggi che implicitamente passano gli array \textit{per riferimento}.
    \hfill \vspace{12pt}

    Naturalmente, possiamo passare esplicitamente un puntatore ad un array in modo che tutte le modifiche che la funzione apporta agli elementi array siano visibili al chiamante.
    Questa funzione azzera il contenuto di un array \verb"[32]byte":
    \begin{lstlisting}[frame = single,label={lst:lstlisting3-1.11}]
func zero(ptr *[32]byte) {
    *ptr = [32]byte{}
}
    \end{lstlisting}
    L'array letterale \verb"[32]byte{}" produce un array di 32 byte.
    Ogni elemento dell'array ha il valore zero per del byte, ovvero la traduzione di \verb"0" in byte.
    \hfill \vspace{12pt}

    L'utilizzo di un puntatore a un array è efficiente e consente alla funzione chiamata di mutare la variabile del chiamante, ma gli array sono ancora intrinsecamente inflessibili a causa delle loro dimensioni fisse.
    La funzione \verb"zero" non accetta un puntatore a una variabile di \verb"[16]byte", per esempio, né c'è alcun modo di aggiungere o rimuovere elementi di matrice.
    Per queste ragioni, oltre a casi speciali come l'hash di dimensione fissa di SHA256, gli array sono raramente usati come parametri di funzione o come risultati;
    invece, usiamo le slice.
\end{document}
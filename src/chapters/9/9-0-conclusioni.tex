Tutti i linguaggi di programmazione riflettono la filosofia di programmazione dei propri creatori.
Il progetto Go è nato con l'obiettivo di limitare la proliferazione della complessità all'interno dei programmi.
Per riuscire a raggiungere l'obiettivo i creatori del linguaggio hanno scelto di rimuovere alcuni costrutti ridondanti (in Go esiste solo un ciclo for esempio).
Abbiamo visto che in Go non si usa il concetto di oggetto così come in altri linguaggi (Java, per esempio), e da questa differenza ne derivano approcci completamente diversi alla programmazione orientata agli oggetti (OOP).
In Go gli oggetti sono struct che possono essere esportati o meno, e ad essi vengono associati metodi da richiamare in altre parti di codice.
Un oggetto $x$ può condividere i propri comportamenti, ovvero metodi, con un oggetto $y$ nel caso venga inserito fra i campi di $x$.
Si è anche visto come in Go ogni tipo (base e compositi) sia considerato estensione dell'interfaccia vuota \verb|interface{}|, priva di metodi e campi.
Queste caratteristiche permettono a Go di trovare utilizzo anche come linguaggio di programmazione di basso livello.
Go offre pure una propria visione di thread creando la goroutine.
La goroutine si differenzia dal thread non per il servizio che offre, ma perché ha una dimensione standard molto più piccola e molto più flessibile di quella di un thread del sistema operativo.
Se un thread richiede almeno 2MB, una goroutine ne richiede almeno 2KB: il rapporto è circa 1:1000.

Alcune delle caratteristiche del linguaggio Go che portano i programmatori ad esserne attrati, specialmente per il cloud, è la sua velocità e la sua capacità di ottenere un file binario statico privo di dipendenze, per cui si è in grado di ottenere un programma scritto in un dato sistema operativo e di trasferirlo su qualunque altra macchina, senza vincoli, perché questo funzionerà senza richiedere installazioni aggiuntive o creare conflitti nelle dipendenze.
Il grande vantaggio offerto da Go nel cloud è la possibilità di offrire un file binario statico molto piccolo in termini di spazio.
Preso per esempio un Docker container (entità per distribuire il carico di lavoro nel cloud), Go riesce a restituire allo sviluppatore un Docker file delle dimensioni di qualche decina di megabytes, mentre con altri linguaggi di programmazione - Java, Python, ecc.\ - le dimensioni diventano facilmente 10 volte tanto, a raggiungere file di qualche gigabyte.

Dal 2009 ad oggi, Go è riuscito ad acquisire molta popolarità, infatti è oggi utilizzato per l'implementazione dei casi d'uso di American Express, Dropbox, Meta, Google, Microsoft, Twitter, Uber, ecc.
In un sondaggio per i programmatori del 2017 Go è comparso nella top 20 fra tutti i linguaggi di programmazione, ponendosi nella top 10 in termini di crescita e apprezzamento per i programmatori.

Go non è riuscito a trovare spazio nello sviluppo di applicazioni desktop, non perché non sia possibile, ma perché è complesso.
Go ha trovato comunque sempre più uso e spazio come linguaggio per i casi d'uso sul lato server, per costruire servizi di rete, applicazioni web e tutto ciò che può essere eseguito in rete.

Secondo alcuni sondaggi degli ultimi anni, Go sta diventando sempre più protagonista nell'implementazione dei servizi offerti dai server.
La sua diffusione sembra essere in crescita ancora oggi.


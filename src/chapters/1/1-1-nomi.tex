%
I nomi delle funzioni, le variabili, le costanti, i tipi, le etichette delle dichiarazioni e i pacchetti in Go seguono una semplice regola: un nome inizia con una lettera (cioè, qualunque cosa Unicode consideri una lettera) o un underscore e poi un qualsiasi numero di lettere, cifre e sottolineature.

Go ha 25 \textit{parole chiave} che non possono essere usate come nomi.
\begin{table}[H]
    \centering
    \begin{tabular}{ l l l l l }
        \verb|break|    & \verb|default|     & \verb|func|   & \verb|interface| & \verb|select| \\
        \verb|case|     & \verb|defer|       & \verb|go|     & \verb|map|       & \verb|struct| \\
        \verb|chan|     & \verb|else|        & \verb|goto|   & \verb|package|   & \verb|switch| \\
        \verb|const|    & \verb|fallthrough| & \verb|if|     & \verb|range|     & \verb|type|   \\
        \verb|continue| & \verb|for|         & \verb|import| & \verb|return|    & \verb|var|
    \end{tabular}
    \label{tab:table11}
\end{table}

Inoltre ci sono nomi \textit{predichiarati} per le costanti, i tipi e le funzioni incorporate.
\begin{table}[H]
    \centering
    \begin{tabular}{ l l }
        Costanti: & \verb|true| \verb|false| \verb|iota| \verb|nil|                                                   \\
        Tipi:     & \verb|int| \verb|int8| \verb|int16| \verb|int32| \verb|int64|                                     \\
        & \verb|uint| \verb|uint8| \verb|uint16| \verb|uint32| \verb|uint64| \verb|uintptr|                 \\
        & \verb|float32| \verb|float64| \verb|complex128| \verb|complex64|                                  \\
        & \verb|bool| \verb|byte| \verb|rune| \verb|string| \verb|error|                                    \\
        Funzioni: & \verb|make| \verb|len| \verb|cap| \verb|new| \verb|append| \verb|copy| \verb|close| \verb|delete| \\
        & \verb|complex| \verb|real| \verb|imag|                                                            \\
        & \verb|panic| \verb|recover|
    \end{tabular}
    \label{tab:table12}
\end{table}
Questi nomi non sono riservati, quindi possono essere utilizzati nelle dichiarazioni.

Quando un'entità è dichiarata all'interno di una funzione, è \textit{locale} a tale funzione.
Se la variabile è dichiarata non funzionante, è visibile a tutti i file nel pacchetto.
La prima lettera di un nome determina la sua visibilità attraverso i pacchetti.
Se il nome inizia con una lettera maiuscola, si dice che sia \textit{esportato} perché è visibile a tutti i pacchetti al di fuori del proprio.
I nomi dei pacchetti sono sempre in minuscolo.

Non ci sono limiti alla lunghezza dei nomi, ma le convenzioni e gli stili nei programmi Go preferiscono i nomi brevi, e se ci sono più parole, la notazione camel-case è preferita rispetto al snake-case.
Quindi si preferisce chiamare le variabili secondo lo stile \verb|archivioRetiPetri|, piuttosto dello stile \verb|archivio_reti_petri|.
Se si desidera inserire acronimi all'interno di un nome, la convenzione richiede l'uso di caratteri minuscoli o solo caratteri maiuscoli;
si preferisce \verb|fileHTML| o una variabile \verb|htmlFile| a una variabile \verb|fileHtml|.


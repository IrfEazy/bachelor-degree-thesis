Un'istruzione nomina un'entità del programma e specifica alcune o tutte le sue proprietà.
Ci sono quattro tipi di dichiarazioni: \verb|var|, \verb|const|, \verb|type| e \verb|func|.

Un programma Go è archiviato in uno o più file i quali nomi finiscono in \verb|.go|.
Ognuno dei file inizia con una dichiarazione del package che dice di quale package fa parte il file.
La dichiarazione del \verb|package| è seguita da dichiarazioni di \verb|import|, e quindi una sequenza di dichiarazioni di \textit{livello package} di tipi, variabili, costanti, e funzioni, in qualunque ordine.

Per esempio, questo programma dichiara una costante, una funzione, e una coppia di variabili:
\begin{lstlisting}[frame=single, label={lst:lstlisting1-2.1}, literate={°}{\textdegree}1]
package main

import %*``*\)fmt%*''*\)

const boilingF = 212.0

func main() {
    var f = boilingF
    var c = (f - 32) * 5 / 9
    fmt.Printf(%*``*\)boiling point = %g°F or %g°C\n%*''*\), f, c)
}
\end{lstlisting}
\begin{lstlisting}[language=bash, frame=L, label={lst:lstlisting1-2.2}, literate={°}{\textdegree}1]
$ ./boiling
boiling point = 212°F o 100°C
\end{lstlisting}
La costante \verb|boilingF| è un'istruzione a livello di pacchetto, quindi tutti i file sorgente nel pacchetto vedranno questa variabile.
Al contrario, la variabile locale \verb|f| è visibile solo alla funzione \verb|main| e nessun altro.


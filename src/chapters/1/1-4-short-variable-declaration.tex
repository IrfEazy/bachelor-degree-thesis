%
All'interno di una funzione, una forma alternativa chiamata \textit{short variable declaration} può essere usata per dichiarare e inizializzare le variabili locali.
Questa istruzione assume la forma di \verb|nome := espressione| e il tipo di \verb|nome| è determinato dal tipo di \verb|espressione|.
Esempi di brevi istruzioni di variabili sono:
\begin{lstlisting}[frame=single, label={lst:lstlisting1-3-1.1}]
anim := gif.GIF{LoopCount: nframes}
freq := rand.Float64() * 3.0
t := 0.0
\end{lstlisting}
A causa della loro compattezza e flessibilità, le short variable declaration sono usate per dichiarare e inizializzare la maggior parte delle variabili locali.
Una dichiarazione \verb|var| tende ad essere riservata alle variabili locali che hanno bisogno di un tipo esplicito diverso da quello dell'espressione di inizializzazione, o per quando alla variabile viene assegnato un valore in seguito e il suo valore iniziale non è importante.
\begin{lstlisting}[frame=single, label={lst:lstlisting1-3-1.2}]
i := 100                  // un int
var boiling float64 = 100 // un float64

var names []string
var err error
var p Point
\end{lstlisting}
Come per le dichiarazioni \verb|var|, più variabili possono essere dichiarate e inizializzate con la stessa short variable declaration
\begin{lstlisting}[frame=single, label={lst:lstlisting1-3-1.3}]
i, j := 0, 1
\end{lstlisting}
ma le istruzioni con più espressioni di inizializzazione dovrebbero essere usate solo quando aiutano la leggibilità.

Va sottolineato che \verb|:=| è una dichiarazione, mentre \verb|=| è un'assegnazione.
Un'istruzione multi-variabile non dovrebbe essere confusa con un'\textit{assegnazione tupla}, in cui ad ogni variabile sul lato sinistro viene assegnato un valore sul lato destro:
\begin{lstlisting}[frame=single, label={lst:lstlisting1-3-1.4}]
i, j = j, i
\end{lstlisting}
in questo caso si scambiano i valori di \verb|i| e \verb|j|.

Come le normali istruzioni \verb|var|, le short variable declaration possono essere usate per le chiamate di funzione.
Una short variable declaration si comporta come un'assegnazione solo per le variabili che sono già state dichiarate nello stesso blocco lessicale;
le dichiarazioni in blocchi esterni sono ignorate.

\subsection{Puntatori}
\label{subsec:puntatori}%
Una variabile è una parte di un archivio contenente un valore.
Le variabili create dalle dichiarazioni sono identificate da un nome, come \verb|x|, ma la maggior parte delle variabili sono identificate solo da espressioni come \verb|x[i]| o \verb|x.f|.
Tutte queste espressioni leggono il valore di una variabile, tranne quando appaiono sul lato sinistro di un'assegnazione, nel qual caso viene assegnato un nuovo valore alla variabile.

Un valore di un \textit{puntatore} è un \textit{indirizzo} di una variabile.
Un puntatore è quindi la posizione in cui un valore è memorizzato.
Non tutti i valori hanno un indirizzo, ma ogni variabile ne ha uno.
Con un puntatore, è possibile leggere o aggiornare \textit{indirettamente} il valore di una variabile, senza l'uso o la conoscenza del nome di una variabile, sempre ammesso abbia un nome.

Se una variabile è dichiarata \verb|var x int|, l'espressione \verb|&x| (``\verb|x| address'') restituisce un puntatore ad una variabile intera, che è un valore di tipo \verb|*int|, si legge ``pointer to int''.
Se questo valore è definito come \verb|p|, si dirà che ``\verb|p| punta a \verb|x|'', o equivalentemente ``\verb|p| contiene l'indirizzo di \verb|x|''.
La variabile a cui \verb|p| punta è indicata da \verb|*p|.
L'espressione \verb|*p| restituisce il valore di quella variabile, un \verb|int|, ma poiché \verb|*p| denota una variabile, allora può anche apparire a sinistra di un'assegnazione, nel qual caso aggiorna la variabile.
\begin{lstlisting}[frame=single, label={lst:lstlisting1-3-2.1}]
x := 1
p := &x // p, di tipo *int, punta a x
fmt.Println(*p)
*p = 2 // equivalente a x = 2
fmt.Println(x)
\end{lstlisting}
Output:
\begin{lstlisting}[language=bash, frame=L, label={lst:lstlisting1-3-2.2}]
1
2
\end{lstlisting}
Il valore zero di un puntatore per ogni tipo è \verb|nil|.
Se \verb|p| punta a una variabile, allora vale sempre \verb|p != nil|.
I puntatori sono comparabili;
due puntatori sono uguali se e solo se puntano alla stessa variabile o se entrambi sono \verb|nil|.
\begin{lstlisting}[frame=single, label={lst:lstlisting1-3-2.3}]
var x, y int
fmt.Println(&x == &x, &x == &y, &x == nil)
\end{lstlisting}
Output:
\begin{lstlisting}[language=bash, frame=L, label={lst:lstlisting1-3-2.4}]
true false false
\end{lstlisting}
Con i puntatori, possiamo ottenere le modifiche alle variabili locali da altre funzioni.
Per esempio,
\begin{lstlisting}[frame=single, label={lst:lstlisting1-3-2.5}]
func incr(p *int) int {
    *p++ // incrementa il puntato da p, non cambia p
    return *p
}

func main() {
    v := 1
    incr(&v) // side effect: v vale ora 2
    fmt.Println(incr(&v))
    fmt.Println(v)
}
\end{lstlisting}
Output:
\begin{lstlisting}[language=bash, frame=L, label={lst:lstlisting1-3-2.6}]
3
3
\end{lstlisting}

\subsection{La funzione new}
\label{subsec:la_funzione_new}%
Un altro modo per creare una variabile è usare una nuova funzione incorporate.
L'espressione \verb|new(T)| crea una \textit{variabile senza nome} di tipo \verb|T|, inizializzata al valore zero di \verb|T|, e restituisce il suo indirizzo, che è un valore di tipo \verb|*T|.
\begin{lstlisting}[frame=single, label={lst:lstlisting1-3-3.1}]
p := new(int) // p, che %*\textit{è}*\) di tipo *int, indica una variabile int
              // senza nome
fmt.Println(*p)
*p = 2 // imposta l'int senza nome a 2
fmt.Println(*p)
\end{lstlisting}
Output:
\begin{lstlisting}[language=bash, frame=L, label={lst:lstlisting1-3-3.2}]
0
2
\end{lstlisting}
Una variabile creata con la funzione \verb|new| non è diversa da una normale variabile locale se non per l'assenza del nome.
Questa funzione è usata perché è sintatticamente conveniente per generare dinamicamente un numero illimitato di variabili, come costruire strutture dati complesse e flessibili (per esempio, alberi e grafi).
Ogni chiamata a \verb|new| restituisce una variabile distinta con un indirizzo univoco:
\begin{lstlisting}[frame=single, label={lst:lstlisting1-3-3.3}]
p := new(int)
q := new(int)
fmt.Println(p == q)
\end{lstlisting}
Output:
\begin{lstlisting}[language=bash, frame=L, label={lst:lstlisting1-3-3.4}]
false
\end{lstlisting}
Tuttavia, c'è un'eccezione a questa regola: due variabili i cui tipi non portano informazioni e la loro dimensione è zero, come \verb|struct{}| o \verb|[0]int|, a seconda dell'implementazione possono ottenere lo stesso indirizzo.
Tuttavia, la funzione \verb|new| è usata raramente perché è più conveniente sfruttare i campi delle struct.

\subsection{La durata delle variabili}
\label{subsec:la_durata_delle_variabili}%
La \textit{durata} della variabile è l'intervallo di tempo durante il quale la variabile esiste durante l'esecuzione del programma.
La durata di una variabile a livello di pacchetto è uguale all'intera esecuzione del programma.
Al contrario, le variabili locali hanno una durata dinamica: una nuova istanza viene creata ogni volta che l'istruzione di dichiarazione viene eseguita, e la variabile vive fino a quando non diventa \textit{irraggiungibile}, momento in cui il suo archivio può essere riciclato.
Anche i parametri e i risultati delle funzioni sono variabili locali;
vengono creati ogni volta che viene chiamata una funzione invece di un parametro.

Ad esempio,
\begin{lstlisting}[frame=single, label={lst:lstlisting1-3-4.1}]
for t := 0.0; t < cycles*2*math.Pi; t += res {
    x := math.Sin(t)
    y := math.Sin(t*freq + phase)
    img.SetColorIndex(size+int(x*size+0.5),
        size+int(y*size+0.5), blackIndex)
}
\end{lstlisting}
la variabile \verb|t| viene creata ogni volta che inizia il ciclo \verb|for|, e le nuove variabili \verb|x| e \verb|y| vengono create ad ogni iterazione del ciclo.

Questo discorso diventa importante per il programmatore in Go, dove spesso fa uso di puntatori a oggetti di breve durata all'interno di oggetti di lunga durata, come variabili globali, perché così facendo impedirà al garbage collector di recuperare gli oggetti di breve durata.


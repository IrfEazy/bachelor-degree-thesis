%
Il valore detenuto da una variabile viene aggiornato da una dichiarazione di assegnazione, che nella sua forma più semplice ha una variabile a sinistra del segno \verb|=| e un'espressione a destra.
\begin{lstlisting}[frame=single, label={lst:lstlisting1-4.1}]
x = 1                       // variabile nominata
*p = true                   // variabile indiretta
person.name = %*``*\)bob%*''*\)         // campo struct
count[x] = count[x] * scale // elemento array o slice o map
\end{lstlisting}

\subsection{Assegnazione di tuple}
\label{subsec:assegnazione_di_tuple}%
Un'altra forma di assegnazione, nota come \textit{assegnazione di tuple}, permette di assegnare più variabili contemporaneamente.
Tutte le espressioni di destra vengono valutate prima che una delle variabili venga aggiornata, rendendo questo modulo più utile quando alcune delle variabili appaiono su entrambi i lati dell'assegnazione, come accade, ad esempio, quando si scambiano i valori di due variabili:
\begin{lstlisting}[frame=single, label={lst:lstlisting1-4-1.1}]
x, y = y, x
a[i], a[j] = a[j], a[i]
\end{lstlisting}
o quando si calcola il massimo comune divisore (MCD) di due interi:
\begin{lstlisting}[frame=single, label={lst:lstlisting1-4-1.2}]
func gcd(x, y int) int {
    for y != 0 {
        x, y = y, x%y
    }
    return x
}
\end{lstlisting}
o quando si calcola iterativamente l'\textit{n}-esimo numero di Fibonacci:
\begin{lstlisting}[frame=single, label={lst:lstlisting1-4-1.3}]
func fib(n int) int {
    x, y := 0, 1
    for i := 0; i < n; i++ {
        x, y = y, x+y
    }
    return x
}
\end{lstlisting}

\subsection{Assegnabilità}
\label{subsec:assegnabilita}%
Le istruzioni di assegnazione sono una forma esplicita di assegnazione, ma ci sono molti posti in un programma in cui un'assegnazione avviene \textit{implicitamente}: una chiamata di funzione assegna implicitamente i valori di argomento alle variabili di parametro corrispondenti;
un'istruzione \verb|return| assegna implicitamente gli operandi di \verb|return| alle corrispondenti variabili di risultato.

Un'assegnazione, esplicita o implicita, è sempre legale se il lato sinistro (la variabile) e il lato destro (il valore) hanno lo stesso tipo.
Più in generale, l'assegnazione è legale se solo se il valore è \textit{assegnabile} al tipo di variabile.

Per sapere se due valori possano essere confrontati con \verb|==| e \verb|!=| bisogna vedere la loro assegnabilità: in ogni confronto, il primo operando deve essere assegnabile al tipo del secondo operando, o viceversa.


\documentclass[../../thesis.tex]{subfiles}
\begin{document}
    \subsection{Mutex di lettura/scrittura: sync.RWMutex}\label{subsec:mutex-di-lettura/scrittura:-sync.rwmutex}
    Per permettere alle operazioni di sola lettura di essere eseguite in parallelo, e solo alle operazioni di scrittura di avere completo ed esclusivo accesso alle variabili, si può utilizzare il mutex \verb"sync.RWMutex", per poter utilizzare il lock detto \textit{multiple readers, single writer} lock:
    \begin{lstlisting}[frame = single,label={lst:lstlisting9-3.1}]
var mu sync.RWMutex
var balance int

func Balance() int {
    mu.RLock() // reader lock
    defer mu.RUnclock()
    return balance
}
    \end{lstlisting}
    La funzione \verb"Balance" ora chiama i metodi \verb"RLock" e \verb"RUnlock" per acquisire e rilasciare un \textit{reader} o \textit{shared} lock.
    La funzione \textit{Deposit} chiamerà sempre i metodi \verb"mu.Lock" e \verb"mu.Unlock" per acquisire e rilasciare un \textit{writer} o \textit{exclusive} lock.
\end{document}
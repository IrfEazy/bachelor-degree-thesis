\documentclass[../../thesis.tex]{subfiles}
\begin{document}
    \subsection{Race condition}\label{subsec:race-condition}
    In un programma sequenziale, ovvero un programma con un'unica goroutine, gli step di un programma vengono eseguiti in un ordine familiare determinato dalla logica del programma.
    Per esempio, in una sequenza di istruzioni, il primo viene eseguito prima del secondo, e così via.
    In un programma con due o più goroutine, gli step che una goroutine deve eseguire vengono sempre eseguiti in un ordine familiare, ma in generale non è possibile sapere se un evento $x$ in una goroutine accade prima di un evento $y$ in un'altra goroutine, o se accade dopo, o se simultaneamente.
    Quando non si conosce l'ordine di esecuzione di due eventi, essi sono detti \textit{concorrenti}.
    \hfill \vspace{12pt}

    Considerando una funzione che lavora correttamente in un programma sequenziale, tale funzione è \textit{concurrency-safe} se continua a lavorare correttamente anche se chiamata in modo concorrente.
    Si può generalizzare questa definizione ad un insieme di funzioni che si richiamano, come i metodi e le operazioni di un tipo particolare.
    Un tipo è concurrency-safe se tutti i suoi metodi e operazioni accessibili sono concurrency-safe.
    \hfill \vspace{12pt}

    Si può produrre un programma concurrency-safe senza rendere ogni tipo concreto concurrency-safe.
    Infatti, i tipi concurrency-safe sono l'eccezione piuttosto che la regola, per cui bisogna accedere ad una variabile concorrente solo se la documentazione di quel tipo dice che è sicuro.
    Gli accessi concorrenti a molte variabili sono evitate sia \textit{confinandole} in goroutine singole che mantenendo un alto livello di invarianza di \textit{mutua esclusione}.
    \hfill \vspace{12pt}

    In contrasto, le funzioni esportate a livello di package ci si aspetta \textit{siano} in generale concurrency-safe.
    Fintanto che le variabili a livello di package non sono confinate in una singola goroutine, le funzioni che le modificano devono assicurare la mutua esclusione.
    \hfill \vspace{12pt}

    Ci sono molte ragioni per cui una funzione non debba lavorare quando chiamata in modo concorrente, inclusi i deadlock, livelock e la starvation delle risorse.
    In questo capitolo verrà solo affrontata la \textit{race condition}.
    \hfill \vspace{12pt}

    Una race condition è una situazione in cui il programma non restituisce il corretto risultato per qualche operazione di più goroutine.
    Le race condition sono perniciose perché possono rimanere latenti in un programma e apparire poco frequentemente, se non sotto un grosso carico di lavoro o quando usate in certe piattaforme o architetture.
    Questo le rende difficili da riprodurre e da diagnosticare.
    \hfill \vspace{12pt}

    Prendiamo un tipico esempio per chiarire la serietà del problema:
    \begin{lstlisting}[frame = single, label = {lst:lstlisting9-1.1}]
var balance int

func Deposit(amount int) { balance = balance + amount }

func Balance() int { return balance }
    \end{lstlisting}
    Se in questo programma si richiamano le funzioni in modo concorrente, \verb"Balance" non è più garantito essere corretto.
    \clearpage
    \newpage
    Si considerino le seguenti goroutine:
    \begin{lstlisting}[frame = single, label = {lst:lstlisting9-1.2}]
// Alice:
go func() {
    bank.Deposit(200)		     // A1
    fmt.Println("=", bank.Balance()) // A2
}()

// Bob:
go bank.Deposit(100) // B
    \end{lstlisting}
    Alice deposita \dollar 200, quindi controlla il suo bilancio, mentre Bob deposita \dollar 100.
    Fintanto che le istruzioni \verb"A1" e \verb"A2" vengono eseguite in modo concorrente con \verb"B", si può cadere in errore pensando di avere solo tre scenari (A1+A2+B, A1+B+A2, B+A1+A2).
    Il problema è che B può accadere simultaneamente ad A1, e quindi avere un risultato di questo tipo:
    \begin{lstlisting}[label={lst:lstlisting9-1.3}]
Data	race
        0
A1r     0       ... = balance + amount
B       100
A1w     200     balance = ...
A2      "= 200"
    \end{lstlisting}
    Dopo \verb"A1r", l'espressione \verb"balance + amount" vale $200$, così questo sarà il valore scritto durante \verb"A1w", anche se nel frattempo viene eseguito un'altro deposito.
    Il bilancio finale è di soli \dollar 200.
    \hfill \vspace{12pt}

    Questo programma contiene la race condition detta \textit{data race}.
    Un data race accade ogni volta che due goroutine accedono alla stessa variabile in modo concorrente e almeno uno dei due accessi sia una scrittura.
    \hfill \vspace{12pt}

    Questo problema diventa ancor più vulnerabile e quindi evidente se i data race coinvolgono una variabile di un tipo più grande, come un'interfaccia, una stringa o una slice.
    \hfill \vspace{12pt}

    Il primo modo per evitare una race condition è non fare scritture.
    Si consideri la seguente map, popolata in modo pigro in quanto ogni chiave è richiesta per la prima volta.
    Se \verb"Icon" è invocato in modo sequenziale, il programma lavora bene, ma se \verb"Icon" è invocato in modo concorrente, allora c'è un data race per l'accesso alla map.
    \begin{lstlisting}[frame = single, label = {lst:lstlisting9-1.4}]
var icons = make(map[string]image.Image)

func loadIcon(name string) image.Image

// NOTA: non %*\textit{è}*) concurrency-safe
func Icon(name string) image.Image {
    icon, ok := icons[name]
    if !ok {
        icon = loadIcon(name)
        icons[name] = icon
    }
    return icon
}
    \end{lstlisting}
    Se invece venisse inizializzata la map con tutte le entry necessarie prima di creare la seconda goroutine e non la si modificasse mai più, allora si avrebbe la concurrency-safe perché non ci sarebbero scritture.
    \clearpage
    \newpage
    \begin{lstlisting}[frame = single, label = {lst:lstlisting9-1.5}]
var icons = map[string]image.Image {
    "spades.png":   loadIcon("spades.png"),
    "hearts.png":   loadIcon("hearts.png"),
    "diamonds.png": loadIcon("diamonds.png"),
    "clubs.png":    loadIcon("clubs.png"),
}

// Concurrency-safe
func Icon(name string) image.Image { return icons[name] }
    \end{lstlisting}
    Quest'ultima implementazione rende la funzione concurrency-safe perché l'assegnazione alla map viene fatta durante l'inizializzazione del package, che \textit{avviene sempre prima} di eseguire la funzione \verb"main".
    \hfill \vspace{12pt}

    Fintanto che le goroutine non possono accedere in modo diretto alle variabili, esse devono usare i channel per inviare una richiesta alle goroutine confinate per interrogare o aggiornare una loro variabile.
    Il mantra di Go è infatti ``Do not communicate by sharing memory;
    instead, share memory by communicating".
    Una goroutine che gestisce l'accesso ad una variabile confinata tramite l'uso di richieste su channel è detta \textit{goroutine monitor} per quella variabile.
    \begin{lstlisting}[frame = single, label = {lst:lstlisting9-1.6}]
var deposits = make(chan int) // invia l'ammontare di deposito
var balances = make(chan int) // riceve il bilancio

func Deposit(amount int) { deposits <- amount }
func Balance() int       { return <-balances }

func teller() {
    var balance int // balance %*\textit{è}*) confintato alla goroutine teller
    for {
        select {
        case amount := <-deposits:
            balance += amount
        case balances <- balance:
        }
    }
}

func init() {
    go teller() // avvia la goroutine monitor
}
    \end{lstlisting}
    Anche se la variabile non può essere confinata da una singola goroutine per l'intera sua vita, il confinamento può comunque essere una soluzione al problema per l'accesso concorrente.
\end{document}
\documentclass[../../thesis.tex]{subfiles}
\begin{document}
    \subsection{Sincronizzazione di memoria}\label{subsec:sincronizzazione-di-memoria}
    Nei computer moderni ci sono una dozzina di processori, ognuno con la propria cache locale.
    Per efficienza, le scritture alla memoria sono propagate solo all'interno del processore e caricate in memoria centrale solo quando necessarie.
    Tali scritture possono anche essere caricate in memoria in ordine diverso rispetto a come sono state scritte nella goroutine.
    Le primitive di sincronizzazione come i channel di comunicazione e le operazioni di mutex fanno sì che il processore carichi tutte le scritture accumulate cosicché gli effetti dell'esecuzione della goroutine siano garantiti essere visibili alle goroutine in esecuzione sugli altri processori.
    \hfill \vspace{12pt}

    Si consideri l'output del seguente codice:
    \begin{lstlisting}[frame = single,label={lst:lstlisting9-4.1}]
var x, y int
go func() {
    x = 1                   // A1
    fmt.Print("y:", y, " ") // A2
}()
go func() {
    y = 1                   // B1
    fmt.Print("x:", x, " ") // B2
}()
    \end{lstlisting}
    Dato che le due goroutine sono concorrenti e accedono alle variabili condivise senza mutua esclusione, c'è un data race, per cui il programma non può essere deterministico.

    I possibili output:
    \begin{lstlisting}[language = bash, frame = L,label={lst:lstlisting9-4.2}]
y:0 x:1
x:0 y:1
x:1 y:1
y:1 x:1
    \end{lstlisting}
    La quarta linea può essere spiegata con la sequenza \verb"A1+B1+A2+B2" o \verb"B1+A1+A2+B2", per esempio.

    Comunque, queste due sequenze possono avere output:
    \begin{lstlisting}[language = bash, frame = L,label={lst:lstlisting9-4.3}]
x:0 y:0
y:0 x:0
    \end{lstlisting}
    Ma dipende dal compilatore, dalla CPU e da tanti altri fattori.
    Tutti questi problemi possono essere evitati con un uso consistente di semplici pattern.
\end{document}
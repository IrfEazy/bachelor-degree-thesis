%
Tutti i tipi esaminati finora sono stati \textit{tipi concreti}.
Un tipo concreto specifica la rappresentazione esatta dei suoi valori ed espone le operazioni intrinseche di tale rappresentazione, come l'aritmetica per i numeri, o l'indicizzazione, l'\verb|append| e il \verb|range| per le slice.
Un tipo concreto può anche fornire comportamenti aggiuntivi attraverso i suoi metodi.
Quando si ha un valore di tipo concreto, si sa esattamente di cosa \textit{si tratta} e cosa \textit{si può fare} con esso.

C'è un'altra tipologia di tipi in Go chiamata \textit{interfaccia}.
L'interfaccia è un \textit{tipo astratto}.
Essa non espone all'esterno la sua struttura interna e quella dei suoi valori e non rivela l'insieme base di operazioni che questi valori supportano;
solo alcuni metodi sono visibili all'esterno.
Quando si ha un valore di un tipo interfaccia, non si sa nulla di ciò che \textit{è};
si sa solo cosa può \textit{fare}, o più precisamente, i comportamenti che i suoi metodi forniscono.

Ci sono due funzioni simili per la formattazione delle stringhe: \verb|fmt.Printf|, che scrive il risultato sullo standard output (un file), e \verb|fmt.Sprintf|, che restituisce il risultato come una \verb|string|.
Sarebbe un peccato se la parte difficile, formattare il risultato, dovesse essere duplicata a causa di queste differenze superficiali nel modo in cui il risultato viene utilizzato.
Grazie alle interfacce, non succede.
Entrambe queste funzioni sono, in effetti, wrapper intorno a una terza funzione, \verb|fmt.Fprintf|, che è agnostica su ciò che accade al risultato che calcola:
\begin{lstlisting}[frame=single, label={lst:lstlisting6-1.1}]
package fmt

func Fprintf(w io.Writer, format string,
    args ...interface{}) (int, error)

func Printf(format string, args ...interface{}) (int, error) {
    return Fprintf(os.Stdout, format, args...)
}

func Sprintf(format string, args ...interface{}) string {
    var buf bytes.Buffer
    Fprintf(&buf, format, args...)
    return buf.String()
}
\end{lstlisting}
Il prefisso \verb|F| di \verb|Fprintf| sta per \textit{file} e indica che l'output formattato dovrebbe essere scritto nel file fornito come primo argomento.
Nel caso di \verb|Printf|, l'argomento, \verb|os.Stdout|, è un \verb|*os.file|.
Nel caso di \verb|Sprintf|, tuttavia, l'argomento non è un file, anche se superficialmente può somigliare: \verb|&buf| è un puntatore a un buffer di memoria in cui i byte possono essere scritti.

Il primo parametro di \verb|Fprintf| non è nemmeno un file.
È un \verb|io.Writer|, che è un tipo di interfaccia con la seguente dichiarazione:
\begin{lstlisting}[frame=single, label={lst:lstlisting6-1.2}]
package io

// Writer is the interface that wraps the basic Write method.
type Writer interface {
    // Write writes len(p) bytes from p to the underlying
    // data stream. It returns the number of bytes written from
    // p (0 <= n <= len(p)) and any error encountered that
    // caused the write to stop early. Write must return a
    // non-nil error if it returns n < len(p). Write must not
    // modify the slice data, even temporarily.
    //
    // Implementations must not retain p.
    Write(p []byte) (n int, err error)
}
\end{lstlisting}
L'interfaccia \verb|io.Writer| definisce il contratto tra \verb|Fprintf| e i suoi chiamanti.
Da un lato, il contratto richiede che il chiamante fornisca un valore di un tipo concreto come \verb|*os.File| o \verb|*bytes.Buffer| che ha un metodo chiamato \verb|Write| con la firma e il comportamento appropriati.
D'altra parte, il contratto garantisce che \verb|Fprintf| farà il suo lavoro dato qualsiasi valore che soddisfi l'interfaccia \verb|io.Writer|.
\verb|Fprintf| non può presumere che stia scrivendo su un file o una memoria, solo che possa chiamare \verb|Write|.

Si può tranquillamente passare un valore di qualsiasi tipo concreto che soddisfi \verb|io.Writer| come primo argomento a \verb|fmt.Fprintf|.
Questa libertà di sostituire un tipo con un altro che soddisfi la stessa interfaccia è chiamata \textit{sostituibilità} ed è un segno distintivo della programmazione orientata agli oggetti.

Viene presentato ora un esempio di tipo concreto che soddisfa l'interfaccia \verb|io.Writer|.
Il metodo di scrittura del tipo \verb|*ByteCounter| conta soltanto i byte scritti prima di scartarli.
\begin{lstlisting}[frame=single, label={lst:lstlisting6-1.3}]
type ByteCounter int

func (c *ByteCounter) Write(p []byte) (int, error) {
    *c += ByteCounter(len(p)) // converte int in ByteCounter
    return len(p), nil
}
\end{lstlisting}
Poiché \verb|*ByteCounter| soddisfa il contratto di \verb|io.Writer|, lo si può passare in input a \verb|Fprintf|, che fa la sua formattazione stringa ignaro di questo cambiamento;
il \verb|ByteCounter| accumula correttamente la lunghezza del risultato.
\begin{lstlisting}[frame=single, label={lst:lstlisting6-1.4}]
var c ByteCounter
c.Write([]byte(%*``*\)hello%*''*\)))
fmt.Println(c)

c = 0 // reset del contatore
var name = %*``*\)Dolly%*''*\)
fmt.Fprintf(&c, %*``*\)hello, %s%*''*\), name)
fmt.Println(c)
\end{lstlisting}
Output:
\begin{lstlisting}[language=bash, frame=L, label={lst:lstlisting6-1.5}]
5
12
\end{lstlisting}
Oltre a \verb|io.Writer|, c'è un'altra interfaccia di grande importanza per il package \verb|fmt|.
\verb|Fprintf| e \verb|Fprintln| forniscono un modo per i tipi di controllare come vengono stampati i loro valori.
Dichiarare un metodo \verb|String| rende un tipo capace di soddisfare una delle interfacce più utilizzate di tutti, \verb|fmt.Stringer|:
\begin{lstlisting}[frame=single, label={lst:lstlisting6-1.6}]
package fmt

// The String method is used to print values passed
// as an operand to any format that accepts a string
// or to an unformatted printer such as Print.
type Stringer interface {
    String() string
}
\end{lstlisting}
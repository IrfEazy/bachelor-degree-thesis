La programmazione concorrente, programma visto come composizione di numerose attività autonome, è l'elemento più importante di questi tempi.
I server web gestiscono richieste per migliaia di client contemporaneamente.
Le app degli smartphone e dei tablet gestiscono le animazioni sull'interfaccia utente mentre simultaneamente eseguono la computazione e le richieste di rete in background.
Anche i tradizionali problemi di batch (lettura dei dati, computazione e scrittura degli output) usano la concorrenza per nascondere la latenza delle operazioni I/O così da sfruttare anche i multiprocessori dei computer moderni, i quali, negli anni, continuano a crescere in numero ma non in velocità.

Go offre due stili di programmazione concorrente.
In questo capitolo vengono esponte le goroutine e i channel, compatibili con la \textit{CSP} o \textit{communicating sequential processes}, un modello di concorrenza dove i valori sono visti come attività indipendenti (goroutine), ma con le variabili confinate alle singole attività.
Nel prossimo capitolo verranno illustrati gli aspetti più tradizionali del modello \textit{shared memory multithreading}.


\section{Goroutine}
\label{sec:goroutine}%
\input{chapters/7/7-1-goroutine}


\section{Esempio: clock server concorrente}
\label{sec:esempio_clock_server_concorrente}
\input{chapters/7/7-2-esempio-clock-server-concorrente}

\section{I channel}
\label{sec:channel}
%
Se in Go le goroutine sono le attività di un programma concorrente, i channel sono le connessioni fra di loro.
Un channel è un meccanismo di comunicazione che permette di inviare valori ad un'altra goroutine.
Ogni channel è un condotto per i valori di un tipo particolare, detto \textit{tipo elementare} del channel.
Il tipo del channel che ha elementi di tipo \verb|int| è indicato come \verb|chan int|.

Per creare un channel viene usata la funzione built-in \verb|make|:
\begin{lstlisting}[frame=single, label={lst:lstlisting7-4.1}]
ch := make(chan int) // ch ha tipo `chan int'
\end{lstlisting}
Come per le map, un channel è un \textit{riferimento} ad una struttura dati creata da \verb|make|.
Quando si fa una copia ad un channel o quando viene passato come argomento ad una funzione, in realtà se ne sta copiando il riferimento, così il chiamante e il chiamato si riferiranno alla stessa struttura dati.
Come per gli altri tipi referenziati, il valore zero di un channel è il \verb|nil|.

Due channel dello stesso tipo possono essere confrontati con \verb|==|.
Il confronto è vero se entrambi sono riferimenti alla stessa struttura dati channel.
Un channel può anche essere confrontato con un \verb|nil|.

Un channel ha due operazioni principali, \textit{send} e \textit{receive}, conosciuti insieme come operazioni di \textit{communication}.
Un operazione di send trasmette, per mezzo del channel, un valore da una goroutine ad un altra goroutine che esegue una corrispondente operazione di receive.
Entrambe le operazioni sono scritte usando l'operatore \verb|<-|.
In un'operazione di send, \verb|<-| separa il channel dall'operando valore.
In un'operazione di receive \verb|<-| precede l'operando channel.
Un'operazione di receive il cui risultato non viene utilizzato è un'istruzione valida.
\begin{lstlisting}[frame=single, label={lst:lstlisting7-4.2}]
ch <- x // un'istruzione di send

x = <-ch // un'espressione di receive in un'operazione di
         // assegnamento
<-ch     // un'istruzione di receive; il risultato %*\textit{è}*\) scartato
\end{lstlisting}
I channel supportano una terza operazione, \textit{close}, che imposta un flag sul channel ad indicare che nessun valore verrà più inviato tramite esso;
tentare comunque un invio causa un panic.
Le operazioni di receive su un channel chiuso restituiscono i valori che sono stati inviati prima della chiusura fino a quando non finiscono;
qualunque successiva operazione di receive viene risolta immediatamente producendo il valore zero del tipo elementare del channel.

Per chiudere un channel, bisogna chiamare la funzione built-in \verb|close|:
\begin{lstlisting}[frame=single, label={lst:lstlisting7-4.3}]
close(ch)
\end{lstlisting}
Un channel creato con una chiamata a \verb|make| viene detto un \textit{unbuffered} channel, ma \verb|make| accetta un secondo argomento opzionale, un intero detto \textit{capacity} del channel.
Se la capacity del channel è diversa da zero, \verb|make| crea un \textit{buffered} channel.
\begin{lstlisting}[frame=single, label={lst:lstlisting7-4.4}]
ch = make(chan int)    // unbuffered channel
ch = make(chan int, 0) // unbuffered channel
ch = make(chan int, 3) // buffered channel con capacity 3
\end{lstlisting}

\subsection{Unbuffered channel}
\label{subsec:unbuffered_channel}%
Un'operazione di send su un unbuffered channel blocca la goroutine mittente fino a quando un'altra goroutine esegue una corrispondente operazione di receive sullo stesso channel, momento in cui il valore è trasmesso e entrambe le goroutine possono proseguire la loro esecuzione.
Al contrario, se l'operazione di receive viene eseguita prima, la goroutine destinataria viene bloccata fino a quando un'altra goroutine eseguirà una send sullo stesso channel.

La comunicazione su un unbuffered channel costringe le goroutine, mittente e destinatario, a \textit{sincronizzarsi}.
Per questa ragione, gli unbuffered channel sono qualche volta detti channel \textit{sincronizzati}.
Quando un valore è inviato su un unbuffered channel, la ricezione del valore \textit{avviene prima} del risveglio della goroutine mittente.

Quando \textit{x} non viene eseguito nè prima di \textit{y} nè dopo \textit{y}, si dice che \textit{x è concorrente ad y}.
Questo non vuol dire necessariamente che \textit{x} e \textit{y} sono simultanei, piuttosto che non è possibile fare ipotesi sul loro ordine d'esecuzione.

Per far sì che la main goroutine attenda la fine della goroutine in background prima di chiudere il programma, si può usare un channel per sincronizzare le due goroutine:
\begin{lstlisting}[frame=single, label={lst:lstlisting7-4-1.1}]
func main() {
   conn, err := net.Dial(%*``*\)tcp%*''*\), %*``*\)localhost:8000%*''*\))
   if err != nil {
      log.Fatal(err)
   }
   done := make(chan struct{})
   go func() {
      io.Copy(os.Stdout, conn) // NOTA: gli errori sono ignorati
      log.Println(%*``*\)done%*''*\))
      done <- struct{}{} // si avvisa la main goroutine
   }()
   mustCopy(conn, os.Stdin)
   conn.Close()
   <-done // attende la fine della goroutine in background
}

func mustCopy(dst io.Writer, src io.Reader) {
   if _, err := io.Copy(dst, src); err != nil {
      log.Fatal(err)
   }
}
\end{lstlisting}
Quando l'utente chiude lo stream di standard input, \verb|mustCopy| si conclude e la main goroutine chiama \verb|conn.Close()|, chiudendo entrambe le parti della connesione alla rete.
La chiusura del lato scrittura della connessione permette al server di vedere una condizione di end-of-file.
La chiusura del lato lettura della connessione causa alla chiamata di \verb|io.Copy| sulla goroutine in background di restituire un errore ``read from closed connection'' (lettura da una connessione chiusa).

Prima di restituire il risultato, la goroutine in background registra un messaggio, quindi invia un valore sul channel \verb|done|.
La main goroutine attende di ricevere questo valore prima di restituire anche lei il risultato.
Come risultato il programma registra sempre il messaggio ``\verb|done|'' prima di terminare sul lato client.

I messaggi inviati sui channel hanno due aspetti importanti.
Ogni messaggio ha un valore, ma qualche volta sono importanti anche la comunicazione in sè e il momento in cui questa avviene.
I messaggi sono definiti \textit{eventi} quando si desidera porre accento su quest'aspetto.
Quando l'evento non porta informazioni aggiuntive allora il suo unico obiettivo è la sincronizzazione, e in questo caso viene usato come tipo elementare del channel il tipo \verb|struct{}|, anche se è comune usare un channel di \verb|bool| o \verb|int| per lo stesso obiettivo fintanto che \verb|done <- 1| è più immediato di \verb|done <- struct{}{}|.

\subsection{Pipeline}
\label{subsec:pipeline}%
I channel possono essere usati per connettere le goroutine insieme cosicché l'output di uno è l'input dell'altro.
Questi è detto \textit{pipeline}.
Il seguente programma consiste di tre goroutine connesse da due channel, come mostrato schematicamente in figura.
\begin{center}
    \includegraphics[width=0.8\linewidth]{figures/figure7.1}
\end{center}
La prima goroutine, \textit{counter}, genera gli interi 0, 1, 2, \ldots, e li invia su un channel alla seconda goroutine, \textit{squarer}, che riceve ogni valore, lo eleva al quadrato, e invia il risultato su un altro channel alla terza goroutine, \textit{printer}, che riceve i valori e li stampa.
Per chiarezza di quest'esempio, si è intenzionalmente scelto una funzione davvero semplice per spiegare il funzionamento della pipeline.
\begin{lstlisting}[frame=single, label={lst:lstlisting7-4-2.1}]
func main() {
    naturals := make(chan int)
    squares := make(chan int)

    // Counter
    go func() {
        for x := 0; ; x++ {
            naturals <- x
        }
    }()

    // Squarer
    go func() {
        for {
            x := <-naturals
            squares <- x * x
        }
    }()

    // Printer (nella main goroutine)
    for {
        fmt.Println(<-squares)
    }
}
\end{lstlisting}
Se il mittente conosce che non ci sono più valori da inviare sul channel, è bene comunicarlo alla goroutine destinataria così da non farlo attendere.
Questo è realizzato con la \textit{chiusura} del channel con la funzione built-in \verb|close|:
\begin{lstlisting}[frame=single, label={lst:lstlisting7-4-2.2}]
close(naturals)
\end{lstlisting}
Dopo che un channel è stato chiuso, qualunque altra operazione di send restituirà un panic.
Dopo che il channel chiuso sia stato \textit{svuotato}, ovvero dopo che l'ultimo elemento inviato sia stato recepito, tutte le successive operazioni di receive verranno risolte restituendo un valore zero per il tipo elementare del channel.
Chiudendo il channel \verb|naturals| si causerà al ciclo di squarer di ricevere uno stream senza fine di valori zero, e di inviare tali valori al printer.

Non esiste un modo diretto per capire se un channel è stato chiuso, ma esiste una variante dell'operazione di receive che produce due risultati: l'elemento ricevuto dal channel più un valore booleano, denominato per convenzione \verb|ok|, che è \verb|true| nel caso di un'operazione di receive completata con successo e \verb|false| nel caso di un'operazione di receive su un channel chiuso e svuotato.
Usando questa funzionalità, il ciclo di squarer può essere modificato così da interromperlo quando il channel \verb|naturals| è svuotato e chiudere a cascata il channel \verb|squares|.
\begin{lstlisting}[frame=single, label={lst:lstlisting7-4-2.3}]
go func() {
    for {
        x, ok := <-naturals
        if !ok {
            break // il channel %*\textit{è}*\) stato chiuso e svuotato
        }
        squares <- x * x
    }
    close(squares)
}()
\end{lstlisting}
Il linguaggio permette di usare un ciclo \verb|range| per iterare anche su un channel.
Questo è sintatticamente più conveniente per ricevere tutti i valori inviati su un channel e terminare il ciclo dopo l'ultimo.
\begin{lstlisting}[frame=single, label={lst:lstlisting7-4-2.4}]
func main() {
    naturals := make(chan int)
    squares := make(chan int)

    // Counter
    go func() {
        for x := 0; x < 100; x++ {
            naturals <- x
        }
        close(naturals)
    }()

    // Squarer
    go func() {
        for x := range naturals {
            squares <- x * x
        }
        close(squares)
    }()

    // Printer (nella main goroutine)
    for x := range squares {
        fmt.Println(x)
    }
}
\end{lstlisting}
Non è sempre necessario chiudere un channel, lo è nel momento in cui è importante avvisare la goroutine destinaria che tutti i valori sono stati inviati, altrimenti può anche essere tralasciato perché il garbage collector si occuperà di determinare quando un channel è diventato irraggiungibile e quindi riciclarlo anche se non chiuso.
(Questo discorso non vale per i file;
i file devono sempre essere chiusi tramite chiamata al metodo \verb|Close|).

\subsection{Tipi di channel unidirezionali}
\label{subsec:tipi_di_channel_unidirezionali}%
Non appena un programma cresce, è naturale spezzare grandi funzioni in pezzi più piccoli.
La funzione \verb|main| proposta alle pagine precedenti può essere suddivisa in tre funzioni:
\begin{lstlisting}[frame=single, label={lst:lstlisting7-4-3.1}]
func counter(out chan int)
func squarer(out, in chan int)
func printer(in chan int)
\end{lstlisting}
La funzione \verb|squarer|, posto in mezzo alla pipeline, riceve due parametri, il channel di input e il channel di output.
Entrambi hanno lo stesso tipo, ma i loro usi opposti: \verb|in| è solo per ricevere, mentre \verb|out| solo per inviare.
I nomi \verb|in| e \verb|out| rafforzano questa idea, ma nulla vieta a \verb|squarer| di inviare su \verb|in| e di ricevere da \verb|out|.

Quando un channel è passato come parametro di una funzione, è quasi sempre dato con l'intento di usarlo esclusivamente per inviare o esclusivamente per ricevere.

Per documentare questo intento e prevenire un uso scorretto, il type system di Go offre i tipi di channel \textit{unidirezionali} per permettere solo una fra le operazioni di send e receive.
Il tipo \verb|chan<- int|, un channel di \textit{solo invio} di \verb|int|, permette solo gli invii.
Al contrario, il tipo \verb|<-chan int|, un channel di \textit{sola ricezione} di \verb|int|, permette solo le ricezioni.
Violazioni di questi usi sono individuati a compile time.

Dato che l'operazione di \verb|close| asserisce che nessuna operazione di send verrà più effettuata su un channel, allora solo la goroutine mittente potrà chiamarla, e per questa ragione è un errore a compile time provare a chiudere un channel di sola ricezione.
\begin{lstlisting}[frame=single, label={lst:lstlisting7-4-3.2}]
func counter(out chan<- int) {
    for x := 0; x < 100; x++ {
        out <- x
    }
    close(out)
}

func squarer(out chan<- int, int <-chan int) {
    for v := range in {
        out <- v * v
    }
    close(out)
}

func printer(in <-chan int) {
    for v := range in {
        fmt.Println(v)
    }
}
\end{lstlisting}
\begin{lstlisting}[frame=single, label={lst:lstlisting7-4-3.3}]
func main() {
    naturals := make(chan int)
    squares := make(chan int)

    go counter(naturals)
    go squarer(squares, naturals)
    printer(squares)
}
\end{lstlisting}
La chiamata \verb|counter(naturals)| converte \verb|naturals| dal tipo \verb|chan int| al tipo \verb|chan<- int| in modo implicito.
La chiamata a \verb|printer(squares)| esegue implicitamente una conversione simile a \verb|<-chan int|.
Le conversioni dei tipi di channel da bidirezionali a unidirezionali sono permesse in ogni assegnazione.
Non esiste però modo di tornare indietro: un channel convertito in unidirezionale non può più tornare ad essere bidirezionale.

\subsection{Buffered channel}
\label{subsec:buffered_channel}%
Un buffered channel ha una coda di elementi.
La dimensione massima della coda è decisa quando viene creato il channel, dall'argomento capacity in \verb|make|.
La seguente istruzione crea un buffered channel in grado di ospitare al più tre valori \verb|string|.
La figura mostra anche una rappresentazione di \verb|ch| e del channel a cui si riferisce.
\begin{lstlisting}[frame=single, label={lst:lstlisting7-4-4.1}]
ch = make(chan string, 3)
\end{lstlisting}
\begin{center}
    \includegraphics[width=0.5\linewidth]{figures/figure7.2}
\end{center}
Un'operazione di send su un buffered channel inserisce un elemento in coda alla fila d'attesa , e l'operazione di receive rimove un elemento dalla testa della fila.
Se il channel è pieno, l'operazione di send blocca la propria goroutine fino a quando non viene fatto spazio in seguito all'operazione di receive di un'altra goroutine.
Al contrario, se il channel è vuoto, un'operazione di receive blocca la propria goroutine fino a quando un'altra goroutine non esegue un'operazione di send sullo stesso channel.
\begin{lstlisting}[frame=single, label={lst:lstlisting7-4-4.2}]
ch <- %*``*\)A%*''*\)
ch <- %*``*\)B%*''*\)
ch <- %*``*\)C%*''*\)
\end{lstlisting}
Dopo aver eseguito queste tre istruzioni, il channel è pieno e una quarta operazione di send verrà bloccata.
\begin{center}
    \includegraphics[width=0.5\linewidth]{figures/figure7.3}
\end{center}
Se si riceve un valore,
\begin{lstlisting}[frame=single, label={lst:lstlisting7-4-4.3}]
fmt.Println(<-ch)
\end{lstlisting}
Output:
\begin{lstlisting}[language=bash, frame=L, label={lst:lstlisting7-4-4.4}]
A
\end{lstlisting}
il channel non sarà più nè pieno nè vuoto, quindi sia un'operazione di send che una di receive potranno procedere senza essere bloccati.
In questo modo, il buffer del channel separa le goroutine mittenti e destinatari.
\begin{center}
    \includegraphics[width=0.5\linewidth]{figures/figure7.4}
\end{center}
Nel caso un programma voglia conoscere la capacità del buffer del channel può richiamare la funzione \verb|cap|:
\begin{lstlisting}[frame=single, label={lst:lstlisting7-4-4.5}]
fmt.Println(cap(ch))
\end{lstlisting}
Output:
\begin{lstlisting}[language=bash, frame=L, label={lst:lstlisting7-4-4.6}]
3
\end{lstlisting}
Quando applicato al channel, la funzione \verb|len| restituisce il numero di elementi al momento presenti nel buffer.

Il seguente esempio presenta un'applicazione di un buffered channel.
Tale programma fa richieste in parallelo a tre \textit{mirror}, ovvero server equivalenti ma geograficamente distribuiti.
Essi inviano la loro risposta su un buffered channel, quindi ricevono e restituiscono solo la prima risposta, che è la più veloce ad arrivare.
Quindi \verb|mirrorQuery| restituisce un risultato anche prima della risposta dei due server più lenti.
\begin{lstlisting}[frame=single, label={lst:lstlisting7-4-4.7}]
func mirrorQuery() string {
responses := make(chan string, 3)
    go func() { responses <- request(%*``*\)asia.gopl.io%*''*\)) }()
    go func() { responses <- request(%*``*\)europe.gopl.io%*''*\)) }()
    go func() { responses <- request(%*``*\)americas.gopl.io%*''*\)) }()
    return <-responses // Ritorna la risposta pi%*\textit{ù}*\) veloce
}

func request(hostname string) (response string) { /* ... */ }
\end{lstlisting}
La scelta fra un unbuffered e buffered channel e la scelta della capacity per il buffered channel possono entrambe influenzare la correttezza del programma.
Gli unbuffered channel offrono una forte garanzia di sincronizzazione perché ogni operazione di send è sincronizzata con la corrispettiva operazione di receive;
con i buffered channel, queste operazioni sono slegate.
Inoltre, quando si conosce la dimensione massima del numero di valori che possono essere inviati su un channel, è utile creare un buffered channel di quella dimensione ed effettuare tutti i send ancor prima che il primo valore sia ricevuto dal destinatario.
Un fallimento nell'allocazione di un buffer di sufficiente capacity potrebbe portare il programma ad un deadlock.

\section{Esempio: Chat Server}
\label{sec:esempio_chat_server}
\input{chapters/7/7-4-esempio-chat-server}


\documentclass[../thesis.tex]{subfiles}
\begin{document}
    \newpage
    \section*{Introduzione}
    \begin{quotation}
        \begin{small}
            \textit{Go is an open-source programming language that makes it easy to build simple, reliable, and efficient software} (Dal sito web \verb"golang.org" di Go)
        \end{small}
    \end{quotation}

    Go è un progetto ideato nel 2007 da Robert Griesemer, Rob Pike e Ken Thompson di Google, ed è stato annunciato nel novembre 2009.
    Go è open-source, quindi il codice sorgente del suo compilatore, le sue librerie e i suoi strumenti sono disponibili gratuitamente a chiunque.
    I suoi programmi possono essere eseguiti su sistemi discendenti da Unix e Microsoft Windows e, in generale, i programmi scritti in uno degli ambienti di sviluppo funzionano anche su altri ambienti senza modifiche precedenti.
    \hfill \vspace{12pt}

    Il linguaggio Go e i suoi strumenti associati hanno l'obiettivo di essere espressivo, efficiente ed efficace.
    Per attirare i programmatori sul mercato, Go è stato implementato in un modo che assomiglia al linguaggio di programmazione C. Come linguaggio di programmazione, si adatta e prende in prestito buone idee da molti altri linguaggi, evitando i dettagli che porterebbero a un codice complesso e inaffidabile.
    Rispetto ad altri linguaggi di programmazione, la libreria di gestione della concorrenza dei processi è nuova ed efficiente e il suo approccio all'astrazione dei dati e alla programmazione orientata agli oggetti è insolitamente flessibile.
    Go è inoltre dotato di un \textit{garbage collector} per la gestione automatica della memoria da parte dei programmi, nonostante sia un linguaggio compilato.
    \hfill \vspace{12pt}

    Go è particolarmente adatto per la creazione di sistemi informatici come server di rete e strumenti e sistemi di servizio per programmatori, ma è un linguaggio molto versatile anche per altri scopi, infatti viene utilizzato nei settori della grafica, delle applicazioni mobili e del machine learning.
    \subfile{0-prefazione-subsection/1-origine-go.tex}
    \subfile{0-prefazione-subsection/2-progetto-go.tex}
    \clearpage
\end{document}
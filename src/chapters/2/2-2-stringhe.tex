%
Una stringa è una sequenza immutabile di byte.
Le stringhe possono contenere dati arbitrari, inclusi byte con valore \verb|0|, ma di solito contengono testo leggibile dall'uomo.
Le stringhe di testo sono convenzionalmente interpretate come sequenze codificate UTF-8 di punti di codice Unicode (rune).
\begin{lstlisting}[frame = single, label = {lst:lstlisting2.1}]
s := %*``*\)hello, world%*''*\)
fmt.Println(len(s))
fmt.Println(s[0], s[7])
\end{lstlisting}
Output:
\begin{lstlisting}[language = bash, frame = L, label = {lst:lstlisting2.2}]
12
104 119 (`h' and `w')
\end{lstlisting}
Il tentativo di accedere a un byte al di fuori di questo intervallo si traduce in un \verb|panic|:
\begin{lstlisting}[frame = single, label = {lst:lstlisting2.3}]
C := s[len(s)] // panic: indice fuori dal range
\end{lstlisting}
L'operazione di \textit{sottostringa} \verb|s[i:j]| produce una nuova stringa composta dai byte della stringa originale che inizia dall'indice \verb|i| e continua fino, ma non include, il byte all'indice \verb|j|.
Il risultato contiene \verb|j-i| byte.
\begin{lstlisting}[frame = single, label = {lst:lstlisting2.4}]
fmt.Println(s[0:5])
\end{lstlisting}
Output:
\begin{lstlisting}[language = bash, frame = L, label = {lst:lstlisting2-2.5}]
hello
\end{lstlisting}
Uno o entrambi gli operandi \verb|i| e \verb|j| possono essere omessi, nel qual caso i valori predefiniti di \verb|0| (l'inizio della stringa) e \verb|len(s)| (la sua fine) sono assunti, rispettivamente.

L'operatore \verb|+| crea una nuova stringa concatenando due stringhe.
Le stringhe possono essere confrontate con operatori di confronto come \verb|==| e \verb|<|;
il confronto viene fatto byte per byte, quindi il risultato è l'ordinamento lessicografico naturale.

I valori di stringa sono immutabili: la sequenza di byte contenuta in un valore di stringa non può mai essere modificata, anche se ovviamente possiamo assegnare un nuovo valore a una \textit{variabile} di stringa.

Immutabilità significa che è sicuro per due copie di una stringa condividere la stessa memoria sottostante, il che rende conveniente copiare le stringhe di qualsiasi lunghezza.
Allo stesso modo, una stringa \verb|s| e una sottostringa \verb|s[7:]| possono tranquillamente condividere gli stessi dati, quindi anche l'operazione di sottostringa è economica.
Nessuna nuova memoria viene allocata in entrambi i casi.

\subsection{Stringhe costanti}
\label{subsec:stringhe_costanti}%
Poiché i file sorgente Go sono sempre codificati in UTF-8 e le stringhe di testo Go sono convenzionalmente interpretate come UTF-8, possiamo includere punti di codice Unicode fra i caratteri di stringa.


All'interno delle virgolette di una stringa letterale, le \textit{sequenze di escape} che iniziano con una barra rovesciata \verb|\| possono essere utilizzate per inserire valori di byte arbitrari nella stringa.

I byte arbitrari possono anche essere inclusi in stringhe letterali usando escape esadecimali o ottali.
Un \textit{escape esadecimale} è scritta \verb|\x|\textit{hh}, con esattamente due cifre esadecimali \textit{h} (in maiuscolo o minuscolo).
Un \textit{escape ottale} è scritta \verb|\x|\textit{ooo} con esattamente tre cifre ottali \textit{o} (da \verb|0| a \verb|7|) non superiori a \verb|\377|.
Entrambi denotano un singolo byte con il valore specificato.

\subsection{Unicode}
\label{subsec:unicode}%
Unicode (\verb|unicode.org|) raccoglie tutti i caratteri di tutti i sistemi di scrittura del mondo, oltre a accenti e altri segni diacritici, codici di controllo come tabulazione e ritorno a capo, e un sacco di esoterica, e assegna a ciascuno un numero standard chiamato \textit{punto di codice Unicode} o, nella terminologia Go, una \textit{runa}.

Il tipo di dati naturale per contenere una singola runa è \verb|int32|.
Potremmo rappresentare una sequenza di rune come una sequenza di valori \verb|int32|.
In questa rappresentazione, chiamata UTF-32 o UCS-4, la codifica di ogni punto di codice Unicode ha la stessa dimensione, 32 bit.
Questo è semplice e uniforme, ma utilizza molto più spazio del necessario poiché la maggior parte del testo leggibile dal computer è in ASCII, che richiede solo 8 bit (quindi 1 byte) per carattere.

\subsection{UTF-8}
\label{subsec:utf8}%
UTF-8 è una codifica a lunghezza variabile dei punti di codice Unicode (unità di byte).
UTF-8 è stato inventato da Ken Thompson e Rob Pike, due dei creatori di Go, ed è ora uno standard Unicode.
Usa tra 1 e 4 byte per rappresentare ogni runa, ma solo 1 byte per i caratteri ASCII\@.
I bit di ordine elevato del primo byte della codifica di una runa indicano quanti byte seguono.
Un \verb|0| di alto ordine indica un ASCII a 7 bit, dove ogni runa prende solo 1 byte, quindi è identico all'ASCII convenzionale.
Un ordine alto \verb|110| indica che la runa prende 2 byte;
il secondo byte inizia con \verb|10|.
Le rune più grandi hanno codifiche analoghe.

Una codifica a lunghezza variabile preclude l'indicizzazione diretta per accedere al carattere \textit{n}-esimo di una stringa, ma UTF-8 ha molte altre proprietà apprezzabili a compensare tale difetto.
La codifica è compatta, compatibile con ASCII e auto-sincronizzante: è possibile trovare l'inizio di un carattere eseguendo il backup di non più di tre byte.
Può essere decodificato da sinistra a destra senza alcuna ambiguità o necessità di un lookahead.
La codifica di nessuna runa è una sottostringa di qualsiasi altra, o di una sequenza di altre, quindi una runa è individuabile semplicemente cercando i suoi byte, senza preoccuparsi del contesto precedente.

Il pacchetto \verb|unicode| fornisce funzioni per lavorare con singole rune (come distinguere le lettere dai numeri, o convertire una lettera maiuscola in una minuscola), e il pacchetto \verb|unicode/utf8| fornisce funzioni per codificare e decodificare le rune come byte usando UTF-8.

Molti caratteri Unicode sono difficili da digitare su una tastiera o da distinguere visivamente da caratteri simili;
alcuni sono addirittura invisibili, gli escape Unicode in Go ci permettono di specificarli in base al loro valore numerico.
Ci sono due forme, \verb|\u|\textit{hhhhhh} per un valore a 16 bit e \verb|\U|\textit{hhhhhhhh} per un valore a 32 bit, dove ogni \textit{h} è una cifra esadecimale.

\subsection{Stringhe e byte slices}
\label{subsec:stringhe_e_byte_slices}%
I pacchetti standard particolarmente importanti per manipolare le stringhe sono quattro: \verb|bytes|, \verb|strings|, \verb|strconv| e \verb|unicode|.
Il pacchetto \verb|strings| fornisce molte funzioni per la ricerca, la sostituzione, il confronto, il taglio, la divisione e l'unione delle stringhe.

Il pacchetto \verb|bytes| ha funzioni simili per manipolare porzioni di byte, di tipo \verb|[]byte|, che condividono alcune proprietà con le stringhe.
Poiché le stringhe sono immutabili, la creazione di stringhe in modo incrementale può comportare un sacco di allocazione e copie.
In questi casi, è più efficiente usare il tipo \verb|bytes.Buffer|.

Una stringa contiene un array di byte che, una volta creata, è immutabile.
Al contrario, gli elementi di una slice di byte possono essere liberamente modificati.
Le stringhe possono essere convertite in byte slice e viceversa:
\begin{lstlisting}[frame = single, label = {lst:lstlisting2-2-4.1}]
s := %*``*\)abc%*''*\)
b := []byte(s)
s2 := string(b)
\end{lstlisting}
Concettualmente, la conversione \verb|[]byte(s)| alloca un nuovo array di byte che contiene una copia dei byte di \verb|s|, e produce una slice che fa riferimento alla totalità di quell'array.
In generale la copia è necessaria per garantire che i byte di \verb|s| rimangano invariati nel caso quelli di \verb|b| vengano successivamente modificati.
La conversione da byte slice a string con \verb|string(b)| fa anch'esso una copia, per assicurare l'immutabilità della stringa risultante \verb|s2|.

Per evitare conversioni e allocazioni di memoria non necessarie, molte delle funzioni di utilità nel pacchetto \verb|bytes| parallelizzano direttamente le loro controparti nel pacchetto \verb|strings|.

Il pacchetto \verb|bytes| fornisce il tipo \verb|Buffer| per una manipolazione efficiente delle slice byte.
Un \verb|Buffer| inizia vuoto, ma cresce accumulando dati di tipi quali \verb|string|, \verb|byte| e \verb|[]byte|.
Le costanti sono espressioni il cui valore è noto al compilatore e la cui valutazione è garantita che si verifichi al momento della compilazione, non al tempo di esecuzione.
Il tipo sottostante di ogni costante è un tipo di base: booleano, stringa o numero.

Una dichiarazione \verb|const| definisce valori denominati che sembrano sintatticamente variabili ma il cui valore è costante, il che impedisce cambiamenti accidentali (o nefasti) durante l'esecuzione del programma.
Per esempio, una costante è più appropriata di una variabile per una costante matematica come \verb|pi|, poiché il suo valore non cambierà:
\begin{lstlisting}[frame = single, label = {lst:lstlisting2-3.1}]
const pi = 3.14159 // approssimato; math.Pi %*\textit{è}*\) migliore
\end{lstlisting}
Quando una sequenza di costanti è dichiarata come un gruppo, l'espressione a destra può essere omessa per tutti tranne che per il primo del gruppo, il che implica che l'espressione precedente e il suo tipo debbano essere nuovamente usati.
Ad esempio:
\begin{lstlisting}[frame = single, label = {lst:lstlisting2-3.2}]
const (
    a = 1
    b
    c = 2
    d
)

fmt.Println(a, b, c, d)
\end{lstlisting}
Output:
\begin{lstlisting}[language = bash, frame = L, label = {lst:lstlisting2-3.3}]
1 1 2 2
\end{lstlisting}
Tuttavia, questo non è molto utile nel caso la copia implicita dell'espressione del lato destro valuti sempre la stessa cosa, ma se potesse variare?
Otteniamo \verb|iota|.

\subsection{Il generatore costante iota}
\label{subsec:il_generatore_costante_iota}%
Una dichiarazione \verb|const| può usare il \textit{generatore costante} \verb|iota|, che viene usato per creare una sequenza di valori correlati senza specificare esplicitamente ciascuno di essi.
In una dichiarazione \verb|const|, il valore di \verb|iota| inizia a zero e aumenta di uno per ogni elemento della sequenza.

Tipi di questo tipo sono spesso chiamati \textit{enumerazioni}, o \textit{enum} in breve.
\begin{lstlisting}[frame = single, label = {lst:lstlisting2-3-1.1}]
type Weekday int

const (
    Sunday Weekday = iota
    Monday
    Tuesday
    Wednesday
    Thursday
    Friday
    Saturday
)
\end{lstlisting}
Questo dichiara \verb|Sunday| essere 0, \verb|Monday| essere 1, e così via.

Possiamo usare \verb|iota| anche in espressioni più complesse, per esempio possiamo assegnare ai 5 bit più bassi di un intero senza segno un nome distinto e un'interpretazione booleana:
\begin{lstlisting}[frame = single, label = {lst:lstlisting2-3-1.2}]
type Flags uint

const (
    FlagUp Flags = 1 << iota // %*\textit{è}*\) up
    FlagBroadcast            // supporta la capacit%*\textit{à}*\) di accesso
                             // broadcast
    FlagLoopback             // %*\textit{è}*\) un'interfaccia di loopback
    FlagPointToPoint         // appartiene a un collegamento
                             // punto a punto
    FlagMulticast            // supporta la capacit%*\textit{à}*\) di accesso
                             // multicast
)
\end{lstlisting}
Così come incrementa \verb|iota|, ad ogni costante viene assegnato il valore di \verb|1 << iota|, che valuta le potenze di due, ciascuno corrispondente ad un singolo bit.
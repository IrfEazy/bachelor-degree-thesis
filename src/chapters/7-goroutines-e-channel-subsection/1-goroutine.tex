\documentclass[../../thesis.tex]{subfiles}
\begin{document}
    \subsection{Goroutine}\label{subsec:goroutine}
    In Go, ogni attività concorrente in esecuzione è detta \textit{goroutine}.
    Dato un programma con due funzioni, uno che svolge operazioni di computazione e l'altra che scrive qualche dato in output, si supponga che entrambi non si richiamino a vicenda.
    Un programma sequenziale potrebbe chiamare una funzione e solo in seguito l'altra, ma in un programma \textit{concorrente} con due o più goroutine, le chiamate ad \textit{entrambe} le funzioni possono essere svolte contemporaneamente.
    \hfill \vspace{12pt}

    Si può pensare per ora alle goroutine come a thread perché la loro differenza è essenzialmente in termini quantitativi e non qualitativi.
    \hfill \vspace{12pt}

    Quando un programma viene avviato, la sua unica goroutine è quella che richiama la funzione \verb"main", definita \textit{main goroutine}.
    Le nuove goroutine sono create dall'istruzione \verb"go".
    Sintatticamente, un'istruzione \verb"go" è una funzione ordinaria o una chiamata ad un metodo con la parola chiave \verb"go" come prefisso.
    Un'istruzione \verb"go" permette alla funzione di essere chiamata da una nuova e goroutine.
    \begin{lstlisting}[frame = single,label={lst:lstlisting7-1.1}]
f()    // chiama f(); attende il risultato
go f() // crea una nuova goroutine che chiama f(); non c'%*\textit{è}*) attesa
    \end{lstlisting}
    Nel seguente esempio, la main goroutine computa il 45esimo numero di Fibonacci.
    Dato che utilizza un algoritmo inefficiente, essa rimarrà in esecuzione per un tempo considerevole, durante il quale si vorrebbe dire all'utente che il programma è ancora in esecuzione tramite la stampa di uno ``spinner" testuale.
    \begin{lstlisting}[frame = single,label={lst:lstlisting7-1.2}]
func main() {
    go spinner(100 * time.Millisecond)
    const n = 45
    fibN := fib(n) // lento 
    fmt.Printf("\rFibonacci(%d) = %d\n", n, fibN)
}

func spinner(delay time.Duration) {
    for {
        for _, r := range `-\|/` {
            fmt.Printf("\r%c", r)
            time.Sleep(delay)
        }
    }
}

func fib(x int) int {
    if x < 2 {
        return x
    }
    return fib(x-1) + fib(x-2)
}
    \end{lstlisting}
    Una volta computato il risultato, la funzione \verb"main" lo stampa e quindi effettua il return.
    Quando questo succede, tutte le goroutine vengono bruscamente terminate e il programma si chiude.
    Non esistono altri modi per terminare una goroutine se non chiudere il programma o effettuare il return della funzione \verb"main", questo perché una goroutine non può terminarne un'altra;
    esistono tuttavia possibilità di metterle in comunicazione.
\end{document}
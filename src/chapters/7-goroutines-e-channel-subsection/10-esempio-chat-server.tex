\documentclass[../../thesis.tex]{subfiles}
\begin{document}
    \clearpage
    \newpage

    \subsection{Esempio: Chat Server}\label{subsec:esempio:-chat-server}
    Un chat server permette a molti utenti di inviarsi l'un l'altro messaggi testuali.
    Ci sono quattro tipologie di goroutine in questo programma.
    C'è un'istanza a testa per la goroutine \verb"main" e \verb"broadcast", e per ogni connessione client c'è una goroutine \verb"handleConn" e una \verb"clientWriter".
    \hfill \vspace{12pt}

    Il compito della main goroutine è rimanere in ascolto e di accettare connessioni di rete in input da parte di client.
    Per ognuno di essi, il main crea una nuova goroutine \verb"handleConn".
    \begin{lstlisting}[frame = single, label = {lst:lstlisting7-10.1}]
func main() {
    listener, err := net.Listen("tcp", "localhost:8000")
    if err != nil {
        log.Fatal(err)
    }

    go broadcaster()
    for {
        conn, err := listener.Accept()
        if err != nil {
            log.Print(err)
            continue
        }
        go handleConn(conn)
    }
}
    \end{lstlisting}
    Il broadcaster ha una variabile locale \verb"clients" che tiene registrato l'insieme di tutti i client correntemente connessi.
    Per ogni client viene memorizzata solo l'identità del loro channel di messaggi in uscita.
    \begin{lstlisting}[frame = single, label = {lst:lstlisting7-10.2}]
type client chan<- string // un channel di messaggi in uscita

var (
    entering = make(chan client)
    leaving  = make(chan client)
    messages = make(chan string) // tutti i messaggi client in arrivo
)

func broadcaster() {
    clients := make(map[client]bool) // tutti i client connessi
    for {
        select {
        case msg := <-messages:
            // Trasmette il messaggio in entrata ai
            // channel dei messaggi in uscita di tutti i client.
            for cli := range clients {
                cli <- msg
            }
        case cli := <-entering:
            clients[cli] = true
        case cli := <-leaving:
            delete(clients, cli)
            close(cli)
        }
    }
}
    \end{lstlisting}
    Il broadcaster rimane in ascolto sui channel globali \verb"entering" e \verb"leaving" per annunci sull'arrivo o sulla partenza di client.
    Quando riceve un annuncio di questi eventi, aggiorna l'insieme \verb"clients", e se l'evento è di partenza, chiude il channel di messaggi in uscita del client corrispondente.
    Il broadcaster rimane anche in ascolto di eventi sul channel globale \verb"messages", dove ogni client invia tutti i suoi messaggi in arrivo.
    Quando il broadcaster riceve uno di questi eventi, invia a tutti i client connessi il messaggio.
    \hfill \vspace{12pt}

    Ora si veda la goroutine dedicata al client sul server.
    La funzione \verb"handleConn" crea un channel di messaggi in uscita per il proprio client e annuncia l'arrivo dello stesso client al broadcaster sul channel \verb"entering".
    A questo punto legge ogni linea di testo dal client, invia ogni linea al broadcaster sul channel di messaggi in arrivo, aggiungendo come prefisso ad ogni messaggio l'identità del mittente.
    Una volta che non c'è più nulla da leggere dal client, \verb"handleConn" annuncia la partenza del client sul channel \verb"leaving" e chiude la connessione.
    \begin{lstlisting}[frame = single, label = {lst:lstlisting7-10.3}]
func handleConn(conn net.Conn) {
    ch := make(chan string) // messaggi client in uscita
    go clientWriter(conn, ch)

    who := conn.RemoteAddr().String()
    ch <- "You are " + who
    messages <- who + " has arrived"
    entering <- ch

    input := bufio.NewScanner(conn)
    for input.Scan() {
        messages <- who + ": " + input.Text()
    }
    // NOTA: si ignorano potenziali errori di input.Err()

    leaving <- ch
    messages <- who + " has left"
    conn.Close()
}

func clientWriter(conn net.Conn, ch <-chan string) {
    for msg := range ch {
        fmt.Fprintln(conn, msg) // NOTA: si ignorano gli errori di rete
    }
}
    \end{lstlisting}
    In aggiunta, \verb"handleConn" crea una goroutine \verb"clientWriter" per ogni client che riceve un messaggio in broadcast sul channel di messaggi in uscita del client e li scrive sulla connessione di rete del client.
    Il ciclo del writer del client termina quando il broadcaster chiude il channel dopo aver ricevuto la notifica di \verb"leaving".
    \hfill \vspace{12pt}

    Per il lato client invece viene utilizzato il programma esposto nel paragrafo \textbf{Unbuffered channel} di questo capitolo.
    \clearpage
    \newpage

    Una simulazione da linea di comando:
    \begin{lstlisting}[language = bash, frame = L, label = {lst:lstlisting7-10.4}]
$ go build chat
$ go build netcat3
$ ./chat &
$ ./netcat3
You are 127.0.0.1:64208         $ ./netcat3
127.0.0.1:64211 has arrived     You are 127.0.0.1:64211
Hi!
127.0.0.1:64208: Hi!            127.0.0.1:64208: Hi!
                                Hi yourself.
127.0.0.1:64211 Hi yourself.    127.0.0.1:64211: Hi yourself.
^C
                                127.0.0.1:64208 has left
$ ./netcat3
You are 127.0.0.1:64216         127.0.0.1:64216 has arrived
                                Welcome.
127.0.0.1:64211: Welcome.       127.0.0.1:64211: Welcome.
                                ^C
127.0.0.1:64211 has left
    \end{lstlisting}
    Mentre il server ospita una sessione di chat per $n$ client, il programma esegue $2n+2$ goroutine comunicanti in modo concorrente, anche se necessita di un'esplicita operazione di lock.
    La \verb"clients" map è confinata in una singola goroutine, nel broadcaster, così non può essere acceduta in modo concorrente.
    Le uniche variabili condivise fra più goroutine sono i channel e le istanze di \verb"net.Conn", i quali sono entrambi \textit{concurrency safe}.
\end{document}
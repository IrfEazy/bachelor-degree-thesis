\documentclass[../../thesis.tex]{subfiles}
\begin{document}
    \subsection{I channel}\label{subsec:i-channel}
    Se in Go le goroutine sono le attività di un programma concorrente, i channel sono le connessioni fra di loro.
    Un channel è un meccanismo di comunicazione che permette di inviare valori ad un'altra goroutine.
    Ogni channel è un condotto per i valori di un tipo particolare, detto \textit{tipo elementare} del channel.
    Il tipo del channel che ha elementi di tipo \verb"int" è indicato come \verb"chan int".
    \hfill \vspace{12pt}

    Per creare un channel viene usata la funzione built-in \verb"make":
    \begin{lstlisting}[frame = single,label={lst:lstlisting7-4.1}]
ch := make(chan int) // ch ha tipo `chan int`
    \end{lstlisting}
    Come per le map, un channel è un \textit{riferimento} ad una struttura dati creata da \verb"make".
    Quando si fa una copia ad un channel o quando viene passato come argomento ad una funzione, in realtà se ne sta copiando il riferimento, così il chiamante e il chiamato si riferiranno alla stessa struttura dati.
    Come per gli altri tipi referenziati, il valore zero di un channel è il \verb"nil".
    \hfill \vspace{12pt}

    Due channel dello stesso tipo possono essere confrontati con \verb"==".
    Il confronto è vero se entrambi sono riferimenti alla stessa struttura dati channel.
    Un channel può anche essere confrontato con un \verb"nil".
    \hfill \vspace{12pt}

    Un channel ha due operazioni principali, \textit{send} e \textit{receive}, conosciuti insieme come operazioni di \textit{communication}.
    Un operazione di send trasmette, per mezzo del channel, un valore da una goroutine ad un altra goroutine che esegue una corrispondente operazione di receive.
    Entrambe le operazioni sono scritte usando l'operatore \verb"<-".
    In un'operazione di send, \verb"<-" separa il channel dall'operando valore.
    In un'operazione di receive \verb"<-" precede l'operando channel.
    Un'operazione di receive il cui risultato non viene utilizzato è un'istruzione valida.
    \begin{lstlisting}[frame = single,label={lst:lstlisting7-4.2}]
ch <- x // un'istruzione di send

x = <-ch // un'espressione di receive in un'operazione di assegnamento
<-ch     // un'istruzione di receive; il risultato %*\textit{è}*) scartato
    \end{lstlisting}
    I channel supportano una terza operazione, \textit{close}, che imposta un flag sul channel ad indicare che nessun valore verrà più inviato tramite esso;
    tentare comunque un invio causa un panic.
    Le operazioni di receive su un channel chiuso restituiscono i valori che sono stati inviati prima della chiusura fino a quando non finiscono;
    qualunque successiva operazione di receive viene risolta immediatamente producendo il valore zero del tipo elementare del channel.
    \hfill \vspace{12pt}

    Per chiudere un channel, bisogna chiamare la funzione built-in \verb"close":
    \begin{lstlisting}[frame = single,label={lst:lstlisting7-4.3}]
close(ch)
    \end{lstlisting}
    Un channel creato con una chiamata a \verb"make" viene detto un \textit{unbuffered} channel, ma \verb"make" accetta un secondo argomento opzionale, un intero detto \textit{capacity} del channel.
    Se la capacity del channel è diversa da zero, \verb"make" crea un \textit{buffered} channel.
    \begin{lstlisting}[frame = single,label={lst:lstlisting7-4.4}]
ch = make(chan int)    // unbuffered channel
ch = make(chan int, 0) // unbuffered channel
ch = make(chan int, 3) // buffered channel con capacity 3
    \end{lstlisting}
    \subfile{4-channel-subsection/1-unbuffered-channel}
    \subfile{4-channel-subsection/2-pipeline}
    \subfile{4-channel-subsection/3-tipi-di-channel-unidirezionali}
    \subfile{4-channel-subsection/4-buffered-channel}
\end{document}
\documentclass[../../thesis.tex]{subfiles}
\begin{document}
    \subsection{Esempio: clock server concorrente}\label{subsec:esempio:-clock-server-concorrente}
    La rete è un dominio naturale dove usare la concorrenza perché vengono tipicamente gestite molte connessioni provenienti da diversi client allo stesso tempo, consci che ogni client è in genere indipendente da tutti gli altri.
    \hfill \vspace{12pt}

    Esponiamo come primo esempio un clock server sequenziale che scrive l'orario corrente al client ad ogni secondo:
    \begin{lstlisting}[frame = single, label = {lst:lstlisting7-2.1}]
// Clock1 %*\textit{è}*) un server TCP che periodicamente scrive l'orario.
func main() {
    listener, err := net.Listen("tcp", "localhost:8000")
    if err != nil {
        log.Fatal(err)
    }
    for {
        conn, err := listener.Accept()
        if err != nil {
            log.Print(err) // p.e., connessione fallita
            continue
        }
        handleConn(conn) // gestisce una connessione alla volta
    }
}

func handleConn(c net.Conn) {
    defer c.Close()
    for {
        _, err := io.WriteString(c, time.Now().Format("15:04:05\n"))
        if err != nil {
            return // p.e., client disconnesso 
        }
        time.Sleep(1 * time.Second)
    }
}
    \end{lstlisting}
    La funzione \verb"Listen" crea un \verb"net.Listener", un oggetto che ``ascolta" l'arrivo di connessioni in ingresso sullo porta di rete designata, in questo caso la porta TCP \verb"localhost:8000".
    Il metodo \verb"Accept" del listener blocca il listener stesso in attesa di una richiesta di connessione, quindi restituisce un oggetto \verb"net.Conn" rappresentante la connessione attesa.
    \hfill \vspace{12pt}

    La funzione \verb"handleConn" gestisce una completa connessione del client.
    In un ciclo, essa scrive l'orario corrente, \verb"time.Now()", al client.
    Finché \verb"net.Conn" soddisfa l'interfaccia \verb"io.Writer", si può scrivere direttamente al client.
    Il ciclo finisce quando la scrittura fallisce, che molto probabilmente accade quando il client si disconnette, momento in cui \verb"handleConn" chiude il proprio lato della connessione usando la chiama differita a \verb"Close" e torna ad attendere la richiesta di una connessione. \\
    Sul lato client il programma è il seguente:
    \begin{lstlisting}[frame = single, label = {lst:lstlisting7-2.2}]
// Netcat1 %*\textit{è}*) un client TCP di sola lettura
func main() {
    conn, err := net.Dial("tcp", "localhost:8000")
    if err != nil {
        log.Fatal(err)
    }
    defer conn.Close()
    mustCopy(os.Stdout, conn)
}

func mustCopy(dst io.Writer, src io.Reader) {
    if _, err := io.Copy(dst, src); err != nil {
        log.Fatal(err)
    }
}
    \end{lstlisting}
    Questo programma legge i dati dalla connessione e li scrive sullo standard output fino a quando non viene incontrato una condizione di end-of-file o un errore.
    Eseguendo due client allo stesso tempo si esegue il seguente risultato:
    \begin{lstlisting}[language = bash, frame = L, label = {lst:lstlisting7-2.3}]
$ ./netcat1
13:56:34
13:56:35        $ ./netcat1
13:56:36
^C
                13:56:38
                13:56:39
                13:56:40
                ^C
    \end{lstlisting}
    In questa implementazione il secondo client è obbligato ad aspettare che il primo client finisca perché il server è \textit{sequenziale}; il server si occupa di un client alla volta.
    Per rendere il server concorrente serve solo una piccola modifica: l'aggiunta della parola chiave \verb"go" alla chiamata di \verb"handleConn" fa sì che ogni chiamata venga eseguita nella propria goroutine.
    \begin{lstlisting}[frame = single, label = {lst:lstlisting7-2.4}]
func main() {
    listener, err := net.Listen("tcp", "localhost:8000")
    if err != nil {
        log.Fatal(err)
    }
    for {
        conn, err := listener.Accept()
        if err != nil {
            log.Print(err) // p.e., connessione fallita
            continue
        }
        go handleConn(conn) // gestisce le connessioni  in modo concorrente
    }
}
    \end{lstlisting}
    Ora più client possono ricevere l'orario contemporaneamente:
    \begin{lstlisting}[language = bash, frame = L, label = {lst:lstlisting7-2.5}]
$ ./netcat2
13:58:54
13:58:55        $ ./netcat2
13:58:56        13:58:56
13:58:57        13:58:57
13:58:58        13:58:58
13:58:59        ^C
13:59:00
13:59:01        $ ./netcat2
13:59:02        13:59:02
13:59:03        13:59:03
^C              13:59:04
                ^C
    \end{lstlisting}
\end{document}
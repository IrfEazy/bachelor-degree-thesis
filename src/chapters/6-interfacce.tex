\documentclass[../thesis.tex]{subfiles}
\begin{document}
    \newpage


    \section{Interfacce}\label{sec:interfacce}
    I tipi di interfaccia esprimono generalizzazioni o astrazioni sui comportamenti di altri tipi.
    Generalizzando, le interfacce ci permettono di scrivere funzioni più flessibili e adattabili perché non sono legate ai dettagli di una particolare implementazione.
    \hfill \vspace{12pt}

    Ciò che rende le interfacce di Go così distintive è che sono \textit{soddisfatte implicitamente}.
    In altre parole, non c'è bisogno di dichiarare tutte le interfacce che un dato tipo concreto soddisfa;
    basta semplicemente possedere i metodi necessari.
    Questo design consente di creare nuove interfacce che sono soddisfatte dai tipi concreti esistenti senza modificare i tipi esistenti, il che è particolarmente utile per i tipi definiti in pacchetti che non si controllano.
    \subfile{6-interfacce-subsection/1-interfacce-come-contratti}
    \subfile{6-interfacce-subsection/2-tipi-di-interfaccia}
    \subfile{6-interfacce-subsection/3-soddisfazione-interfaccia}
    \subfile{6-interfacce-subsection/4-parsing-flags-con-flag.value}
    \subfile{6-interfacce-subsection/5-valori-di-interfaccia}
    \subfile{6-interfacce-subsection/6-interfaccia-error}
    \clearpage
\end{document}
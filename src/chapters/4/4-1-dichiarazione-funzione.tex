Una dichiarazione di funzione ha un nome, un elenco di parametri, un elenco opzionale di risultati e un corpo:
\begin{lstlisting}[label={lst:lstlisting4-1.1}]
func nome(elenco-parametri) (elenco-risultati) {
   corpo
}
\end{lstlisting}
L'elenco dei parametri specifica i nomi e i tipi dei \textit{parametri} della funzione, che sono la variabile locale i cui valori o \textit{argomenti} sono forniti dal chiamante.
L'elenco dei risultati specifica i tipi di valori restituiti dalla funzione.
Se la funzione restituisce un risultato senza nome o nessun risultato, le parentesi sono facoltative e di solito omesse.
Omettere la lista dei risultati vuol dire che la funzione non restituisce alcun valore ed è chiamata solo per i suoi effetti.

Una sequenza di parametri o risultati dello stesso tipo può essere scomposta in modo che il tipo stesso sia scritto solo una volta.

In seguito vengono presentati quattro modi per dichiarare una funzione con due parametri e un risultato, tutti di tipo \verb|int|.
L'identificatore vuoto può essere utilizzato per sottolineare che un parametro non è utilizzato.
\begin{lstlisting}[frame=single, label={lst:lstlisting4-1.2}]
func add(x int, y int) { return x + y }
func sub(x, y int) (z int) { z = x - y; return }
func first(x int, _ int) int { return x }
func zero(int, int) int { return 0 }

fmt.Printf(%*``*\)%T\n%*''*\), add)
fmt.Printf(%*``*\)%T\n%*''*\), sub)
fmt.Printf(%*``*\)%T\n%*''*\), first)
fmt.Printf(%*``*\)%T\n%*''*\), zero)
\end{lstlisting}
Output:
\begin{lstlisting}[language=bash, frame=L, label={lst:lstlisting4-1.3}]
func(int, int) int
func(int, int) int
func(int, int) int
func(int, int) int
\end{lstlisting}
Il tipo di una funzione è talvolta detta la sua \textit{firma}.
Due funzioni hanno lo stesso tipo o firma se hanno la stessa sequenza di tipi di parametri e la stessa sequenza di tipi di risultato.
I nomi dei parametri e dei risultati non influenzano il tipo, né il fatto che siano stati dichiarati o meno utilizzando la forma fattorizzata.

Gli argomenti vengono passati \textit{per valore}, quindi la funzione riceve una copia di ogni argomento;
le modifiche alla copia non influenzano il chiamante.
Tuttavia, se l'argomento contiene un qualche tipo di riferimento, come un puntatore, una slice, una map, una funzione o un channel, allora il chiamante può essere influenzato da qualsiasi modifica che la funzione apporta \textit{indirettamente} alle variabili cui fa riferimento l'argomento.

Si può occasionalmente incontrare una dichiarazione di funzione senza un corpo, indicando che la funzione è implementata in un linguaggio diverso da Go.
Tale dichiarazione definisce la firma della funzione.
\begin{lstlisting}[frame=single, label={lst:lstlisting4-1.4}]
package math

func Sin(x float64) float64 // implementato in linguaggio
                            // assembly p.e.
\end{lstlisting}
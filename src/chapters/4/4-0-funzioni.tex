Una funzione ci permette di avvolgere una sequenza di istruzioni come un'unità che può essere chiamata da altrove in un programma, anche più volte.
Le funzioni permettono di suddividere un grande lavoro in parti più piccole.
Una funzione nasconde i dettagli della sua implementazione ai suoi utenti.
Per tutti questi motivi, le funzioni sono una parte critica di tutto il linguaggio di programmazione.


\section{Dichiarazione di funzione}
\label{sec:dichiarazione_di_funzione}%
Una dichiarazione di funzione ha un nome, un elenco di parametri, un elenco opzionale di risultati e un corpo:
\begin{lstlisting}[label={lst:lstlisting4-1.1}]
func nome(elenco-parametri) (elenco-risultati) {
   corpo
}
\end{lstlisting}
L'elenco dei parametri specifica i nomi e i tipi dei \textit{parametri} della funzione, che sono la variabile locale i cui valori o \textit{argomenti} sono forniti dal chiamante.
L'elenco dei risultati specifica i tipi di valori restituiti dalla funzione.
Se la funzione restituisce un risultato senza nome o nessun risultato, le parentesi sono facoltative e di solito omesse.
Omettere la lista dei risultati vuol dire che la funzione non restituisce alcun valore ed è chiamata solo per i suoi effetti.

Una sequenza di parametri o risultati dello stesso tipo può essere scomposta in modo che il tipo stesso sia scritto solo una volta.

In seguito vengono presentati quattro modi per dichiarare una funzione con due parametri e un risultato, tutti di tipo \verb|int|.
L'identificatore vuoto può essere utilizzato per sottolineare che un parametro non è utilizzato.
\begin{lstlisting}[frame=single, label={lst:lstlisting4-1.2}]
func add(x int, y int) { return x + y }
func sub(x, y int) (z int) { z = x - y; return }
func first(x int, _ int) int { return x }
func zero(int, int) int { return 0 }

fmt.Printf(%*``*\)%T\n%*''*\), add)
fmt.Printf(%*``*\)%T\n%*''*\), sub)
fmt.Printf(%*``*\)%T\n%*''*\), first)
fmt.Printf(%*``*\)%T\n%*''*\), zero)
\end{lstlisting}
Output:
\begin{lstlisting}[language=bash, frame=L, label={lst:lstlisting4-1.3}]
func(int, int) int
func(int, int) int
func(int, int) int
func(int, int) int
\end{lstlisting}
Il tipo di una funzione è talvolta detta la sua \textit{firma}.
Due funzioni hanno lo stesso tipo o firma se hanno la stessa sequenza di tipi di parametri e la stessa sequenza di tipi di risultato.
I nomi dei parametri e dei risultati non influenzano il tipo, né il fatto che siano stati dichiarati o meno utilizzando la forma fattorizzata.

Gli argomenti vengono passati \textit{per valore}, quindi la funzione riceve una copia di ogni argomento;
le modifiche alla copia non influenzano il chiamante.
Tuttavia, se l'argomento contiene un qualche tipo di riferimento, come un puntatore, una slice, una map, una funzione o un channel, allora il chiamante può essere influenzato da qualsiasi modifica che la funzione apporta \textit{indirettamente} alle variabili cui fa riferimento l'argomento.

Si può occasionalmente incontrare una dichiarazione di funzione senza un corpo, indicando che la funzione è implementata in un linguaggio diverso da Go.
Tale dichiarazione definisce la firma della funzione.
\begin{lstlisting}[frame=single, label={lst:lstlisting4-1.4}]
package math

func Sin(x float64) float64 // implementato in linguaggio
                            // assembly p.e.
\end{lstlisting}


\section{Valori di ritorno multipli}
\label{sec:valori_di_ritorno_multipli}%
\input{chapters/4/4-2-valori-ritorno-multipli}


\section{Errori}
\label{sec:errori}%
L'approccio di Go lo distingue da molti altri linguaggi in cui i fallimenti sono segnalati usando \textit{eccezioni}, non valori ordinari.
Anche se Go ha un meccanismo di eccezione di sorta, è usato solo per segnalare errori veramente inaspettati che indicano un bug, non gli errori di routine che un programma robusto dovrebbe aspettarsi.

Una funzione per la quale il fallimento è un comportamento atteso restituisce un risultato aggiuntivo, convenzionalmente l'ultimo.
Se il fallimento ha una sola possibile causa, il risultato è un booleano, di solito chiamato \verb|ok|.
Più spesso il fallimento può avere una varietà di cause per le quali il chiamante avrà bisogno di una spiegazione.
In tali casi, il tipo di risultato aggiuntivo è \verb|error| (un tipo di interfaccia).

La ragione di questo progetto è che le eccezioni tendono ad impigliare la descrizione di un errore con il flusso di controllo richiesto per gestirlo, spesso portando ad un risultato indesiderato: gli errori di routine vengono segnalati all'utente finale sotto forma di una traccia di stack incomprensibile, pieno di informazioni sulla struttura del programma, ma privo di contesto intelligibile su ciò che è andato storto.

Al contrario, i programmi Go usano meccanismi di controllo di flusso ordinari come \verb|if| e \verb|return| per rispondere agli errori.
Questo stile richiede innegabilmente una maggiore attenzione alla logica di gestione degli errori.


\section{Valori di funzione}
\label{sec:valori_di_funzione}%
Le funzioni sono \textit{valori di prima classe} in Go: come altri valori, i valori di funzione hanno tipi, e possono essere assegnati a variabili o passati a o restituiti da funzioni.
Un valore di funzione può essere chiamato come qualsiasi altra funzione.
\begin{lstlisting}[frame=single, label={lst:lstlisting4-5.1}]
func square(n int) int { return n * n }
func negative(n int) int { return -n }
func product(m, n int) int { return m * n }

f := square
fmt.Println(f(3))

f = negative
fmt.Println(f(3))
fmt.Printf(%*``*\)%T\n%*''*\), f)

f = product // compile error: non pu%*\textit{ò}*\) assegnare
            // func(int, int) int a func(int) int
\end{lstlisting}
Output:
\begin{lstlisting}[language=bash, frame=L, label={lst:lstlisting4-5.2}]
9
-3
func(int) int
\end{lstlisting}
Il valore zero di un tipo di funzione è \verb|nil|.
Chiamare un valore di funzione \verb|nil| provoca il panic:
\begin{lstlisting}[frame=single, label={lst:lstlisting4-5.3}]
var f func(int) int
f(3) // panic: chiamata ad una funzione nil
\end{lstlisting}
I valori di funzione possono essere confrontati con ``nil'':
\begin{lstlisting}[frame=single, label={lst:lstlisting4-5.4}]
var f func(int) int
if f != nil {
    f(3)
}
\end{lstlisting}
ma non sono paragonabili, quindi non sono confrontati tra loro o utilizzati come chiavi in una mappa.

I valori delle funzioni ci permettono di parametrizzare le nostre funzioni non solo sui dati, ma anche sul comportamento.
Le librerie standard contengono molti esempi.
Per esempio, \verb|strings.Map| applica una funzione ad ogni carattere di una stringa, unendo i risultati per creare un'altra stringa.
\begin{lstlisting}[frame=single, label={lst:lstlisting4-5.5}]
func add1(r rune) rune { return r + 1 }

fmt.Println(strings.Map(add1, %*``*\)HAL-9000%*''*\)))
fmt.Println(strings.Map(add1, %*``*\)VMS%*''*\)))
fmt.Println(strings.Map(add1, %*``*\)Admix%*''*\)))
\end{lstlisting}
Output:
\begin{lstlisting}[language=bash, frame=L, label={lst:lstlisting4-5.6}]
IBM.:111
WNT
Benjy
\end{lstlisting}


\section{Funzioni anonime}
\label{sec:funzioni_anonime}%
Le funzioni con nome possono essere dichiarate solo a livello di pacchetto, ma possiamo usare una \textit{function literal} per denotare un valore di funzione all'interno di qualsiasi espressione.
Una funzione literal è scritta come una dichiarazione di funzione, ma senza un nome che segue la parola chiave \verb|func|.
È un'espressione, e il suo valore è detto \textit{funzione anonima}.

I function literal permettono di definire una funzione nel loro punto d'uso:
\begin{lstlisting}[frame=single, label={lst:lstlisting4-6.1}]
strings.Map(func(r rune) rune { return r + 1 }, %*``*\)HAL-9000%*''*\))
\end{lstlisting}
Ancora più importante, le funzioni definite in questo modo hanno accesso all'intero ambiente lessicale, quindi la funzione interna può riferirsi alle variabili dalla funzione che la racchiude:
\begin{lstlisting}[frame=single, label={lst:lstlisting4-6.2}]
func squares() func() int {
    var x int
    return func() int {
        x++
        return x * x
    }
}
\end{lstlisting}
\begin{lstlisting}[frame=single, label={lst:lstlisting4-6.3}]
func main() {
    f := squares()
    fmt.Println(f())
    fmt.Println(f())
    fmt.Println(f())
    fmt.Println(f())
}
\end{lstlisting}
Output:
\begin{lstlisting}[language=bash, frame=L, label={lst:lstlisting4-6.4}]
1
4
9
16
\end{lstlisting}
La funzione \verb|squares| restituisce un'altra funzione, di tipo \verb|func() int|.
La prima chiamata a \verb|squares| crea una variabile locale \verb|x| e restituisce una funzione anonima che, ogni volta che viene chiamata, incrementa \verb|x| e ne restituisce il suo quadrato.
Una seconda chiamata a \verb|squares| crea una seconda variabile \verb|x| e restituisce una nuova funzione anonima che incrementa quella variabile.

L'esempio \verb|squares| dimostra che i valori di funzione non sono solo codice ma possono avere stato.
La funzione interna anonima può accedere e aggiornare le variabili locali della funzione \verb|squares| che la racchiude.
Questi riferimenti di variabili nascoste sono il motivo per cui le funzioni si classificano come tipi di riferimento e il perché i valori di funzione non sono comparabili.
I valori di funzione come questi sono effettuati facendo uso di una tecnica detta \textit{closure} ed i programmatori di Go usano spesso questo termine per i valori di funzione.

Quando una funzione anonima richiede la ricorsione, bisogna prima dichiarare una variabile e quindi assegnare la funzione anonima a quella variabile.
Se i due passi sono combinati nella dichiarazione (con un'istruzione \verb|:=|), la function literal non rientrerebbe nell'ambito della variabile, quindi non avrebbe modo di chiamarsi ricorsivamente.


\section{Funzioni variadic}
\label{sec:funzioni_variadic}%
\input{chapters/4/4-6-funzioni-variadic}


\section{Chiamate con funzione differita}
\label{sec:chiamate_con_funzione_differita}%
\input{chapters/4/4-7-chiamate-funzione-differita}


\section{Panic}
\label{sec:panic}%
Il sistema di tipo di Go cattura gli errori al momento della compilazione, ma altri, come l'accesso all'array out-of-bounds o il puntatore nil, la dereferenza, richiedono controlli al momento dell'esecuzione.
Quando la routine Go rileva questi errori, lancia il \verb|panic|.

Durante un tipico panic, l'esecuzione normale si ferma, tutte le chiamate di funzione differite vengono eseguite e il programma si blocca con un messaggio di log.
Questo messaggio di log include il valore di panic, che di solito è un messaggio di errore di qualche tipo, e, per ogni goroutine, una traccia di stack che mostra la pila di chiamate di funzione che erano attive al momento del panico.
Questo messaggio di log ha spesso abbastanza informazioni per diagnosticare la causa principale del problema senza eseguire di nuovo il programma, quindi dovrebbe sempre essere incluso in una segnalazione di bug su un programma in preda al panic.


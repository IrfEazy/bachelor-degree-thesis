Le funzioni con nome possono essere dichiarate solo a livello di pacchetto, ma possiamo usare una \textit{function literal} per denotare un valore di funzione all'interno di qualsiasi espressione.
Una funzione literal è scritta come una dichiarazione di funzione, ma senza un nome che segue la parola chiave \verb|func|.
È un'espressione, e il suo valore è detto \textit{funzione anonima}.

I function literal permettono di definire una funzione nel loro punto d'uso:
\begin{lstlisting}[frame=single, label={lst:lstlisting4-6.1}]
strings.Map(func(r rune) rune { return r + 1 }, %*``*\)HAL-9000%*''*\))
\end{lstlisting}
Ancora più importante, le funzioni definite in questo modo hanno accesso all'intero ambiente lessicale, quindi la funzione interna può riferirsi alle variabili dalla funzione che la racchiude:
\begin{lstlisting}[frame=single, label={lst:lstlisting4-6.2}]
func squares() func() int {
    var x int
    return func() int {
        x++
        return x * x
    }
}
\end{lstlisting}
\begin{lstlisting}[frame=single, label={lst:lstlisting4-6.3}]
func main() {
    f := squares()
    fmt.Println(f())
    fmt.Println(f())
    fmt.Println(f())
    fmt.Println(f())
}
\end{lstlisting}
Output:
\begin{lstlisting}[language=bash, frame=L, label={lst:lstlisting4-6.4}]
1
4
9
16
\end{lstlisting}
La funzione \verb|squares| restituisce un'altra funzione, di tipo \verb|func() int|.
La prima chiamata a \verb|squares| crea una variabile locale \verb|x| e restituisce una funzione anonima che, ogni volta che viene chiamata, incrementa \verb|x| e ne restituisce il suo quadrato.
Una seconda chiamata a \verb|squares| crea una seconda variabile \verb|x| e restituisce una nuova funzione anonima che incrementa quella variabile.

L'esempio \verb|squares| dimostra che i valori di funzione non sono solo codice ma possono avere stato.
La funzione interna anonima può accedere e aggiornare le variabili locali della funzione \verb|squares| che la racchiude.
Questi riferimenti di variabili nascoste sono il motivo per cui le funzioni si classificano come tipi di riferimento e il perché i valori di funzione non sono comparabili.
I valori di funzione come questi sono effettuati facendo uso di una tecnica detta \textit{closure} ed i programmatori di Go usano spesso questo termine per i valori di funzione.

Quando una funzione anonima richiede la ricorsione, bisogna prima dichiarare una variabile e quindi assegnare la funzione anonima a quella variabile.
Se i due passi sono combinati nella dichiarazione (con un'istruzione \verb|:=|), la function literal non rientrerebbe nell'ambito della variabile, quindi non avrebbe modo di chiamarsi ricorsivamente.
\documentclass[../thesis.tex]{subfiles}
\begin{document}
    \newpage


    \section{Tipi di dati di base}\label{sec:tipi-di-dati-di-base}
    I tipi di Go rientrano in quattro categorie: \textit{tipi base}, \textit{tipi aggregati}, \textit{tipi di riferimento} e \textit{tipi di interfaccia}.
    I tipi di base includono numeri, stringhe e booleani.
    I tipi aggregati formano tipi di dati più complicati combinando valori di diversi tipi semplici.
    I tipi di riferimento sono un gruppo diverso che include puntatori, slice, map, funzioni e channel, ma ciò che hanno in comune è che tutti si riferiscono a variabili di programma o di stato \textit{indirettamente}, in modo che l'effetto di un'operazione applicata a un riferimento sia osservato da tutte le copie di tale riferimento.
    \subfile{2-tipi-dati-base-subsection/1-numeri-interi}
    \subfile{2-tipi-dati-base-subsection/2-stringhe}
    \subfile{2-tipi-dati-base-subsection/3-costanti}
    \clearpage
\end{document}
\begin{quotation}
    \begin{small}
        \textit{Go is an open-source programming language that makes it easy to build simple, reliable, and efficient software} (Dal sito web \verb|golang.org| di Go)
    \end{small}
\end{quotation}
Go è un progetto ideato nel 2007 da Robert Griesemer, Rob Pike e Ken Thompson di Google, ed è stato annunciato nel novembre 2009.
Go è open-source, quindi il codice sorgente del suo compilatore, le sue librerie e i suoi strumenti sono disponibili gratuitamente a chiunque.
I suoi programmi possono essere eseguiti su sistemi discendenti da Unix e Microsoft Windows e, in generale, i programmi scritti in uno degli ambienti di sviluppo funzionano anche su altri ambienti senza modifiche precedenti.

Il linguaggio Go e i suoi strumenti associati hanno l'obiettivo di essere espressivo, efficiente ed efficace.
Per attirare i programmatori sul mercato, Go è stato implementato in un modo che assomiglia al linguaggio di programmazione C\@.
Come linguaggio di programmazione, si adatta e prende in prestito buone idee da molti altri linguaggi, evitando i dettagli che porterebbero a un codice complesso e inaffidabile.
Rispetto ad altri linguaggi di programmazione, la libreria di gestione della concorrenza dei processi è nuova ed efficiente e il suo approccio all'astrazione dei dati e alla programmazione orientata agli oggetti è insolitamente flessibile.
Go è inoltre dotato di un \textit{garbage collector} per la gestione automatica della memoria da parte dei programmi, nonostante sia un linguaggio compilato.

Go è particolarmente adatto per la creazione di sistemi informatici come server di rete e strumenti e sistemi di servizio per programmatori, ma è un linguaggio molto versatile anche per altri scopi, infatti viene utilizzato nei settori della grafica, delle applicazioni mobili e del machine learning.
\section*{L'origine di Go}
Go è talvolta descritto come un ``linguaggio C-like'', o come ``C del XXI secolo''.
Da C, Go ha ereditato sintassi, flussi di controllo, tipi di dati di base, chiamate di valore nel passaggio dei parametri, puntatori e, soprattutto, enfasi C su programmi che compilano con codice macchina efficiente e cooperano naturalmente con le astrazioni del sistema operativo corrente.

Tuttavia, ci sono altri antenati nell'albero genealogico Go.
Un grande flusso di influenza proviene dalle lingue Wirth, a cominciare da Pascal.
Modula-2 ha ispirato il concetto di pacchetto.
Oberon ha eliminato la distinzione tra i file di interfaccia del modulo e i file di implementazione del modulo.
Oberon-2 ha influenzato la sintassi per gli imballaggi, le importazioni e le dichiarazioni, in particolare le dichiarazioni di metodo.

Un altro lignaggio tra gli antenati di Go, e uno che distingue Go tra i recenti linguaggi di programmazione, è una sequenza di piccoli linguaggi di ricerca sviluppati presso i Bell Labs, tutti ispirati dal concetto di \textit{Communicating Sequential Processes} (CSP) o processi di comunicazione sequenziale in italiano.
In CSP, i programmi sono una composizione parallela di processi senza condivisione dello stato;
i processi comunicano e sincronizzano utilizzando i canali.
Tuttavia CSP era solo un linguaggio formale, e non un linguaggio di programmazione.

La prima implementazione del CSP in un linguaggio di programmazione si chiamava Squeak, che offriva un linguaggio per la gestione di eventi mouse e tastiera creando canali in modo statico.
Questo è stato seguito da Newsqueak, che ha offerto dichiarazioni C-like ed espressioni sintattiche e tipo di notazione Pascal-like.
È stato puramente un linguaggio funzionale con garbage collector ancora una volta specializzata in tastiera, mouse, e la gestione delle finestre degli eventi.
I canali sono stati creati dinamicamente e memorizzati in variabili.
Ha seguito il linguaggio di programmazione Alef in cui è stato rimosso il garbage collector, rendendo la competizione dei linguaggi di programmazione molto più feroce.

Go offre la possibilità di definire array dinamici con un accesso casuale efficiente, ma consente anche protocolli di condivisione sofisticati che ricordano le liste concatenate.

\subsection*{Progetto Go}
Tutti i linguaggi di programmazione riflettono la filosofia di programmazione dei loro creatori.
Il progetto Go è nato dalla frustrazione di Google di avere numerosi sistemi software che hanno sofferto di un'esplosione di complessità.

Il progetto Go comprendeva il proprio linguaggio, i propri strumenti standard e le biblioteche, e un'agenda culturale di radicale semplicità.
Go ha un garbage collector, un sistema di pacchetti, funzioni di prim'ordine, ricchezza lessicale, un'interfaccia di chiamata di sistema e stringhe immutabili in cui il testo è generalmente codificato in UTF-8, ma ha relativamente poche caratteristiche.
Ad esempio, non offre la conversione numerica implicita, costruttori o decostruttori, possibilità di sovraccarico dell'operatore, valori parametrici predefiniti, ereditarietà, generic, eccezioni, macro, annotazioni di funzione e Thread-Local Storage (TLS).
Il linguaggio garantisce la retrocompatibilità: i vecchi programmi Go possono essere compilati ed eseguiti da nuove versioni di compilatori e librerie standard.

Go incoraggia la conoscenza del design contemporaneo del sistema informatico, in particolare l'importanza della località.
I suoi tipi di dati e la maggior parte delle librerie di strutture dati sono costruiti per funzionare naturalmente senza inizializzazione esplicita o un costruttore implicito, quindi l'allocazione di poca memoria e la scrittura della memoria stessa è nascosta nel codice.
I tipi aggregati di Go (strutture e dati) mantengono i loro elementi direttamente, richiedendo meno memoria, meno allocazione e reindirizzamento rispetto ai linguaggi che utilizzano campi indiretti.
E poiché i computer moderni sono macchine parallele, Go ha funzionalità di competizione basate su CSP\@.
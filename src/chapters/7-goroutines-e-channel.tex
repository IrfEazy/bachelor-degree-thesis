\documentclass[../thesis.tex]{subfiles}
\begin{document}
    \newpage


    \section{Goroutine e channel}\label{sec:goroutine-e-channel}
    La programmazione concorrente, programma visto come composizione di numerose attività autonome, è l'elemento più importante di questi tempi.
    I server web gestiscono richieste per migliaia di client contemporaneamente.
    Le app degli smartphone e dei tablet gestiscono le animazioni sull'interfaccia utente mentre simultaneamente eseguono la computazione e le richieste di rete in background.
    Anche i tradizionali problemi di batch (lettura dei dati, computazione e scrittura degli output) usano la concorrenza per nascondere la latenza delle operazioni I/O così da sfruttare anche i multiprocessori dei computer moderni, i quali, negli anni, continuano a crescere in numero ma non in velocità.
    \hfill \vspace{12pt}

    Go offre due stili di programmazione concorrente.
    In questo capitolo vengono esponte le goroutine e i channel, compatibili con la \textit{CSP} o \textit{communicating sequential processes}, un modello di concorrenza dove i valori sono visti come attività indipendenti (goroutine), ma con le variabili confinate alle singole attività.
    Nel prossimo capitolo verranno illustrati gli aspetti più tradizionali del modello \textit{shared memory multithreading}.
    \subfile{7-goroutines-e-channel-subsection/1-goroutine}
    \subfile{7-goroutines-e-channel-subsection/2-esempio-server-di-clock-concorrente}
    \subfile{7-goroutines-e-channel-subsection/4-channel}
    \subfile{7-goroutines-e-channel-subsection/10-esempio-chat-server}
    \clearpage
\end{document}
\documentclass[../../thesis.tex]{subfiles}
\begin{document}
    \subsection{Valori ed espressioni del metodo}\label{subsec:valori-ed-espressioni-del-metodo}
    Di solito si seleziona e si chiama un metodo nella stessa espressione, come in \verb"p.Distance()", ma è possibile separare queste due operazioni.
    Il selettore \verb"p.Distance" produce un \textit{valore del metodo}, una funzione che lega un metodo (\verb"Point.Distance") a un valore ricevitore specifico \verb"p".
    Questa funzione può essere invocata senza un valore di ricevitore;
    ha bisogno solo degli argomenti non riceventi.
    \begin{lstlisting}[frame = single,label={lst:lstlisting5-4.1}]
p := Point{1, 2}
q := Point{4, 6}

distanceFromP := p.Distance // valore di metodo
fmt.Println(distanceFromP(q))
var origin Point // {0, 0}
fmt.Println(distanceFromP(origin))

scaleP := p.ScaleBy // valore di metodo
scaleP(2)	    // p diventa (2, 4)
scaleP(3)	    // ora (6, 12)
scaleP(10)	    // ora (60, 120)
    \end{lstlisting}
    Output:
    \begin{lstlisting}[language = bash, frame = L,label={lst:lstlisting5-4.2}]
5
2.23606797749979
    \end{lstlisting}
    Correlato al valore del metodo è l'\textit{espressione del metodo}.
    Quando si chiama un metodo, al contrario di una funzione ordinaria, bisogna fornire il ricevitore in modo speciale utilizzando la sintassi del selettore.
    Un'espressione di metodo, scritta \verb"T.f" o \verb"(*T).f" dove \verb"T" è un tipo, produce un valore di funzione con un primo parametro regolare che prende il posto del ricevitore, quindi può essere chiamato nel modo usuale.
    \begin{lstlisting}[frame = single,label={lst:lstlisting5-4.3}]
p := Point{1, 2}
q := Point{4, 6}

distance := Point.Distance
fmt.Println(distance(p, q))
fmt.Printf("%T\n", distance)

scale := (*Point).ScaleBy
scale(&p, 2)
fmt.Println(p)
fmt.Printf("%T\n", scale)
    \end{lstlisting}
    \newpage\noindent Output:
    \begin{lstlisting}[language = bash, frame = L,label={lst:lstlisting5-4.4}]
5
func(Point, Point) floa64
{2 4}
func(*Point, float64)
    \end{lstlisting}
    Le espressioni del metodo possono essere utili quando è necessario un valore per rappresentare una scelta tra diversi metodi appartenenti allo stesso tipo in modo da poter chiamare il metodo scelto con diversi ricevitori.
    Nell'esempio seguente la variabile \verb"op" rappresenta l'addizione o il metodo di sottrazione del tipo \verb"Point", e \verb"Path.TranslateBy" la chiama per ogni punto del \verb"Path":
    \begin{lstlisting}[frame = single,label={lst:lstlisting5-4.5}]
type Point struct { X, Y float64 }

func (p Point) Add(q Point) Point { return Point{p.X + q.X, p.Y + q.Y} }
func (p Point) Sub(q Point) Point { return Point{p.X - q.X, p.Y - q.Y} }

type Path []Point

func (path Path) TranslateBy(offset Point, add bool) {
    var op func(p, q Point) Point
    if add {
        op = Point.Add
    } else {
        op = Point.Sub
    }
    for i := range path {
        // Chiama o path[i].Add(offset) o path[i].Sub(offset)
        path[i] = op(path[i], offset)
    }
}
    \end{lstlisting}
\end{document}
\documentclass[../../thesis.tex]{subfiles}
\begin{document}
    \subsection{Tipi di composizione per Struct Embedding}\label{subsec:tipi-di-composizione-per-struct-embedding}
    Si consideri il tipo \verb"ColoredPoint":
    \begin{lstlisting}[frame = single, label = {lst:lstlisting5-3.1}]
import "image/color"

type Point struct { X, Y float64 }

type ColoredPoint struct {
    Point
    Color color.RGBA
}
    \end{lstlisting}
    \verb"Point" è stato \textit{inserito} in \verb"ColoredPoint" per fornire i campi \verb"X" e \verb"Y".
    Se si vuole, si possono selezionare i campi di \verb"ColoredPoint" che sono stati forniti dal \verb"Point" incorporato senza menzionare il \verb"Point".
    Un meccanismo simile si applica ai \textit{metodi} del \verb"Point".
    Si possono chiamare i metodi del campo \verb"Point" incorporato usando un ricevitore di tipo \verb"ColoredPoint", anche se \verb"ColoredPoint" non ha metodi dichiarati:
    \begin{lstlisting}[frame = single, label = {lst:lstlisting5-3.2}]
red := color.RGBA{255, 0, 0, 255}
blue := color.RGBA{0, 0, 255, 255}
var p = ColoredPoint{Point{1, 1}, red}
var q = ColoredPoint{Point{5, 4}, blue}
fmt.Println(p.Distance(q.Point))
p.ScaleBy(2)
q.ScaleBy(2)
fmt.Println(p.Distance(q.Point))
    \end{lstlisting}
    Output:
    \begin{lstlisting}[language = bash, frame = L, label = {lst:lstlisting5-3.3}]
5
10
    \end{lstlisting}
    I metodi di \verb"Point" sono stati \textit{promossi} a \verb"ColoredPoint".
    In questo modo, l'embedding permette di costruire tipi complessi con molti metodi mediante la \textit{composizione} di diversi campi, ognuno dei quali fornisce alcuni metodi.
    \hfill \vspace{12pt}

    Si osservino le chiamate a \verb"Distance" sopra. \verb"Distance" ha un parametro di tipo \verb"Point", e \verb"q" non è un \verb"Point", quindi anche se \verb"q" ha un campo incorporato di quel tipo, dobbiamo selezionarlo esplicitamente.
    Tentare di passare \verb"q" sarebbe un errore:
    \begin{lstlisting}[frame = single, label = {lst:lstlisting5-3.4}]
p.Distance(q) // compile error: impossibile usare q (ColoredPoint) come
              // Point
    \end{lstlisting}
    Il tipo di campo anonimo può essere un \textit{puntatore} a un tipo denominato, nel qual caso i campi e i metodi sono promossi indirettamente dall'oggetto puntato.
    L'aggiunta di un altro livello di indirezione ci permette di condividere strutture comuni e variare dinamicamente le relazioni tra gli oggetti.
    La dichiarazione di \verb"ColoredPoint" qui sotto incorpora un \verb"*Point":
    \begin{lstlisting}[frame = single, label = {lst:lstlisting5-3.5}]
type ColoredPoint struct {
    *Point
    Color color.RGBA
}

p := ColoredPoint{&Point{1, 1}, red}
q := ColoredPoint{&Point{5, 4}, blue}
fmt.Println(p.Distance(*q.Point))
q.Point = p.Point // p e q ora condividono lo stesso punto
p.ScaleBy(2)
fmt.Println(*p.Pointm *q.Point)
    \end{lstlisting}
    Output:
    \begin{lstlisting}[language = bash, frame = L, label = {lst:lstlisting5-3.6}]
5
{2, 2} {2, 2}
    \end{lstlisting}
    Un tipo di struct può avere più di un campo anonimo.
    Se avessimo dichiarato ColoredPoint come
    \begin{lstlisting}[frame = single, label = {lst:lstlisting5-3.7}]
type ColoredPoint struct {
    Point
    color.RGBA
}
    \end{lstlisting}
    un valore di questo tipo avrebbe tutti i metodi di \verb"Point", tutti i metodi di \verb"RGBA", e tutti i metodi aggiuntivi dichiarati su \verb"ColoredPoint" direttamente.
    Quando il compilatore risolve un selettore come \verb"p.ScaleBy" in un metodo, cerca prima un metodo dichiarato direttamente chiamato \verb"ScaleBy", poi metodi promossi dai campi embedded di \verb"ColoredPoint", quindi metodi promossi dai campi embedded all'interno di \verb"Point" e \verb"RGBA", e così via.
    Il compilatore segnala un errore se il selettore è ambiguo perché due metodi sono stati promossi dallo stesso rango.
    \hfill \vspace{12pt}

    I metodi possono essere dichiarati solo su tipi nominati (come \verb"Point") e puntatori ad essi (\verb"*Point"), ma grazie all'embedding, è possibile e talvolta utile che anche i tipi di struct \textit{senza nome} abbiano metodi.
    \hfill \vspace{12pt}

    Questo esempio mostra parte di una semplice cache implementata usando due variabili a livello di pacchetto, una mutex e la mappa che custodisce:
    \begin{lstlisting}[frame = single, label = {lst:lstlisting5-3.8}]
var (
    mu sync.Mutex // custodisce mapping
    mapping = make(map[string]string)
)
    \end{lstlisting}
    \clearpage
    \newpage
    \begin{lstlisting}[frame = single, label = {lst:lstlisting5-3.9}]
func Lookup(key string) string {
    mu.Lock()
    v := mapping[key]
    mu.Unlock()
    return v
}
    \end{lstlisting}
    La versione qui sotto è funzionalmente equivalente ma raggruppa insieme le due variabili correlate in una singola variabile a livello di pacchetto, \verb"cache":
    \begin{lstlisting}[frame = single, label = {lst:lstlisting5-3.10}]
var cache = struct {
    sync.Mutex
    mapping map[string]string
} {
    mapping: make(map[string]string)
}

func Lookup(key string) string {
    cache.Lock()
    v := cache.mapping[key]
    cache.Unlock()
    return v
}
    \end{lstlisting}
    La nuova variabile dà nomi più espressivi alle variabili relative alla cache, e dato che \verb"sync.Mutex" è incorporato al suo interno, i suoi metodi \verb"Lock" e \verb"Unlock" sono promossi al tipo di struttura senza nome, permettendoci di bloccare la cache con una sintassi auto-esplicativa.
\end{document}
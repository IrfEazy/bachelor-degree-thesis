\documentclass[../../thesis.tex]{subfiles}
\begin{document}
    \subsection{Incapsulamento}\label{subsec:incapsulamento}
    Una variabile o un metodo di un oggetto viene \textit{incapsulato} se è inaccessibile ai client dell'oggetto.
    L'incapsulamento, a volte chiamato \textit{occultamento delle informazioni}, è un aspetto chiave della program- mazione orientata agli oggetti.
    \hfill \vspace{12pt}

    Go ha un solo meccanismo per controllare la visibilità dei nomi: gli identificatori con il primo carattere maiuscolo vengono esportati dal package in cui sono definiti, e i nomi con il primo carattere non minuscolo non lo sono.
    Lo stesso meccanismo che limita l'accesso ai membri di un package limita anche l'accesso ai campi di una struttura o ai metodi di un tipo.
    Di conseguenza, per incapsulare un oggetto, bisogna farne una struct.
    \hfill \vspace{12pt}

    La conseguenza di questo meccanismo basato sul nome è che l'unità di incapsulamento è il package, non il tipo come in molti altri linguaggi.
    I campi di un tipo di struttura sono visibili a tutto il codice all'interno dello stesso package.
    Se il codice appare in una funzione o in un metodo non fa alcuna differenza.
    \hfill \vspace{12pt}

    L'incapsulamento offre tre vantaggi.
    Innanzitutto, poiché i client non possono modificare direttamente le variabili dell'oggetto, è necessario ispezionare un minor numero di istruzioni per comprendere i possibili valori di tali variabili.
    \hfill \vspace{12pt}

    In secondo luogo, nascondere i dettagli di implementazione impedisce ai client di dipendere dalle cose che potrebbero cambiare, il che dà al progettista una maggiore libertà di far evolvere l'implementazione senza rompere la compatibilità API\@.
    \hfill \vspace{12pt}

    Il terzo vantaggio dell'incapsulamento, e in molti casi il più importante, è che impedisce ai client di impostare arbitrariamente le variabili di un oggetto.
    Poiché le variabili dell'oggetto possono essere impostate solo da funzioni nello stesso pacchetto, l'autore di quel pacchetto può assicurare che tutte queste funzioni mantengano gli invarianti interni dell'oggetto.
\end{document}
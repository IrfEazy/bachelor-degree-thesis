\documentclass[../../thesis.tex]{subfiles}
\begin{document}
    \subsection*{Progetto Go}
    Tutti i linguaggi di programmazione riflettono la filosofia di programmazione dei loro creatori.
    Il progetto Go è nato dalla frustrazione di Google di avere numerosi sistemi software che hanno sofferto di un'esplosione di complessità.
    \hfill \vspace{12pt}

    Il progetto Go comprendeva il proprio linguaggio, i propri strumenti standard e le biblioteche, e un'agenda culturale di radicale semplicità.
    Go ha un garbage collector, un sistema di pacchetti, funzioni di prim'ordine, ricchezza lessicale, un'interfaccia di chiamata di sistema e stringhe immutabili in cui il testo è generalmente codificato in UTF-8, ma ha relativamente poche caratteristiche.
    Ad esempio, non offre la conversione numerica implicita, costruttori o decostruttori, possibilità di sovraccarico dell'operatore, valori parametrici predefiniti, ereditarietà, generic, eccezioni, macro, annotazioni di funzione e Thread-Local Storage (TLS).
    Il linguaggio garantisce la retrocompatibilità: i vecchi programmi Go possono essere compilati ed eseguiti da nuove versioni di compilatori e librerie standard.
    \hfill \vspace{12pt}

    Go incoraggia la conoscenza del design contemporaneo del sistema informatico, in particolare l'importanza della località.
    I suoi tipi di dati e la maggior parte delle librerie di strutture dati sono costruiti per funzionare naturalmente senza inizializzazione esplicita o un costruttore implicito, quindi l'allocazione di poca memoria e la scrittura della memoria stessa è nascosta nel codice.
    I tipi aggregati di Go (strutture e dati) mantengono i loro elementi direttamente, richiedendo meno memoria, meno allocazione e reindirizzamento rispetto ai linguaggi che utilizzano campi indiretti.
    E poiché i computer moderni sono macchine parallele, Go ha funzionalità di competizione basate su CSP\@.
\end{document}
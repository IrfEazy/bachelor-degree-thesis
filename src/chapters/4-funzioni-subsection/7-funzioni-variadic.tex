\documentclass[../../thesis.tex]{subfiles}
\begin{document}
    \subsection{Funzioni variadic}\label{subsec:funzioni-variadic}
    Una \textit{funzione variadic} può essere chiamata con un numero variabile di argomenti.
    Gli esempi più familiari sono \verb"fmt.Printf" e le sue varianti. \verb"Printf" richiede un argomento fisso all'inizio, quindi accetta qualsiasi numero di argomenti successivi.
    \hfill \vspace{12pt}

    Per dichiarare una funzione variadic, il tipo del parametro finale è preceduto da un ellissi, ``\verb"..."", che indica che la funzione può essere chiamata con qualsiasi numero di argomenti di questo tipo.
    \begin{lstlisting}[frame = single, label = {lst:lstlisting4-7.1}]
func sum(vals ...int) int {
   total := 0
   for _, val := range vals {
      total += val
   }
   return total
}
    \end{lstlisting}
    La funzione \verb"sum" restituisce la somma di zero o più argomenti \verb"int".
    All'interno del corpo della funzione, il tipo di \verb"vals" è una slice \verb"[]int".
    Quando si invoca \verb"sum", qualunque numero di valori può essere fornito come parametro \verb"vals".
    \begin{lstlisting}[frame = single, label = {lst:lstlisting4-7.2}]
fmt.Println(sum())
fmt.Println(sum(3))
fmt.Println(sum(1, 2, 3, 4))
    \end{lstlisting}
    Output:
    \begin{lstlisting}[language = bash, frame = L, label = {lst:lstlisting4-7.3}]
0
3
10
    \end{lstlisting}
    Implicitamente, il chiamante alloca un array, copia gli argomenti in esso, e passa una slice dell'intero array alla funzione.
    L'ultima chiamata dell'ultimo esempio si comporta come la chiamata del seguente esempio, in cui si mostra come invocare una funzione variadic quando gli argomenti sono già in una slice: si deve posizionare un ellissi dopo l'argomento finale.
    \begin{lstlisting}[frame = single, label = {lst:lstlisting4-7.4}]
values := []int{1, 2, 3, 4}
fmt.Println(sum(values...))
    \end{lstlisting}
    Output:
    \begin{lstlisting}[language = bash, frame = L, label = {lst:lstlisting4-7.5}]
10
    \end{lstlisting}
    Sebbene il parametro \verb"...int" si comporti come una slice all'interno del corpo della funzione, il tipo di una funzione variadic è distinto dal tipo di una funzione con un parametro slice ordinario.
    \begin{lstlisting}[frame = single, label = {lst:lstlisting4-7.6}]
func f(...int) {}
func g([]int) {}

fmt.Printf("%T\n", f)
fmt.Printf("%T\n", g)
    \end{lstlisting}
    Output:
    \begin{lstlisting}[language = bash, frame = L, label = {lst:lstlisting4-7.7}]
func(...int)
func([]int)
    \end{lstlisting}
\end{document}
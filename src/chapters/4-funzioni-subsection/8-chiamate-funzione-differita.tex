\documentclass[../../thesis.tex]{subfiles}
\begin{document}
    \subsection{Chiamate con funzione differita}\label{subsec:chiamate-con-funzione-differita}
    Sintatticamente, un'istruzione \verb"defer" è una funzione ordinaria o una chiamata di metodo preceduta dalla parola chiave \verb"defer".
    Le espressioni di funzione e argomento vengono valutate quando l'istruzione viene eseguita, ma la chiamata effettiva viene \textit{differita} fino a quando la funzione che contiene l'istruzione \verb"defer" non è terminata, sia che si tratti normalmente di eseguire un'istruzione return, o anormalmente, dal panic.
    Qualsiasi numero di chiamate può essere differito;
    vengono eseguite al contrario dell'ordine in cui sono state differite.
    \hfill \vspace{12pt}

    Una dichiarazione di \verb"defer" viene spesso utilizzata con operazioni associate come aprire e chiudere, connettersi e disconnettersi o bloccare e sbloccare per garantire che le risorse vengano rilasciate in tutti i casi, indipendentemente dalla complessità del flusso di controllo.
    Il posto giusto per un'istruzione \verb"defer" che rilascia una risorsa è immediatamente dopo che la risorsa è stata acquisita con successo.
    \begin{lstlisting}[frame = single, label = {lst:lstlisting4-8.1}]
package ioutil

func ReadFile(filename string) ([]byte, error) {
    f, err := os.Open(filename)
    if err != nil {
        return nil, err
    }
    defer f.Close()
    return ReadAll(f)
}
    \end{lstlisting}
    \begin{lstlisting}[frame = single, label = {lst:lstlisting4-8.2}]
var mu sync.Mutex
var m = make(map[string]int)

func lookup(key string) int {
    mu.Lock()
    defer mu.Unlock()
    return m[key]
}
    \end{lstlisting}
    \begin{lstlisting}[frame = single, label = {lst:lstlisting4-8.3}]
func bigSlowOperation() {
    defer trace("bigSlowOperation")() // non si dimentichino le parentesi
                                      // extra per avviare la funzione
    time.Sleep(10 * time.Second)      // si simula un'operazione lenta
}

func trace(msg string) func() {
    start := time.Now()
    log.Printf("enter %s", msg)
    return func() { log.Printf("exit %s (%s)", msg, time.Since(start)) }
}
    \end{lstlisting}
    Output:
    \begin{lstlisting}[language = bash, frame = L, label = {lst:lstlisting4-8.4}]
2022/03/05 18:46:39 enter bigSlowOperation
2022/03/05 18:46:49 exit bigSlowOperation (10.000589217s)
    \end{lstlisting}
    Le funzioni differite vengono eseguite \textit{dopo} che le istruzioni di ritorno hanno aggiornato le variabili dei risultati della funzione.
    Poiché una funzione anonima può accedere alle variabili della sua funzione di racchiudimento, inclusi i risultati denominati, una funzione anonima differita può osservare i risultati della funzione.
    \begin{lstlisting}[frame = single, label = {lst:lstlisting4-8.5}]
func double(x int) (result int) {
    defer func() { fmt.Printf("double(%d) = %d\n", x, result) }()
    return x + x
}

func triple(x int) (result int) {
    defer func() { result += x }()
    return double(x)
}

func main() {
    fmt.Println(triple(4))
}
    \end{lstlisting}
    Output:
    \begin{lstlisting}[language = bash, frame = L, label = {lst:lstlisting4-8.6}]
double(4) = 8
12
    \end{lstlisting}
\end{document}
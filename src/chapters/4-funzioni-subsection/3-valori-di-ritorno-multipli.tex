\documentclass[../../thesis.tex]{subfiles}
\begin{document}
    \subsection{Valori di ritorno multipli}\label{subsec:valori-di-ritorno-multipli}
    Una funzione può restituire più di un risultato.
    Il risultato della chiamata di una funzione multi-valore è una tupla di valori.
    Il chiamante della funzione deve esplicitare l'assegnazione dei valori alle variabili nel caso una di esse verrà in seguito usata.
    \begin{lstlisting}[frame = single, label = {lst:lstlisting4-3.1}]
links, err := findLinks(url)
    \end{lstlisting}
    Per ignorare uno dei valori, assegnare uno di essi agli identificatori blank:
    \begin{lstlisting}[frame = single, label = {lst:lstlisting4-3.2}]
links, _ := findLinks(url) // gli errori sono ignorati
    \end{lstlisting}
    Il risultato di una chiamata multi-valore può essere a sua volta restituito da una funzione di chiamata (multi-valori), come in questa funzione:
    \begin{lstlisting}[frame = single, label = {lst:lstlisting4-3.3}]
func findLiksLog(url string) ([]string, error) {
    log.Printf("findLinks %s", url)
    return findLinks(url)
}
    \end{lstlisting}
    Una chiamata multi-valore può apparire come unico argomento quando si chiama una funzione con più parametri.
    Anche se raramente usato nel codice di produzione, questa caratteristica è a volte conveniente durante il debug in quanto ci permette di stampare tutti i risultati di una chiamata utilizzando una singola istruzione.
    Le due istruzioni di stampa qui sotto hanno lo stesso effetto.
    \begin{lstlisting}[frame = single, label = {lst:lstlisting4-3.4}]
log.Println(findLinks(url))

links, err := findLinks(url)
log.Println(links, err)
    \end{lstlisting}
    In una funzione con risultati denominati, gli operandi di un'istruzione return possono essere omessi.
    Questo è chiamato a \textit{bare return}.
    \begin{lstlisting}[frame = single, label = {lst:lstlisting4-3.5}]
func CountWordsAndImages(url string) (words, images int, err error) {
    resp, err := http.Get(url)
    if err != nil {
        return
    }
    doc, err := html.Parse(resp.Body)
    resp.Body.Close()
    if err != nil {
        err = fmt.Errorf("parsing HTML: %s", err)
        return
    }
    words, images = countWordAndImages(doc)
    return
}

func countWordsAndImages(n *html.Node) (words, images int) { /* ... */ }
    \end{lstlisting}
    Un brave return è un modo abbreviato per restituire ciascuna delle variabili di risultato nominate in ordine, quindi nella funzione sopra, ogni dichiarazione di ritorno è equivalente a:
    \begin{lstlisting}[frame = single, label = {lst:lstlisting4-3.6}]
return word, images, err
    \end{lstlisting}
    In funzioni come questa, con molte istruzioni di ritorno e diversi risultati, i bare return possono ridurre la duplicazione del codice, ma raramente rendono il codice più facile da capire.
\end{document}
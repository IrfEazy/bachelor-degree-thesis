\documentclass[../../thesis.tex]{subfiles}
\begin{document}
    \subsection{Errori}\label{subsec:errori}
    L'approccio di Go lo distingue da molti altri linguaggi in cui i fallimenti sono segnalati usando \textit{eccezioni}, non valori ordinari.
    Anche se Go ha un meccanismo di eccezione di sorta, è usato solo per segnalare errori veramente inaspettati che indicano un bug, non gli errori di routine che un programma robusto dovrebbe aspettarsi.
    \hfill \vspace{12pt}

    Una funzione per la quale il fallimento è un comportamento atteso restituisce un risultato aggiuntivo, convenzionalmente l'ultimo.
    Se il fallimento ha una sola possibile causa, il risultato è un booleano, di solito chiamato \verb"ok".
    Più spesso il fallimento può avere una varietà di cause per le quali il chiamante avrà bisogno di una spiegazione.
    In tali casi, il tipo di risultato aggiuntivo è \verb"error" (un tipo di interfaccia).
    \hfill \vspace{12pt}

    La ragione di questo progetto è che le eccezioni tendono ad impigliare la descrizione di un errore con il flusso di controllo richiesto per gestirlo, spesso portando ad un risultato indesiderato: gli errori di routine vengono segnalati all'utente finale sotto forma di una traccia di stack incomprensibile, pieno di informazioni sulla struttura del programma, ma privo di contesto intelligibile su ciò che è andato storto.
    \hfill \vspace{12pt}

    Al contrario, i programmi Go usano meccanismi di controllo di flusso ordinari come \verb"if" e \verb"return" per rispondere agli errori.
    Questo stile richiede innegabilmente una maggiore attenzione alla logica di gestione degli errori.
\end{document}
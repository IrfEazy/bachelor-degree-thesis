\documentclass[../../thesis.tex]{subfiles}
\begin{document}
    \subsection{L'interfaccia error}\label{subsec:l'interfaccia-error}
    Il tipo predichiarato \verb"error" è un tipo interfaccia con un singolo metodo che restituisce un messaggio d'errore:
    \begin{lstlisting}[frame = single,label={lst:lstlisting6-6.1}]
type error interface {
    Error() string
}
    \end{lstlisting}
    Il modo più semplice per creare un \verb"error" è invocando \verb"errors.New", che restituisce un nuovo \verb"error" per un dato messaggio d'errore.
    L'intero package \verb"errors" è lungo solo 4 righe:
    \begin{lstlisting}[frame = single,label={lst:lstlisting6-6.2}]
package errors

func New(text string) error { return &errorString{text} }

type errorString struct { text string }

func (e *errorString) Error() string { return e.text }
    \end{lstlisting}
    Il sottotipo di \verb"errorString" è una struct, non una stringa, a proteggere una sua rappresentazione da un aggiornamento involontario.
    La ragione per cui il puntatore al tipo \verb"*errorString" soddisfa l'interfaccia \verb"error" è perché si vuole che ogni chiamata a \verb"New" allochi un'istanza distinta di \verb"error" diversa da tutte le altre.
    \hfill \vspace{12pt}

    Le chiamate a \verb"errors.New" sono relativamente poco frequenti perché esiste una funzione wrapper più conveniente, \verb"fmt.Errorf", che permette pure la formattazione delle stringhe.
\end{document}
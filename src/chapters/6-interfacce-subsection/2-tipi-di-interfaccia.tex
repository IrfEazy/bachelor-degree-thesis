\documentclass[../../thesis.tex]{subfiles}
\begin{document}
    \subsection{Tipi di interfaccia}\label{subsec:tipi-di-interfaccia}
    Un tipo di interfaccia specifica un insieme di metodi che un tipo concreto deve possedere per essere considerato un'istanza di quell'interfaccia.
    \hfill \vspace{12pt}

    Un \verb"Reader" rappresenta qualsiasi tipo da cui è possibile leggere byte e un \verb"Closer" un qualsiasi valore che è possibile chiudere, come un file o una connessione di rete.
    \begin{lstlisting}[frame = single, label = {lst:lstlisting6-2.1}]
package io

type Reader interface {
    Read(p []byte) (n int, err error)
}

type Closer interface {
    Close() error
}
    \end{lstlisting}
    Si trovano anche dichiarazioni di nuovi tipi di interfaccia come combinazioni di quelli esistenti.
    \begin{lstlisting}[frame = single, label = {lst:lstlisting6-2.2}]
type ReadWriter interface {
    Reader
    Writer
}

type ReadWriteCloser interface {
    Reader
    Writer
    Closer
}
    \end{lstlisting}
    La sintassi usata sopra, che assomiglia all'embedding di struct, permette di definire un'altra interfaccia senza dover scrivere tutti i suoi metodi.
    Questa operazione è detta \textit{embedding} di un'interfaccia.
    Si poteva anche scrivere l'interfaccia \verb"io.ReadWriter" senza fare l'embedding, anche se in modo meno succinto, in questo modo:
    \begin{lstlisting}[frame = single, label = {lst:lstlisting6-2.3}]
type ReadWriter interface {
    Read(p []byte) (n int, err error)
    Write(p []byte) (n int, err error)
}
    \end{lstlisting}
    \clearpage
    \newpage
    o anche utilizzando un mix dei due stili:
    \begin{lstlisting}[frame = single, label = {lst:lstlisting6-2.4}]
type ReadWriter interface {
    Read(p []byte) (n int, err error)
    Writer
}
    \end{lstlisting}
    L'ordine in cui appaiono i metodi è irrilevante.
    Tutto ciò che conta è l'insieme dei metodi.
\end{document}
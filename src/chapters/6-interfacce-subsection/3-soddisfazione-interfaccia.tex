\documentclass[../../thesis.tex]{subfiles}
\begin{document}
    \subsection{Soddisfacimento dell'interfaccia}\label{subsec:soddisfacimento-dell'interfaccia}
    Un tipo \textit{soddisfa} un'interfaccia se possiede tutti i metodi che l'interfaccia richiede.
    \hfill \vspace{12pt}

    La regola di assegnabilità per le interfacce è molto semplice: un'espressione può essere assegnata a un'interfaccia solo se il suo tipo soddisfa l'interfaccia.
    \begin{lstlisting}[frame = single,label={lst:lstlisting6-3.1}]
var w io.Writer
w = os.Stdout         // OK: *os.File ha il metodo Write
w = new(bytes.Buffer) // OK: *bytes.Buffer ha il metodo Write
w = time.Second       // compile error: time.Duration non ha il metodo
                      // Write

var rwc io.ReadWriteCloser
rwc = os.Stdout         // OK: *os.File ha i metodi Read, Write, Close
rwc = new(bytes.Buffer) // compile error: *bytes.Buffer non ha il metodo
                        // Close
    \end{lstlisting}
    La regola si applica anche quando il lato destro è esso stesso un'interfaccia:
    \begin{lstlisting}[frame = single,label={lst:lstlisting6-3.2}]
w = rwc // OK: io.ReadWriteCloser ha il metodo Write
rwc = w // compile error: io.Writer non ha il metodo Close
    \end{lstlisting}
    Il tipo \verb"interface{}", detto \textit{interfaccia vuota}, è indispensabile.
    Poiché il tipo di interfaccia vuota non pone richieste sui tipi che lo soddisfano, possiamo assegnare \textit{qualsiasi} valore all'interfaccia vuota.
    \begin{lstlisting}[frame = single,label={lst:lstlisting6-3.3}]
var any interface{}
any = true
any = 12.34
any = "hello"
any = map[string]int{"one": 1}
any = new(bytes.Buffer)
    \end{lstlisting}
    Naturalmente, avendo creato un \verb"interface{}" contenente un valore booleano, float, string, map, puntatore, o qualsiasi altro tipo, non è possibile far nulla direttamente al valore che detiene poiché l'interfaccia non ha metodi.
    \hfill \vspace{12pt}

    Poiché il soddisfacimento dell'interfaccia dipende solo dai metodi dei due tipi coinvolti, non è necessario dichiarare la relazione tra un tipo concreto e le interfacce che soddisfa.
    \hfill \vspace{12pt}

    Tipi di interfaccia non vuoti come \verb"io.Writer" sono più spesso soddisfatti da un tipo di puntatore, in particolare quando uno o più dei metodi di interfaccia implica una sorta di mutazione al ricevitore, come fa il metodo \verb"Write".
    Un puntatore a una struttura è un tipo di metodo particolarmente comune.
    \hfill \vspace{12pt}

    Ma i tipi di puntatore non sono affatto gli unici tipi che soddisfano le interfacce, e anche le interfacce con i metodi mutatore possono essere soddisfatte da uno degli altri tipi di riferimento di Go.
    \hfill \vspace{12pt}

    Un tipo concreto può soddisfare molte interfacce non correlate.
    Immaginiamo di avere un programma che organizza o vende manufatti culturali digitalizzati come musica, film e libri.
    Si potrebbe definire la seguente serie di tipi concreti:
    \begin{lstlisting}[frame = single,label={lst:lstlisting6-3.4}]
Album
Book
Movie
Magazine
Podcast
TVEpisode
Track
    \end{lstlisting}
    Possiamo esprimere ogni astrazione di interesse come un'interfaccia.
    Alcune proprietà sono comuni a tutti gli artefatti, come il titolo, la data di creazione e la lista dei creatori (autori e artisti).
    \begin{lstlisting}[frame = single,label={lst:lstlisting6-3.5}]
type Artifact interface {
    Title() string
    Creators() []string
    Created() time.Time
}
    \end{lstlisting}
    Le altre proprietà sono ristrette ad alcuni dei tipi di artefatti.
    La prioprietà relative ai testi stampati sono rilevanti solo ai libri e ai magazine, mentre solo i film e gli episodi televisivi hanno la proprietà della risoluzione dello schermo.
    \begin{lstlisting}[frame = single,label={lst:lstlisting6-3.6}]
types Text interface {
    Pages() int
    Words() int
    PageSize() int
}

type Audio interface {
    Stream() (io.ReadCloser, error)
    RunningTime() time.Duration
    Format() string // p.e., "MP3", "WAV"
}

type Video interface {
    Stream() (io.ReadCloser, error)
    RunningTime() time.Duration
    Format() string // p.e., "MP4", "WMV"
    Resolution() (x, y int)
}
    \end{lstlisting}
    Queste interfacce permettono di aggregare tipi concreti e di esprimere gli aspetti che condividono.
    Per esempio, se bisogna gestire gli oggetti \verb"Audio" e \verb"Video" allo stesso modo, si può definire un'interfaccia \verb"Streamer" per rappresentare gli aspetti che hanno in comune senza cambiare alcun tipo dichiarato già esistente.
    \begin{lstlisting}[frame = single,label={lst:lstlisting6-3.7}]
type Streamer interface {
    Stream() (io.ReadCloser, error)
    RunningTime() time.Duration
    Format() string
}
    \end{lstlisting}
    Ogni gruppo di tipi concreti che si basano sui loro comportamenti condivisi possono essere espressi come un tipo di interfaccia.
    A differenza dei linguaggi fondati sulle classi, dove l'insieme delle classi che soddisfano le interfacce è esplicito, in Go si può definire una nuova astrazione o gruppi di interesse quando diventa necessario, senza andare a modificare le dichiarazioni dei tipi concreti.
    Questo è particolarmente utile quando il tipo concreto sta in un package scritto da un programmatore terzo.
\end{document}
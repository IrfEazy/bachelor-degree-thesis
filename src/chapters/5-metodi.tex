\documentclass[../thesis.tex]{subfiles}
\begin{document}
    \newpage


    \section{Metodi}\label{sec:metodi}
    Dall'inizio degli anni '90, la programmazione orientata agli oggetti (OOP) è stata il paradigma di programmazione dominante nell'industria e nell'istruzione, e quasi tutti i linguaggi ampiamente utilizzati sviluppati da allora hanno offerto supporto a questo approccio.
    Go non fa eccezione.
    \hfill \vspace{12pt}

    Anche se non esiste una definizione universalmente accettata di programmazione orientata agli oggetti, per i nostri scopi, un \textit{oggetto} è semplicemente un valore o una variabile che ha metodi, e un \textit{metodo} è una funzione associata a un particolare tipo.
    Un programma orientato agli oggetti è un programma che utilizza metodi per esprimere le proprietà e le operazioni di ogni struttura dati in modo che i client non debbano accedere direttamente alla rappresentazione dell'oggetto.
    \subfile{5-metodi-subsection/1-dichiarazioni-metodi}
    \subfile{5-metodi-subsection/2-metodi-ricevitore-puntatore}
    \subfile{5-metodi-subsection/3-tipi-composizione-struct-embedding}
    \subfile{5-metodi-subsection/4-valori-espressioni-metodo}
    \subfile{5-metodi-subsection/5-incapsulamento}
    \clearpage
\end{document}